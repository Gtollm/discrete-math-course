\documentclass[a4paper,12pt]{article}
\usepackage{mypreamble}

%% Page setup
\geometry{
    margin=2cm,
    includehead,
    % includefoot,
    headsep=\baselineskip,
}
\pagestyle{fancy}
\fancyfoot{}
\MakeDoubleHeader% {<l1>}{<l2>}{<r1>}{<r2>}
    {\TextHomeworkRus~\#1}
    {Теория множеств}
    {\TextDiscreteMathRus}
    {\IconFall~Осень 2022}

%% Add custom setup below


\begin{document}

\begin{tasks}
    \item Определите истинность заданных утверждений.
    Считайте, что $a \neq b$ -- урэлементы.

    \begin{multicols}{3}
    \begin{subtasks}
        \item $a \in \Set{\Set{a}, b}$
        \item $a \in \Set{a, \Set{b}}$
        \item $\Set{a} \in \Set{a, \Set{a}}$
        \item $\Set{a} \subset \Set{a, b}$
        \item $\Set{a} \subseteq \Set{\Set{a}, \Set{b}}$
        \item $\Set{\Set{a}} \subset \Set{\Set{a}, \Set{a, b}}$
        \item $\Set{\Set{a}, b} \subseteq \Set{a, \Set{a, b}, \Set{b}}$
        \item $\emptyset \in \emptyset$
        \item $\emptyset \subseteq \emptyset$
        \item $\emptyset \subset \emptyset$
        \item $\emptyset \in \Set{\emptyset}$
        \item $\emptyset \subseteq \Set{\Set{\emptyset}}$
        \item $\Set{\emptyset, \emptyset} \subset \Set{\emptyset}$
        \item $\Set{\Set{\emptyset}} \subset \Set{\Set{\emptyset}, \Set{\emptyset}}$
        \item $a \in 2^{\Set{a}}$
        \item $2^{\Set{a, \emptyset}} \subset 2^{\Set{a, b, \emptyset}}$
        \item $\Set{a, b} \subseteq 2^{\Set{a, b}}$
        \item $\Set{a, a} \in 2^{\Set{a, a}}$
        \item $\Set{\Set{a}, \emptyset} \subseteq 2^{\Set{a, a}}$
        \item $\Set{a, \Set{a}} \subset 2^{\Set{a, 2^{\Set{a}}}}$
        \item $\Set{\Set{a, \Set{\emptyset}}} \subseteq 2^{\Set{a, 2^{\emptyset}}}$
    \end{subtasks}
    \end{multicols}


    \item Дано множество-универсум\footnote{Здесь под универсумом имеется в виду множество доступных урэлементов. Считайте, что $\overline{A} = \universalset \setminus A$.} $\universalset = \Set{1, 2, \ldots, 10}$ и его подмножества:
    $A = \Set{x \given x \text{ -- чётное}}$,
    $B = \Set{x {\given} x \text{ -- простое\footnotemark}}$,
    $C = \Set{2, 4, 7, 9}$.
    \footnotetext{Считайте, что 1 \href{https://www.google.com/search?q=is 1 a prime number}{не является} простым числом.}%
    Нарисуйте диаграмму Венна для $A, B, C, \universalset$ и найдите:

    \begin{multicols}{3}
    \begin{subtasks}
        \item $B \symdiff (A \intersection C)$
        \item $\smash{\overline{B} \setminus (A \symdiff C)}$
        \item $\smash{\overline{A \union C} \union (C \symdiff B)}$
        \item $\smash{\card{\Set{A \union B \union 2^{\emptyset} \union 2^{\universalset}}}}$
        \item $\smash{(2^{A} \intersection 2^{C}) \setminus 2^{B}}$
        \item $\smash{2^{B \intersection C} \setminus \Set{2^{\card{2^{\Set{\emptyset}}}}, \card{\overline{B \intersection C}}}}$
    \end{subtasks}
    \end{multicols}


    \item Даны следующие множества\footnote{$\square$ -- самый обыкновенный квадрат, $\Cat$ -- самый обыкновенный кот.}:

    \begin{multicols}{3}
    \begin{items}
        \item $A = \Set{1, 2, 4}$
        \item $B = \Set{\square, \Cat} \union \emptyset$
        \item $C = 2^\emptyset \setminus \Set{\emptyset}$
        \item $D = \Set{\Cat, \card{2^{\Set{\emptyset, C}}}}$
        \item $E = 2^{A \setminus D} \intersection 2^{\Set{\card{B \setminus D}}}$
        \item $\mathrlap{F = 2^{\Set{
            \Set{\emptyset, \emptyset} \setminus \Set{\Set{\emptyset}},
            \Set{\emptyset} \symdiff C,
            \Set{\emptyset, C},
            2^{\emptyset}
        }}}$
    \end{items}
    \end{multicols}

    Найти:

    \begin{multicols}{3}
    \begin{subtasks}
        \item $A \symdiff D$
        \item $E \symdiff 2^{C}$
        \item $B \times E$
        \item $E \times 2^{B}$
        \item $D^{\card{C}}$
        \item $F^3$
    \end{subtasks}
    \end{multicols}


    \item Пусть $A = \Set{3, \card{B}}$, $B = \Set{1, \card{A}, \card{B}}$.
    Найдите, чему равны множества $A$ и $B$.


    \item Изобразите на графиках $\Real^2$ следующие множества точек:

    \begin{multicols}{2}
    \begin{subtasks}
        \item $\Set{1,2,3} \times [1; 3]$
        \item $[1; 4) \times (2; 4] \setminus \Set{\Pair{2, 3}}$
        \item $([1; 6] \times (1; 5]) \setminus ([4; 5] \times (2; 4))$
        \item $\Set{\Pair{x, y} \in [1; 5] \times [1; 4] \given (y > x) \lor (x \geq 4)}$
        \item $\Set{\Pair{x, y} \in (1;5]^2 \given 4(x-2)^2 + 9(y-3)^2 \leq 36}$
        \item $\Set{\Pair{x, y} \in \Natural^2 \given \exists z \in \Natural : x^3 + y^3 = z^3}$
    \end{subtasks}
    \end{multicols}


    \item Подробно докажите (или опровергните) следующие утверждения:

    \begin{subtasks}
        \item Если $A \subseteq B$ и $B \subseteq C$, то $A \subseteq C$.
        \item $\card{\powerset{A}} = 2^{\card{A}}$.
        \item Множество рациональных\footnote{Рациональное число можно представить в виде дроби $m / n$, где $m \in \Integer$ \--- целое, а $n \in \Natural$ \--- натуральное.} чисел $\Rational$ счётно.
        \item $\powerset{\Natural}$ \--- несчётное множество.
    \end{subtasks}

    % \item \ldots
\end{tasks}

\end{document}
