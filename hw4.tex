\documentclass[a4paper,12pt]{article}
\usepackage{mypreamble}

%% Page setup
\geometry{
    margin=2cm,
    includehead,
    % includefoot,
    headsep=\baselineskip,
}
\pagestyle{fancy}
\fancyfoot{}
\MakeDoubleHeader% {<l1>}{<l2>}{<r1>}{<r2>}
    {\TextHomeworkEng~\#4}
    {Formal Logic}
    {\TextDiscreteMathEng}
    {\IconFall~Fall 2023}

%% Add custom setup below

%% Formal proofs
\usepackage{fitch}
% https://www.mathstat.dal.ca/~selinger/fitch/fitchdoc.pdf
% \nddim{height}{topheight}{depth}{labelsep}{indent}{hsep}{justsep}{linethickness}
% Default: \nddim{4.5ex}{3.5ex}{1.5ex}{1em}{1.6em}{.5em}{2.5em}{.2mm}
\nddim%
    {4.0ex}% <height> [4.5ex]
    {3.3ex}% <topheight> [3.5ex]
    {1.2ex}% <depth> [1.5ex]
    {0.7em}% <labelsep> [1em]
    {0.7em}% <indent> [1.6em]
    {0.5em}% <hsep> [.5em]
    {2.0em}% <justsep> [2.5em]
    {0.8pt}% <linethickness> [.2mm]

\begin{document}
\selectlanguage{english}

\begin{tasks}
    % \item Construct complete truth tables for the given sentences. For each sentence, determine if it is a tautology, a contradiction, or a contingency. For each formula, draw its parse tree.

    % \begin{multicols}{2}
    % \begin{subtasks}
    %     \item $(A \implies B) \iff (B \implies A)$
    %     \item $((H \iff H) \implies H) \land \neg(H \implies H)$
    %     \item $\neg(R \land S \land T) \iff (R \lor S \lor T)$
    %     \item $(P \lor (Q \implies \neg P)) \land (P \implies \neg Q)$
    % \end{subtasks}
    % \end{multicols}


    \item For each given set of sentences, determine whether it is logically consistent (jointly satisfiable).

    \begin{multicols}{2}
    \begin{subtasks}
        \item $\neg D$, $(D \lor F)$, $\neg F$
        \item $(T \implies K)$, $\neg K$, $(K \lor \neg T)$
        \item $\neg(A \implies (\neg C \implies B))$, $((B \lor C) \land A)$
        \item $(C \implies B)$, $(D \lor C)$, $\neg B$, $(D \implies B)$
    \end{subtasks}
    \end{multicols}


    \item Complete the following deductive formal proofs by filling in missing formulae and justifications.

    \begin{multicols}{2}
    \begin{subtasks}
        \newcommand{\mybox}{%
            \tikz[baseline=0.5ex] \draw (0,0) rectangle (2em,2.5ex);%
        }

        \item \(\begin{nd}[t]
            \hypo{p1}{H \implies (R \land C)} \premise
            \hypo{p2}{\neg R \lor \neg C} \premise
            \have{a}{\neg(R \land C)} \by{\mybox}{}
            \have[\therefore]{c}{\mybox} \mt{p1,a}
        \end{nd}\)

        \item \(\begin{nd}[t]
            \hypo{p1}{K \land S} \premise
            \hypo{p2}{\neg K} \premise
            \have{a}{\mybox} \by{\mybox}{}
            \have{a}{\mybox} \by{\mybox}{}
            \have[\therefore]{c}{\neg S} \by{\mybox}{}
        \end{nd}\)

        \item \(\begin{nd}[t]
            \hypo{p}{A \implies \neg A} \premise
            \have[\,\vdots]{a}{\mybox} \by{(multiple lines)}{}
            \have[\therefore]{c}{\neg A} \by{LEM \mybox}{}
        \end{nd}\)

        \item \(\begin{nd}[t]
            \hypo{p}{(P \land Q) \lor (P \land R)} \premise
            \open
            \hypo{a1}{\mybox} \by{Assumption}{}
            \have{b1}{P} \by{\mybox}{}
            \close
            \open
            \hypo{a2}{\mybox} \by{Assumption}{}
            \have{b2}{P} \by{\mybox}{}
            \close
            \have[\therefore]{c}{P} \by{\mybox}{}
        \end{nd}\)
    \end{subtasks}
    \end{multicols}


    \item Symbolize the given arguments with well-formed formulae (WFFs) of propositional logic.
    For each argument, determine its validity using a truth table.
    For each \textit{valid} argument, provide a deductive formal proof\footnote{You can check your proofs at \url{https://proofs.openlogicproject.org}. Note that some inference rules may be missing here, \eg contraposition and commutativity \--- nevertheless, you are still allowed to use them in this task.} in Fitch notation.
    For each \textit{invalid} argument, provide a counterexample valuation.

    \begin{subtasks}
        % (P -> Q), ~Q \/ R, P \therefore R
        \item \textit{If philosophers ponder profound problems, their quandaries quell quotidian quibbles.
        Either their quandaries don't quell quotidian quibbles or right reasoning reveals reality (or both).
        Philosophers do ponder profound problems.
        Therefore, right reasoning reveals reality.}

        % A -> (~B \/ ~C), B, ~A \/ ~C \therefore ~A
        \item \textit{If aardvarks are adorable, then either baby baboons don't beat bongos or crocodiles can't consume cute capybaras (or both).
        Baby baboons beat bongos.
        Aardvarks aren't adorable unless crocodiles can't consume cute capybaras.
        Therefore, aardvarks aren't adorable.}

        % ~D -> G, D -> H \therefore G \/ H
        \item \textit{If discipline doesn't defeat deficiency, then geniuses generally get good grades.
        If discipline defeats deficiency, then homework has harmed humanity.
        Therefore, geniuses generally get good grades unless homework has harmed humanity.}

        % C -> ~I, (M /\ D) \/ C, I, \therefore M <-> D
        \item \textit{Crocodiles can consume cute capybaras only if incarcerating iguanas isn't illegal.
        Mad monkeys make mayhem and dinosaurs do disco dance, unless crocodiles consume cute capybaras.
        It is known that incarcerating iguanas is illegal.
        Therefore, dinosaurs do disco dance if and only if mad monkeys make mayhem.}
    \end{subtasks}


    \item For each given argument, construct a deductive proof in Fitch notation using only basic rules.

    \begin{multicols}{2}
    \begin{subtasks}
        \item $\neg\neg A \therefore A$
        \item $((A \implies B) \implies A) \therefore A$
        \item $(\neg B \implies \neg A) \therefore (A \implies B)$
        \item $\neg (A \lor B) \therefore (\neg A \land \neg B)$
        \item $(\neg A \land \neg B) \therefore \neg (A \lor B)$
        \item $(A \implies B) \land (\neg A \implies B) \therefore B$
    \end{subtasks}
    \end{multicols}


    \item For each given tautology, construct a deductive proof in Fitch notation.

    \begin{multicols}{2}
    \begin{subtasks}
        \item $(A \implies B) \lor (B \implies A)$
        \item $A \implies (B \implies A)$
        \item $(\neg B \implies \neg A) \implies ((\neg B \implies A) \implies B)$
        \item $(A \implies (B \implies C)) \implies ((A \implies B) \implies (A \implies C))$
    \end{subtasks}
    \end{multicols}

    % \item \ldots
\end{tasks}

\end{document}
