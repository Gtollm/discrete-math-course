\documentclass[a4paper,10pt]{article}
\usepackage{mypreamble}

%% Page setup
\geometry{
    margin=2cm,
    includehead,
    includefoot,
    headsep=4mm,
}
\pagestyle{fancy}
\fancyfoot{}
\MakeSingleHeader% {<l>}{<r>}
    {\TextCheatsheetEng: Boolean Algebra and Formal Logic}
    {\TextDiscreteMathEng, \IconFall~Fall 2021}
\fancyfoot[C]{\thepage\ of \zpageref{LastPage}}

%% Add custom setup below

% \titlespacing{\type}{left}{above}{below}[right]
\titlespacing{\section}{0pt}{*1}{*0.5}
\titlespacing{\subsection}{0pt}{*1}{*0.5}

\setlength{\tabcolsep}{0pt}

\colorlet{ColorValid}{green!70!black}
\colorlet{ColorInvalid}{red!70!black}

%% checkmark and crossmark symbols
\usepackage{pifont}
\newcommand{\cmark}{\text{\ding{51}}}
\newcommand{\xmark}{\text{\ding{55}}}

\newcommand{\Valid}{\textcolor{ColorValid}{\cmark}}
\newcommand{\Invalid}{\textcolor{ColorInvalid}{\xmark}}
\newcommand{\ValidArgument}{%
    \mathclap{\overset{\smash{\textcolor{ColorValid}{\textit{valid}}}}{\therefore}}}
\newcommand{\InvalidArgument}{%
    \mathclap{\overset{\smash{\textcolor{ColorInvalid}{\textit{invalid}}}}{\therefore}}}

% \input{tex/tikzlibrarykarnaugh.code}
% \kmunitlength=1cm

\newcommand\Vtextvisiblespace[1][.3em]{%
  \mbox{\kern.06em\vrule height.3ex}%
  \vbox{\hrule width#1}%
  \hbox{\vrule height.3ex}}

%% BNF grammar
\usepackage{syntax}

%% Formal proofs
\usepackage{fitch}
% https://www.mathstat.dal.ca/~selinger/fitch/fitchdoc.pdf
% \nddim{height}{topheight}{depth}{labelsep}{indent}{hsep}{justsep}{linethickness}
% Default: \nddim{4.5ex}{3.5ex}{1.5ex}{1em}{1.6em}{.5em}{2.5em}{.2mm}
\nddim%
    {3.6ex}% <height> [4.5ex]
    {3.0ex}% <topheight> [3.5ex]
    {1.0ex}% <depth> [1.5ex]
    {0.8em}% <labelsep> [1em]
    {0.5em}% <indent> [1.6em]
    {0.5em}% <hsep> [.5em]
    {2.0em}% <justsep> [2.5em]
    {0.8pt}% <linethickness> [.2mm]
% \nddim%
%     {4.5ex}% <height> [4.5ex]
%     {3.5ex}% <topheight> [3.5ex]
%     {1.5ex}% <depth> [1.5ex]
%     {0.8em}% <labelsep> [1em]
%     {0.5em}% <indent> [1.6em]
%     {0.5em}% <hsep> [.5em]
%     {2.0em}% <justsep> [2.5em]
%     {0.8pt}% <linethickness> [.2mm]

%% Not-iff
\def\niff{\nleftrightarrow}

%% Extra blackboard-like math font
\DeclareMathAlphabet{\mymathbb}{U}{bbold}{m}{n}

%% Autoscale missing font sizes
\usepackage{anyfontsize}

%% Silence some warnings
\WarningFilter{latexfont}{Font shape `OMS/LinuxLibertineT-TLF/m/n' undefined}
\WarningFilter{latexfont}{Some font shapes were not available, defaults substituted.}

%% Define inline list
\newlist{mylist}{enumerate*}{1}
\setlist[mylist]{label=(\arabic*)}


\begin{document}

\selectlanguage{english}

\setcounter{section}{2}
\section{Boolean Algebra and Formal Logic Cheatsheet}

\subsection{Definitions}

\begin{terms}
    \item \raggedright \textbf{Boolean function} is a function of the form $f: \Bool^n \to \Bool$, where $n \geq 0$ is the \emph{arity} of the function and $\Bool =\nobreak \Set{0, 1} =\nobreak \Set{\bot, \top} =\nobreak \Set{\texttt{F}, \texttt{T}}$ is a Boolean domain.

    \item There are multiple ways to represent a Boolean function (all examples represent the same function):
    \begin{enumerate}[left=6pt .. 18pt]
        \item Truth table, \eg $f = (1010)$, where LSB corresponds to $\mymathbb{0}$, MSB to $\mymathbb{1}$.
        \item Analytically (as a sentence of propositional logic), \eg $f(A,B) = \neg B$.
        \item Sum of minterms, \eg $f = \sum m(0,2) = m_0 + m_2$.\hfill\href{http://www.cs.ucr.edu/~ehwang/courses/cs120a/minterms.pdf}{Minterms}
        \item Product of maxterms, \eg $f = \prod M(1,3) = M_1 \cdot M_3$.\hfill\href{http://www.cs.ucr.edu/~ehwang/courses/cs120a/minterms.pdf#page=2}{Maxterms}
        \item Boolean function number, \eg $f^{2}_{10}$ is the 10-th 2-ary function.\hfill\href{https://mathworld.wolfram.com/BooleanFunction.html}{Boolean function} \\
        Note that Wolfram's \enquote{Boolean operator number} is a slightly different term, which uses the reversed truth table. 10-th Boolean function $f^{2}_{10}$ with the truth table $(1010)$ can be obtained via the query \enquote{5th Boolean function of 2~variables} (note: not 10th!) in WolframAlpha, since $\texttt{rev}(1010_2) = 0101_2 = 5_{10}$.\hfill\href{https://bit.ly/3nFbad5}{WolframAlpha}
        % \item Representative number, \eg $\#f = 1010$.
    \end{enumerate}

    % \item Minterms
    % \item Maxterms

    % \item TODO: more Boolean algebra definitions
\end{terms}


\newpage
\subsection{Propositional logic}

\begin{terms}
    \item \textbf{Proposition} is a statement which can be either true or false.
    \hfill\href{https://en.wikipedia.org/wiki/Truth-bearer}{Truth-bearer}

    \item \textbf{Alphabet} of propositional logic consists of
    \begin{mylist}
        \item atomic symbols and
        \item operator symbols.
    \end{mylist}
    % \hfill\href{https://en.wikipedia.org/wiki/Alphabet_(formal_languages)}{Alphabet}

    % \item \textbf{Braces:} ( and ).

    % \hfill\href{https://en.wikipedia.org/wiki/Formal_language}{Formal language}

    % https://en.wikipedia.org/wiki/Atomic_formula
    \item \textbf{Atomic formula} (\textbf{atom}) is an irreducible formula without logical connectives.
    \begin{terms}
        \item Propositional \textbf{variables}: $A, B, C, \dotsc, Z$. With indices, if needed: $A_1, A_2, \dotsc, Z_1, Z_2, \dotsc$.

        \item \textbf{Logical constants}: $\top$ for always true proposition (\textit{tautology}), $\bot$ for always false proposition (\textit{contradiction}).
    \end{terms}

    \item \textbf{Logical connectives} (\textbf{operators}):
    % \hfill\href{https://en.wikipedia.org/wiki/Logical_connective}{Logical connective}

    \begingroup
    \newcommand{\myA}{\mathcal{P}}
    \newcommand{\myB}{\mathcal{Q}}

    \begin{tabular}{l @{\hspace{1em}} l @{\hspace{2pt}} c}
    % \toprule
        \thead{Type} &
        \thead{Natural meaning} &
        \thead{Symbolization} \\
    \midrule
        \href{https://en.wikipedia.org/wiki/Negation}{Negation} &
        \makecell[lt]{
            It is not the case that $\myA$. \\[-\jot]
            It is false that $\myA$. \\[-\jot]
            It is not true that $\myA$.
        } &
        $\neg \myA$ \\
    \cmidrule(lr){1-3}
        \href{https://en.wikipedia.org/wiki/Logical_conjunction}{Conjunction} &
        \makecell[lt]{
            Both $\myA$ and $\myB$. \\[-\jot]
            $\myA$ but $\myB$. \\[-\jot]
            $\myA$, although $\myB$.
        } &
        $\myA \land \myB$ \\
    \cmidrule(lr){1-3}
        \href{https://en.wikipedia.org/wiki/Logical_disjunction}{Disjunction} &
        \makecell[lt]{
            Either $\myA$ or $\myB$ (or both). \\[-\jot]
            $\myA$ unless $\myB$.
        } &
        $\myA \lor \myB$ \\
    \cmidrule(lr){1-3}
        \href{https://en.wikipedia.org/wiki/Exclusive_or}{Exclusive or (Xor)} &
        \makecell[lt]{
            Either $\myA$ or $\myB$ (but not both). \\[-\jot]
            $\myA$ xor $\myB$.
        } &
        $\myA \xor \myB$ \\
    \cmidrule(lr){1-3}
        \href{https://en.wikipedia.org/wiki/Material_conditional}{\makecell[lt]{Implication \\[-\jot] (Conditional)}} &
        \makecell[lt]{
            If $\myA$, then $\myB$. \\[-\jot]
            $\myA$ only if $\myB$. \\[-\jot]
            $\myB$ if $\myA$.
        } &
        $\myA \implies \myB$ \\
    \cmidrule(lr){1-3}
        \href{https://en.wikipedia.org/wiki/Logical_biconditional}{Biconditional} &
        \makecell[lt]{
            $\myA$, if and only if $\myB$. \\[-\jot]
            $\myA$ iff $\myB$. \\[-\jot]
            $\myA$ just in case $\myB$.
        } &
        $\myA \iff \myB$ \\
    % \bottomrule
    \end{tabular}
    \vspace{1pt}
    \endgroup

    \item \textbf{Sentence} of propositional logic is defined inductively:
    \hfill\href{https://en.wikipedia.org/wiki/Well-formed_formula}{Well-formed formula (WFF)}

    \begin{enumerate}[left=6pt .. 18pt]
        \item Every propositional variable/constant is a sentence.

        \item If $\mathcal{A}$ is a sentence, then $\neg\mathcal{A}$ is a sentence.

        \item If $\mathcal{A}$ and $\mathcal{B}$ are sentences, then $(\mathcal{A} \land \mathcal{B})$, $(\mathcal{A} \lor \mathcal{B})$, $(\mathcal{A} \implies \mathcal{B})$, $(\mathcal{A} \iff \mathcal{B})$ are sentences.

        \item Nothing else is a sentence.
    \end{enumerate}

    \item Well-formed formulae grammar:
    \hfill\href{https://en.wikipedia.org/wiki/Backus-Naur_form}{Backus-Naur form (BNF)}

    \vspace{-2pt}
    \setlength{\grammarparsep}{0pt plus 4pt}
    \setlength{\grammarindent}{6em}
    \begin{grammar}
        <sentence> ::=
             <constant>
        \alt <variable>
        \alt $\neg$ <sentence>
        \alt `(' <sentence> <binop> <sentence> `)'

        <constant> ::=
            $\top$ | $\bot$

        <variable> ::=
            $A$ | \dots | $Z$ | $A_1$ | \dots | $Z_n$

        <binop> ::=
            $\land$ | $\lor$ | $\xor$ | $\implies$ | $\impliedby$ | $\iff$
    \end{grammar}

    \item \textbf{Literal} is a propositional variable or its negation: $\mathcal{L}_i = X_i$ (\textit{positive} literal), $\mathcal{L}_j = \neg X_j$ (\textit{negative} literal).

    \item \textbf{Argument} is a set of logical statements, called \emph{premises}, intended to support or infer a claim (\emph{conclusion}):
    % \hfill\href{https://en.wikipedia.org/wiki/Argument}{Argument}
    \[
        \underbrace{
            \mathcal{A}_1, \mathcal{A}_2, \dotsc, \mathcal{A}_n
        }_{\textit{premises}}
        \quad
        \begin{tikzpicture}[remember picture,overlay,on grid,baseline={(therefore.base)},inner sep=0pt]
            \node (therefore) {$\therefore$};
            \node (label) [below left=11mm and 5mm of therefore] {\small\enquote{therefore}};
            \draw[->,>={Stealth[]},shorten >=2pt] (label.north) to[out=90,in=-100,looseness=1.3] (therefore.south);
        \end{tikzpicture}
        \vrule height 0pt depth 11mm width 0pt\relax
        ~~
        \underbrace{
            \mathcal{C}_{\vphantom{n}}
        }_{\mathclap{\textit{conclusion}}}
    \]

    \item An argument is \textbf{valid} if whenever all the premises are true, the conclusion is also true.
    \hfill\href{https://en.wikipedia.org/wiki/Validity_(logic)}{Validity}

    \item An argument is \textbf{invalid} if there is a case (\textit{a counterexample)} when all the premises are true, but the conclusion is false.
\end{terms}


\newpage
\subsection{Semantics of Propositional Logic}

\begin{terms}
    \item \textbf{Valuation} is any assignment of truth values to propositional variables.
    \hfill\href{https://en.wikipedia.org/wiki/Valuation_(logic)}{Valuation}

    \item $\mathcal{A}$ is a \textbf{tautology} iff it is true on every valuation. Might be symbolized as \enquote{\,${} \entails \mathcal{A}$\,}.
    \hfill\href{https://en.wikipedia.org/wiki/Tautology_(logic)}{Tautology}

    \item $\mathcal{A}$ is a \textbf{contradiction} iff it is false on every valuation. Might be symbolized as \enquote{\,$\mathcal{A} \entails {}$\,}.
    \hfill\href{https://en.wikipedia.org/wiki/Contradiction}{Contradiction}

    \item $\mathcal{A}$ is a \textbf{contingency} iff it might be true on some valuation and false on another. In other words, a \textbf{contingent} proposition is neither a tautology nor a contradiction.
    \hfill\href{https://en.wikipedia.org/wiki/Contingency_(philosophy)}{Contingency}

    \item $\mathcal{A}$ and $\mathcal{B}$ are \textbf{equivalent} (symbolized as $\mathcal{A} \equiv \mathcal{B}$) iff, for every valuation, their truth values agree, \ie there is no valuation in which they have opposite truth values.
    \hfill\href{https://en.wikipedia.org/wiki/Logical_equivalence}{Equivalence}

    \item $\mathcal{A}_1, \mathcal{A}_2, \dotsc, \mathcal{A}_n$ are \textbf{consistent} (\textbf{jointly satisfiable}) iff there is \emph{some} valuation which makes them all true. Sentences are \textbf{inconsistent} (\textbf{jointly unsatisfiable}) iff there is \emph{no} valuation that makes them all true.
    \hfill\href{https://en.wikipedia.org/wiki/Consistency}{Consistency}

    \item The sentences $\mathcal{A}_1, \mathcal{A}_2, \dotsc, \mathcal{A}_n$ \textbf{entail} the sentence $\mathcal{C}$ (symbolized as $\mathcal{A}_1, \mathcal{A}_2, \dotsc, \mathcal{A}_n \entails \mathcal{C}$) if there is no valuation which makes all of $\mathcal{A}_1, \mathcal{A}_2, \dotsc, \mathcal{A}_n$ true and $\mathcal{C}$ false.
    \hfill\href{https://en.wikipedia.org/wiki/Entailment}{Entailment}

    \item If $\mathcal{A}_1, \mathcal{A}_2, \dotsc, \mathcal{A}_n \entails \mathcal{C}$, then the argument $\mathcal{A}_1, \mathcal{A}_2, \dotsc, \mathcal{A}_n \therefore \mathcal{C}$ is \textbf{valid}.
\end{terms}

\begin{figure}[H]
\setstretch{0.7}
\textit{Validity check examples}:

\vspace{6pt}
\(\begin{array}{
    @{~}
    c @{~~} c % Arguments: A,B
    % A \implies B, A \therefore B
    @{~~~} |
    @{~~~} c @{\,} B @{\,} c % A \implies B
    @{~~~} B % A
    @{~~~} c % therefore
    @{~~~} B % B
    % \neg A \implies \neg B \therefore B \implies A
    @{~~~} |
    @{~~~} c @{} c @{\,} B @{\,} c @{} c % \neg A \implies \neg B
    @{~~~} c % therefore
    @{~~~} c @{\,} B @{\,} c % B \implies A
    % A \implies B, B \therefore \neg (B \implies A)
    @{~~~} |
    @{~~~} c @{\,} B @{\,} c % A \implies B
    @{~~~} B % B
    @{~~~} c % therefore
    @{~~~} B @{\,} c @{\,} c @{\,} c % \neg (B \implies A)
    @{~}
}
    A&B  &
    A&\implies&B & A &\ValidArgument& B  &
    \neg&A&\implies&\neg&B &\ValidArgument& B&\implies&A  &
    A&\implies&B & B &\InvalidArgument& \neg&(B&\implies&A) \BotStrut\\
    \hline\TopStrut
    % print('\n'.join(f'{a}&{b}  &  {a}&{not a or b}&{b} & {a} &\\--& {b}  &  {not a}&{a}&{a or not b}&{not b}&{b} &\\--& {b}&{not b or a}&{a}  &  {a}&{not a or b}&{b} & {b} &\\--& {not(not b or a)}&{b}&{not b or a}&{a} \\\\'.replace('True', '1').replace('False', '0') for a,b in [(False, False), (False, True), (True, False), (True, True)]))
    0&0  &
    &1& & 0 &\cdot& 0  &
    1&&1&1& &\Valid& &1&  &
    &1& & 0 &\cdot& 0&&1& \\
    0&1  &
    &1& & 0 &\cdot& 1  &
    1&&0&0& &\cdot& &0&  &
    &1& & 1 &\Valid& 1&&0& \\
    1&0  &
    &0& & 1 &\cdot& 0  &
    0&&1&1& &\Valid& &1&  &
    &0& & 0 &\cdot& 0&&1& \\
    1&1  &
    &1& & 1 &\Valid& 1  &
    0&&1&0& &\Valid& &1&  &
    &1& & 1 &\Invalid& 0&&1& \\
\end{array}\)

\vspace{8pt}
\(\begin{array}{
    @{~}
    c @{~~} c @{~~} c % Arguments: R,S,T
    % R \lor S, S \lor T, \neg R \therefore S \land T
    @{~~~} |
    @{~~~} c @{\,} B @{\,} c % R \lor S
    @{~~~} c @{\,} B @{\,} c % S \lor T
    @{~~~} B @{\,} c % \neg R
    @{~~~} c % therefore
    @{~~~} c @{\,} B @{\,} c % S \land T
    % (R \land S) \implies T \therefore R \implies (S \implies T)
    @{~~~} |
    @{~~~} c @{\,} c @{\,} c @{\,} B @{\,} c % (R \land S) \implies T
    @{~~~} c % therefore
    @{~~~} c @{\,} B @{\,} c @{\,} c @{\,} c % R \implies (S \implies T)
    @{~}
}
    R&S&T  &
    R&\lor&S & S&\lor&T & \neg&R &\InvalidArgument& S&\land&T  &
    (R&\land&S)&\implies&T &\ValidArgument& R&\implies&(S&\implies&T) \BotStrut\\
    \hline\TopStrut
    % print('\n'.join(f'{r}&{s}&{t}  &\n{r}&{r or s}&{s} & {s}&{s or t}&{t} & {not r}&{r} &\\--& {s}&{s and t}&{t}  &\n{r}&{r and s}&{s}&{not(r and s) or t}&{t} &\\--& {r}&{not r or (not s or t)}&{s}&{not s or t}&{t} \\\\'.replace('True','1').replace('False','0') for r in B for s in B for t in B))
    0&0&0  &
    &0& & &0& & 1& &\cdot& &0&  &
    &0&&1&0 &\Valid& 0&1&&1& \\
    0&0&1  &
    &0& & &1& & 1& &\cdot& &0&  &
    &0&&1&1 &\Valid& 0&1&&1& \\
    0&1&0  &
    &1& & &1& & 1& &\Invalid& &0&  &
    &0&&1&0 &\Valid& 0&1&&0& \\
    0&1&1  &
    &1& & &1& & 1& &\Valid& &1&  &
    &0&&1&1 &\Valid& 0&1&&1& \\
    1&0&0  &
    &1& & &0& & 0& &\cdot& &0&  &
    &0&&1&0 &\Valid& 1&1&&1& \\
    1&0&1  &
    &1& & &1& & 0& &\cdot& &0&  &
    &0&&1&1 &\Valid& 1&1&&1& \\
    1&1&0  &
    &1& & &1& & 0& &\cdot& &0&  &
    &1&&0&0 &\cdot& 1&0&&0& \\
    1&1&1  &
    &1& & &1& & 0& &\cdot& &1&  &
    &1&&1&1 &\Valid& 1&1&&1& \\
\end{array}\)
\end{figure}


\newpage
\subsection{Natural Deduction Rules}

\colorlet{colorbasicrule}{green!45!black}
\colorlet{colorderivedrule}{orange!70!black}

\fancypagestyle{naturaldeductionrules}{
    \fancyfoot[L]{\textcolor{colorbasicrule}{\textbf{Green:} basic rules.} \textcolor{colorderivedrule}{\textbf{Orange:} derived rules.}}
}
\thispagestyle{naturaldeductionrules}

\begingroup
\newcommand{\myA}{\mathcal{A}}
\newcommand{\myB}{\mathcal{B}}
\newcommand{\myC}{\mathcal{C}}
\newcommand{\midlineheight}{.7\baselineskip}
\newcommand{\extratopskip}{8pt}
\newcommand{\myheight}{%
    \dimexpr\textheight-16pt-\extratopskip\relax}

% \hypo{<line label>}{<formula>}: line with horizontal bar
% \have{<line label>}{<formula>}: line without horizontal bar
% \open: opens a subproof
% \close: closes a subproof
% \by{<rule}{<line labels}
% \hypo[<symbol>][<offset>]{<line label>}{<formula>}
% \have[<symbol>][<offset>]{<line label>}{<formula>}

\tcbset{
    formalbox/.style={% with arg #1 = <title>
        title={\strut #1},
        fonttitle=\bfseries,
        size=title,
        boxsep=2pt,
        top=0pt,
        bottom=1mm,
        leftupper=0pt,
        rightupper=0pt,
        %%% draft,
    },
    myoverlay/.style={% with arg #1 = <text>
        enhanced,
        overlay={
            \path[
                draw=blue,
                fill=blue!75!white,
                opacity=0.8,
            ] (frame.north east)
            ++(0,-0.8)
            ++(-0.1mm,-0.1mm)
            -- ++(-0.8,0.8)
            -- ++(-0.7,0)
            -- ++(1.5,-1.5)
            -- cycle;
            \path (frame.north east) ++(-0.57,-0.57)
            node[white,rotate=-45] {#1};
        },
    },
}

\newtcolorbox{formalboxbasic}[2][]{% [<style>]{<title>}
    formalbox=#2,
    % myoverlay={Basic},
    colback=green!5,
    colframe=colorbasicrule,
    #1
}
\newtcolorbox{formalboxderived}[2][]{% [<style>]{<title>}
    formalbox=#2,
    % myoverlay={Derived},
    colback=orange!5,
    colframe=colorderivedrule,
    #1
}

\makeatletter
\newcommand{\DrawLine}[1]{% {<color>}
    \vrule height \midlineheight depth \dimexpr \midlineheight-1ex\relax width 0pt\relax
    \begin{tikzpicture}
        \path[use as bounding box] (0,0) -- (\linewidth,0);
        \draw[color=#1,dashed,dash phase=5pt]
            (0-\kvtcb@leftupper-\kvtcb@boxsep,0)
            --(\linewidth+\kvtcb@rightupper+\kvtcb@boxsep,0);
    \end{tikzpicture}}
\makeatother

\vskip\extratopskip

\begin{adjustbox}{minipage=[t][\myheight]{.3\linewidth},valign=t}
    \begin{formalboxbasic}{Reiteration}
        \begin{flusheqs}
        \begin{nd}
            \have[m]{A}{\myA}
            \have[~]{rA}{\myA} \r{A}
        \end{nd}
        \end{flusheqs}
    \end{formalboxbasic}
    \vfill
    \begin{formalboxbasic}{Modus ponens}
        \begin{flusheqs}
        \begin{nd}
            \have[i]{AimpliesB}{\myA \implies \myB}
            \have[j]{A}{\myA}
            \have[~]{B}{\myB} \mp{AimpliesB,A}
        \end{nd}
        \end{flusheqs}
    \end{formalboxbasic}
    \vfill
    \begin{formalboxderived}{Modus tollens}
        \begin{flusheqs}
        \begin{nd}
            \have[i]{AimpliesB}{\myA \implies \myB}
            \have[j]{negB}{\neg\myB}
            \have[~]{negA}{\neg\myA} \mt{AimpliesB,negB}
        \end{nd}
        \end{flusheqs}
    \end{formalboxderived}
    \vfill
    \begin{formalboxbasic}{Negation}
        \begin{flusheqs}
        \begin{nd}
            \open
                \hypo[i]{hypA}{\myA}
                \have[j]{bot}{\bot}
            \close
            \have[~]{negA}{\neg\myA} \ni{hypA-bot}
        \end{nd}
        \end{flusheqs}
        \DrawLine{tcbcolframe}
        \begin{flusheqs}
        \begin{nd}
            \have[i]{negA}{\neg\myA}
            \have[j]{A}{\myA}
            \have[~]{bot}{\bot} \ne{negA,A}
        \end{nd}
        \end{flusheqs}
    \end{formalboxbasic}
    \vfill
    \begin{formalboxderived}{Double negation}
        \begin{flusheqs}
        \begin{nd}
            \have[m]{negnegA}{\neg\neg\myA}
            \have[~]{A}{\myA} \nne{negnegA}
        \end{nd}
        \end{flusheqs}
    \end{formalboxderived}
    \vfill
    \begin{formalboxbasic}{Indirect proof}
        \begin{flusheqs}
        \begin{nd}
            \open
                \hypo[i]{negA}{\neg\myA}
                \have[j]{bot}{\bot}
            \close
            \have[~]{A}{\myA} \ip{negA-bot}
        \end{nd}
        \end{flusheqs}
    \end{formalboxbasic}
    \vfill
    \begin{formalboxderived}{Excluded middle}
        \begin{flusheqs}
        \begin{nd}
            \open
                \hypo[i]{A}{\myA}
                \have[j]{BfromA}{\myB}
            \close
            \open
                \hypo[k]{negA}{\neg\myA}
                \have[l]{BfromNegA}{\myB}
            \close
            \have[~]{B}{\myB} \lem{A-BfromA,negA-BfromNegA}
        \end{nd}
        \end{flusheqs}
    \end{formalboxderived}
\end{adjustbox}%
\hfill%
\begin{adjustbox}{minipage=[t][\myheight]{.3\linewidth},valign=t}
    \begin{formalboxbasic}{Explosion}
        \begin{flusheqs}
        \begin{nd}
            \have[m]{bot}{\bot}
            \have[~]{A}{\myA} \x{bot}
        \end{nd}
        \end{flusheqs}
    \end{formalboxbasic}
    \vfill
    \begin{formalboxbasic}{Conjunction}
        \begin{flusheqs}
        \begin{nd}
            \have[i]{A}{\myA}
            \have[j]{B}{\myB}
            \have[~]{AandB}{\myA \land \myB} \ai{A,B}
        \end{nd}
        \end{flusheqs}
        \DrawLine{tcbcolframe}
        \begin{flusheqs}
        \begin{nd}
            \have[m]{AandB}{\myA \land \myB}
            \have[~]{A}{\myA} \ae{AandB} \ae{AandB}
            \have[~]{B}{\myB} \ae{AandB} \ae{AandB}
        \end{nd}
        \end{flusheqs}
    \end{formalboxbasic}
    \vfill
    \begin{formalboxbasic}{Disjunction}
        \begin{flusheqs}
        \begin{nd}
            \have[m]{A}{\myA}
            \have[~]{AorB}{\myA \lor \myB} \oi{A}
        \end{nd}
        \end{flusheqs}
        \DrawLine{tcbcolframe}
        % \begin{flusheqs}
        % \begin{nd}
        %     \have[m]{A}{\myA}
        %     \have[~]{BorA}{\myB \lor \myA} \oi{A}
        % \end{nd}
        % \end{flusheqs}
        % \DrawLine{tcbcolframe}
        \begin{flusheqs}
        \begin{nd}
            \have[m]{AorB}{\myA \lor \myB}
            \open
                \hypo[i]{hypA}{\myA}
                \have[j]{CfromA}{\myC}
            \close
            \open
                \hypo[k]{hypB}{\myB}
                \have[l]{CfromB}{\myC}
            \close
            \have[~]{C}{\myC} \oe{AorB,hypA-CfromA,hypB-CfromB}
        \end{nd}
        \end{flusheqs}
    \end{formalboxbasic}
    \vfill
    \begin{formalboxderived}{Disjunctive syllogism}
        \begin{flusheqs}
        \begin{nd}
            \have[i]{AorB}{\myA \lor \myB}
            \have[j]{negA}{\neg\myA}
            \have[~]{B}{\myB} \ds{AorB,negA}
        \end{nd}
        \end{flusheqs}
        \DrawLine{tcbcolframe}
        \begin{flusheqs}
        \begin{nd}
            \have[i]{AorB}{\myA \lor \myB}
            \have[j]{negB}{\neg\myB}
            \have[~]{A}{\myA} \ds{AorB,negB}
        \end{nd}
        \end{flusheqs}
    \end{formalboxderived}
    \vfill
    \begin{formalboxderived}{Hypothetical syllogism}
        \begin{flusheqs}
        \begin{nd}
            \have[i]{AimpliesB}{\myA \implies \myB}
            \have[j]{BimpliesC}{\myB \implies \myC}
            \have[~]{AimpliesC}{\myA \implies \myC} \hs{AimpliesB,BimpliesC}
        \end{nd}
        \end{flusheqs}
    \end{formalboxderived}
\end{adjustbox}%
\hfill%
\begin{adjustbox}{minipage=[t][\myheight]{.3\linewidth},valign=t}
    \begin{formalboxbasic}{Conditional}
        \begin{flusheqs}
        \begin{nd}
            \open
                \hypo[i]{hypA}{\myA}
                \have[j]{BfromA}{\myB}
            \close
            \have[~]{AimpliesB}{\myA \implies \myB} \ii{hypA-BfromA}
        \end{nd}
        \end{flusheqs}
    \end{formalboxbasic}
    \vfill
    % \begin{formalboxderived}{Implication}
    %     \begin{flusheqs}
    %     \begin{nd}
    %         \have[m]{AimpliesB}{\myA \implies \myB}
    %         \have[~]{imp}{\neg\myA \lor \myB} \imp{AimpliesB}
    %     \end{nd}
    %     \end{flusheqs}
    % \end{formalboxderived}
    % \vfill
    \begin{formalboxderived}{Contraposition}
        \begin{flusheqs}
        \begin{nd}
            \have[m]{AimpliesB}{\myA \implies \myB}
            \have[~]{contra}{\neg\myB \implies \neg\myA} \contra{AimpliesB}
        \end{nd}
        \end{flusheqs}
    \end{formalboxderived}
    \vfill
    \begin{formalboxbasic}{Biconditional}
        \begin{flusheqs}
        \begin{nd}
            \open
                \hypo[i]{hypA}{\myA}
                \have[j]{BfromA}{\myB}
            \close
            \open
                \hypo[k]{hypB}{\myB}
                \have[l]{AfromB}{\myA}
            \close
            \have[~]{AiffB}{\myA \iff \myB} \ei{hypA-BfromA,hypB-AfromB}
        \end{nd}
        \end{flusheqs}
        \DrawLine{tcbcolframe}
        \begin{flusheqs}
        \begin{nd}
            \have[i]{AiffB}{\myA \iff \myB}
            \have[j]{A}{\myA}
            \have[~]{B}{\myB} \ee{AiffB,A}
        \end{nd}
        \end{flusheqs}
        \DrawLine{tcbcolframe}
        \begin{flusheqs}
        \begin{nd}
            \have[i]{AiffB}{\myA \iff \myB}
            \have[j]{B}{\myB}
            \have[~]{A}{\myA} \ee{AiffB,B}
        \end{nd}
        \end{flusheqs}
    \end{formalboxbasic}
    \vfill
    \begin{formalboxderived}{De Morgan Rules}
        \begin{flusheqs}
        \begin{nd}
            \have[m]{negAorB}{\neg(\myA \lor \myB)}
            \have[~]{negAnegB}{\neg\myA \land \neg\myB} \dm{negAorB}
        \end{nd}
        \end{flusheqs}
        \DrawLine{tcbcolframe}
        \begin{flusheqs}
        \begin{nd}
            \have[m]{negAnegB}{\neg\myA \land \neg\myB}
            \have[~]{negAorB}{\neg(\myA \lor \myB)} \dm{negAnegB}
        \end{nd}
        \end{flusheqs}
        \DrawLine{tcbcolframe}
        \begin{flusheqs}
        \begin{nd}
            \have[m]{negAandB}{\neg(\myA \land \myB)}
            \have[~]{negAornegB}{\neg\myA \lor \neg\myB} \dm{negAandB}
        \end{nd}
        \end{flusheqs}
        \DrawLine{tcbcolframe}
        \begin{flusheqs}
        \begin{nd}
            \have[m]{negAornegB}{\neg\myA \lor \neg\myB}
            \have[~]{negAandB}{\neg(\myA \land \myB)} \dm{negAornegB}
        \end{nd}
        \end{flusheqs}
    \end{formalboxderived}
    \vfill
    More rules can be found in the \\
    \href{https://forallx.openlogicproject.org}{\enquote{forall~x:~Calgary}} book (\href{https://forallx.openlogicproject.org/forallxyyc.pdf#page=416}{p.~406}).
\end{adjustbox}

\endgroup


\newpage
\subsection{Normal Forms}

\begin{terms}
    \item \emph{Disjunctive} forms:
    \begin{terms}
        \item \textbf{Term} is a conjunction of literals: $\mathcal{T} = \biglandnolim{i} \mathcal{L}_i$.

        \item Formula is in \textbf{disjunctive normal form (DNF)} if it is a disjunction of terms: $\textrm{DNF} = \biglornolim{i} \mathcal{T}_i$.

        \item \textbf{Minterm} is conjunction of literals, where \emph{each} variable appears \emph{once}, \eg $m_6 = (A \land B \land \neg C)$.

        \item Formula is in \textbf{canonical DNF (CDNF)} if it is a disjunction of minterms: $\textrm{CDNF} = \biglornolim{i} m_i$.
    \end{terms}

    \item \emph{Conjunctive} forms:
    \begin{terms}
        \item \textbf{Clause} is a disjunction of literals: $\mathcal{C} = \biglornolim{i} \mathcal{L}_i$

        \item Formula is in \textbf{conjunctive normal form (CNF)} if it is a conjunction of clauses: $\textrm{CNF} = \biglandnolim{i} \mathcal{C}_i$.

        \item \textbf{Maxterm} is disjunction of literals, where \emph{each} variable appears \emph{once}, \eg $M_6 = (\neg A \lor \neg B \lor C)$.

        \item Formula is in \textbf{canonical CNF (CCNF)} if it is a conjunction of maxterms: $\textrm{CCNF} = \biglandnolim{i} M_i$.
    \end{terms}

    \item Some other normal forms:
    \begin{terms}
        \item Formula is in \textbf{negation normal form (NNF)} if the negation operator ($\neg$) is only applied to variables and the only other allowed Boolean operators are conjunction ($\land$) and disjunction ($\lor$).

        \item Formula $f$ is in \textbf{Blake canonical form (BCF)} if it is a disjunction of all the \textit{prime implicants} of $f$.

        \item Formula is in \textbf{prenex normal form (PNF)} if it consists of \textit{prefix} \--- quantifiers and bound variables, and \textit{matrix} \--- quatifier-free part.

        \item Formula is in \textbf{Skolem normal form (SNF)} if it is in prenex normal form with only universal first-order quantifiers.

        \item \textbf{Zhegalkin polynomial} is a formula in the following form (\textbf{algebraic normal form (ANF)}):
        \begin{terms}
            \item \(
                f(X_1, \dotsc, X_n) = a_0 \xor
                \bigxorclap{\substack{
                    1 \leq i_1 \leq \dotsb \leq i_k \leq n \\
                    1 \leq k \leq n }}
                (a_{i_1, \dotsc, i_k} \land X_{i_1} \land \dots \land X_{i_k}),
                \text{ where } a_0, a_{i_1, \dotsc, i_k} \in \Bool
            \)

            \item \(
                f(x_1, \dotsc, x_n) = a_0 \xor (a_1 x_1 \xor \dotsb \xor a_n x_n) \xor (a_{1,2} x_1 x_2 \xor \dotsb \xor a_{n-1,n} x_{n-1} x_n) \xor \dotsb \xor a_{1,\dotsc,n} x_1 \dots x_n
            \)
        \end{terms}
    \end{terms}
\end{terms}


% \newpage
\subsection{Conversion to CNF/DNF}

In order to convert \emph{arbitrary} (\ie \emph{any}) Boolean formula to \emph{equivalent} CNF/DNF:
\begin{enumerate}[topsep=2pt, itemsep=2pt]
    \item Eliminate equivalences:
    \par$\mathcal{A} \iff \mathcal{B} ~\rightsquigarrow~ (\mathcal{A} \implies \mathcal{B}) \land (\mathcal{B} \implies \mathcal{A})$

    \item Eliminate implications:
    \par$\mathcal{A} \implies \mathcal{B} ~\rightsquigarrow~ \neg\mathcal{A} \lor \mathcal{B}$

    \item Push negation downwards:
    \par\(\openup-\jot\begin{aligned}[t]
        \neg(\mathcal{A} \lor \mathcal{B}) &~\rightsquigarrow~ \neg\mathcal{A} \land \neg\mathcal{B} \\
        \neg(\mathcal{A} \land \mathcal{B}) &~\rightsquigarrow~ \neg\mathcal{A} \lor \neg\mathcal{B}
    \end{aligned}\)

    \item Eliminate double negation:
    \par$\neg\neg\mathcal{A} ~\rightsquigarrow~ \mathcal{A}$

    \item Push disjunction (for CNF) / conjunction (for DNF) downward:
    \par\(\openup-\jot\begin{aligned}[t]
        (\mathcal{A} \land \mathcal{B}) \lor \mathcal{C} ~\rightsquigarrow_{\textrm{CNF}}~ (\mathcal{A} \lor \mathcal{C}) \land (\mathcal{B} \lor \mathcal{C}) \\
        (\mathcal{A} \lor \mathcal{B}) \land \mathcal{C} ~\rightsquigarrow_{\textrm{DNF}}~ (\mathcal{A} \land \mathcal{C}) \lor (\mathcal{B} \land \mathcal{C}) \\
    \end{aligned}\)

    \item Eliminate $\top$ and $\bot$:
    \par\(\openup-\jot\begin{aligned}[t]
        \mathcal{A} \land \top &~\rightsquigarrow~ \mathcal{A} &\qquad&&
        \mathcal{A} \land \bot &~\rightsquigarrow~ \bot \\
        \mathcal{A} \lor \top &~\rightsquigarrow~ \top &\qquad&&
        \mathcal{A} \lor \bot &~\rightsquigarrow~ \mathcal{A} \\
        \neg\top &~\rightsquigarrow~ \bot &\qquad&&
        \neg\bot &~\rightsquigarrow~ \top
    \end{aligned}\)
\end{enumerate}


\newpage
\subsection{Functional Completeness}

\begin{terms}
    \item A set of Boolean functions $F$ is \textbf{functionally complete}, iff it contains:
    \begin{terms}
        \item at least one function that does \textit{not} preserve zero, \ie $\exists f \in F: f \notin T_0$, and
        \item at least one function that does \textit{not} preserve one, \ie $\exists f \in F: f \notin T_1$, and
        \item at least one function that is \textit{not} self-dual, \ie $\exists f \in F: f \notin S$, and
        \item at least one function that is \textit{not} monotonic, \ie $\exists f \in F: f \notin M$, and
        \item at least one function that is \textit{not} linear function, \ie $\exists f \in F: f \notin L$.
    \end{terms}

    \item A function $f$ is \textbf{zero-preserving} iff it is \texttt{False} on the zero-valuation: \\
    $f \in T_0 \iff f(\mymathbb{0}) = 0$

    \item A function $f$ is \textbf{one-preserving} iff it is \texttt{True} on the one-valuation: \\
    $f \in T_1 \iff f(\mymathbb{1}) = 1$

    \item A function $f$ is \textbf{self-dual} iff it is dual to itself: \\
    $f \in S \iff \forall x_1,\dotsc,x_n \in \Bool : f(x_1, \dotsc, x_n) = \overline{f}(\overline{x}_1, \dotsc, \overline{x}_n)$.

    \item A function $f$ is \textbf{monotonic} iff for every increasing valuations, the function does not decrease: \\
    $f \in M \iff \forall a,b \in \Bool^n: a \preceq b \implies f(a) \leq f(b)$.

    \begin{terms}
        \item Comparison of valuations $a,b \in \Bool^n$ is defined as follows: \\
        $a \preceq b \iff \biglandclap{1 \leq i \leq n} a_i \leq b_i$
    \end{terms}


    \item A function $f$ is \textbf{linear} iff its Zhegalkin polynomial is linear (\ie has a degree at most 1): \\
    $f \in L \iff \operatorname{deg} f_{\xor} \leq 1$
\end{terms}


% \subsection{Truth Tables}

% \begin{tabular}{
%     @{~}
%     C @{~} C
%     @{~~~} |
%     @{~~~} C @{\,} C
%     @{~~~} C @{\,} C @{\,} C
%     @{~~~} C @{\,} C @{\,} C
%     @{~~~} C @{\,} C @{\,} C
%     @{~~~} C @{\,} C @{\,} C
%     @{~~~} C @{\,} C @{\,} C
%     @{~}
% }
% \toprule
%     A & B &
%     \neg & A &
%     A & \land & B &
%     A & \lor & B &
%     A & \xor & B &
%     A & \implies & B &
%     A & \iff & B \BotStrut\\
% \hline\TopStrut
%     0 & 0 &
%     \textbf{1} & 0 &      % \neg
%     0 & \textbf{0} & 0 &  % \land
%     0 & \textbf{0} & 0 &  % \lor
%     0 & \textbf{0} & 0 &  % \xor
%     0 & \textbf{1} & 0 &  % \implies
%     0 & \textbf{1} & 0 \\ % \iff
%     0 & 1 &
%     \textbf{1} & 0 &      % \neg
%     0 & \textbf{0} & 1 &  % \land
%     0 & \textbf{1} & 1 &  % \lor
%     0 & \textbf{1} & 1 &  % \xor
%     0 & \textbf{1} & 1 &  % \implies
%     0 & \textbf{0} & 1 \\ % \iff
%     1 & 0 &
%     \textbf{0} & 1 &      % \neg
%     1 & \textbf{0} & 0 &  % \land
%     1 & \textbf{1} & 0 &  % \lor
%     1 & \textbf{1} & 0 &  % \xor
%     1 & \textbf{0} & 0 &  % \implies
%     1 & \textbf{0} & 0 \\ % \iff
%     1 & 1 &
%     \textbf{0} & 1 &      % \neg
%     1 & \textbf{1} & 1 &  % \land
%     1 & \textbf{1} & 1 &  % \lor
%     1 & \textbf{0} & 1 &  % \xor
%     1 & \textbf{1} & 1 &  % \implies
%     1 & \textbf{1} & 1 \\ % \iff
% \bottomrule
% \end{tabular}


% \subsection{Karnaugh map}

% \tikzset{mykarnaughstyle/.style={
%     karnaugh,
%     x=1\kmunitlength, y=1\kmunitlength,
%     enable indices,
%     % disable bars,
%     binary index,
%     kmindexpos={0.5}{0.15},
%     kmindex/.style={font=\tiny\itshape,blue},
%     kmcell/.style={
%         label={[font=\scriptsize,red,right,xshift=-0.45\kmunitlength,inner sep=1pt]\the\kmindexcounter},
%         font=\bfseries,
%     },
%     kmbar  top sep=0.7\kmunitlength,
%     kmbar left sep=0.7\kmunitlength,
%     grp/.style n args={3}{
%         draw,
%         rectangle,
%         ##1,
%         fill=##1!30,
%         fill opacity=0.3,
%         minimum width=##2\kmunitlength,
%         minimum height=##3\kmunitlength,
%         rounded corners=0.2\kmunitlength,
%     },
%     thick,
% }}

% \newcommand{\drawlabels}{%
%     \begin{scope}[font=\small]
%         \begin{scope}[xshift=0.5\kmunitlength]
%             \node[above] at (0,4) {00};
%             \node[above] at (1,4) {01};
%             \node[above] at (2,4) {11};
%             \node[above] at (3,4) {10};
%         \end{scope}
%         \begin{scope}[yshift=0.5\kmunitlength]
%             \node[left] at (0,3) {00};
%             \node[left] at (0,2) {01};
%             \node[left] at (0,1) {11};
%             \node[left] at (0,0) {10};
%         \end{scope}
%     \end{scope}
%     \draw[sloped,font=\scriptsize,inner sep=2pt] (0,4) -- node[above]{AB} node[below]{CD} +(-0.4,0.4);
% }

% The \textbf{Karnaugh map} is a heuristic method for simplifying Boolean formulas.

% % \begin{figure}[H]
% % \centering
% \begin{tikzpicture}[mykarnaughstyle]
%     \karnaughmapvert{4}%
%     % {$f(A,B,C,D)$}%
%     {}%
%     {{$A$}{$B$}{$C$}{$D$}}%
%     {0110 0110 0110 0110}%
%     {
%         \node[grp={blue}{0.9}{1.9}](n000) at (0.5,2) {};
%         \node[grp={blue}{0.9}{1.9}](n001) at (3.5,2) {};
%         % \draw[blue]
%         %     (n000.north) to [bend  left=25] (n001.north)
%         %     (n000.south) to [bend right=25] (n001.south) ;
%         \node[grp={red}{1.9}{0.9}](n100) at (2,3.5) {};
%         \node[grp={red}{1.9}{0.9}](n110) at (2,0.5) {};
%         % \draw[red]
%         %     (n100.west) to [bend right=25] (n110.west)
%         %     (n100.east) to [bend  left=25] (n110.east) ;
%     }
%     % \drawlabels
% \end{tikzpicture}
% % \end{figure}


\end{document}
