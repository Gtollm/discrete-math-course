\documentclass[a4paper,10pt]{article}
\usepackage{mypreamble}

%% Page setup
\geometry{
    margin=2cm,
    includehead,
    includefoot,
    headsep=8pt,
    footskip=16pt,
}
\pagestyle{fancy}
\MakeSingleHeader% {<l>}{<r>}
    {\TextCheatsheetEng: Graph Theory}%
    {\TextDiscreteMathEng, \IconSpring~Spring 2022}
\fancyfoot{}
\fancyfoot[L]{\tiny Build time: \DTMnow}
\fancyfoot[R]{\tiny Source code can be found at \url{https://github.com/Lipen/discrete-math-course}}
% \fancyfoot[C]{\thepage\ of \zpageref{LastPage}}

%% Add custom setup below

% \titlespacing{\type}{left}{above}{below}[right]
\titlespacing{\section}{0pt}{*1}{*0.5}
\titlespacing{\subsection}{0pt}{*1}{*0.5}

\newcommand{\op}[1]{\operatorname*{#1}}
\newcommand{\dist}[1]{\op{dist}(#1)}
\newcommand{\degree}[1]{\op{deg}(#1)}
\newcommand{\minDegree}[1]{\delta(#1)}
\newcommand{\maxDegree}[1]{\Delta(#1)}
\newcommand{\eccentricity}[1]{\varepsilon(#1)}
\newcommand{\graphRadius}[1]{\op{rad}(#1)}
\newcommand{\graphDiameter}[1]{\op{diam}(#1)}
\newcommand{\graphGirth}[1]{\op{girth}(#1)}
\newcommand{\graphCenter}[1]{\op{center}(#1)}
\newcommand{\graphCentroid}[1]{\op{centroid}(#1)}
% \newcommand{\stabilitynumber}[1]{\alpha(#1)}
% \newcommand{\matchingnumber}[1]{\alpha'(#1)}
\newcommand{\vertexConnectivity}[1]{\varkappa(#1)}
\newcommand{\edgeConnectivity}[1]{\lambda(#1)}
\newcommand{\blockGraph}[1]{\op{B}(#1)}

% \usepackage{cutwin}

\declaretheoremstyle[
    spaceabove=6pt,
    spacebelow=6pt,
    postheadspace=0.5em,
    notefont=\normalfont\scshape,
]{mystyle}
\declaretheorem[style=mystyle]{theorem}
% \declaretheoremstyle[
%     spaceabove=0pt,
%     spacebelow=0pt,
%     postheadspace=0.5em,
%     notefont=\normalfont\scshape,
% ]{mylemmastyle}
% \declaretheorem[style=mylemmastyle]{lemma}

\tikzset{
    dot/.style = {
        draw,
        fill=black,
        shape=circle,
        minimum size=4pt,
        inner sep=0pt,
        outer sep=0pt,
    },
    circ/.style = {
        draw,
        % fill=white,
        shape=circle,
        minimum size=4pt,
        inner sep=0pt,
        outer sep=0pt,
    },
    position/.style args={#1:#2 from #3}{
        at=(#3.#1), anchor=#1+180, shift=(#1:#2)
    },
}


\begin{document}

\selectlanguage{english}

\setcounter{section}{4}
\section{Graph Theory Cheatsheet}

\subsection{Graph Theory%
\texorpdfstring{\normalsize\protect\Href{https://en.wikipedia.org/wiki/Graph_theory}\hfill\href{https://en.wikipedia.org/wiki/Glossary_of_graph_theory}{Glossary}}{}}

\begin{terms}
    \item \textbf{Graph}\Href{https://en.wikipedia.org/wiki/Graph_(discrete_mathematics)} is an ordered pair $G = \Pair{V, E}$, where $V = \Set{v_1,\dotsc,v_n}$ is a set of vertices, and $E = \Set{e_1,\dotsc,e_m}$ is a set of edges.

    \item Simple \textbf{undirected}\Href{https://en.wikipedia.org/wiki/Undirected_graph} graphs have $E \subseteq V^{(2)}$, \ie each edge $e_i \in E$ between vertices $u$ and~$v$ is denoted by $\Set{u,v} \in V^{(2)}$.
    Such~\emph{undirected edges} are also called \emph{links} or \emph{lines}.

    \begin{terms}
        \item $A^{(k)} = \Set{\Set{x_1,\dotsc,x_k} \given x_1 \neq \dotsb \neq x_k \in A} = \Set{ S \given S \subseteq A, \card{S} = k }$ is the set of $k$-sized subsets of~$A$.
    \end{terms}

    \item Simple \textbf{directed}\Href{https://en.wikipedia.org/wiki/Directed_graph} graphs have $E \subseteq V^{2}$, \ie each edge $e_i \in E$ from vertex $u$ to~$v$ is denoted by an ordered pair~$\Pair{u,v} \in V^{2}$.
    Such \emph{directed edges} are also called \emph{arcs} or \emph{arrows}.

    \begin{terms}
        \item $A^k = A \times \dotsb \times A = \Set{\Tuple{x_1,\dotsc,x_k} \given x_1,\dotsc,x_k \in A}$ is the set of $k$-tuples (Cartesian $k$-power of~$A$).
    \end{terms}

    \begin{minipage}{\linewidth}

    % \setlength{\intextsep}{0pt}%
    % \setlength{\columnsep}{0pt}%
    \begin{wrapfigure}{r}{0pt}
        \setlength{\tabcolsep}{2pt}%
        \begin{adjustbox}{tabular={@{} ccc @{}}}
            \tikz[baseline=(label.base)]{
                \path
                    (0,.1)   node[dot] (a) {}
                    (.3,.8)  node[dot] (b) {}
                    (1,.8)   node[dot] (c) {}
                    (.7,.4)  node[dot] (d) {}
                    (1.2,0)  node[dot] (e) {}
                    (.5,-.2) node[dot] (f) {}
                ;
                \draw
                    (a) -- (b) -- (c) -- (e) -- (f) -- (a)
                    (d) edge (a) edge (e) edge (f)
                ;
                \node (label) at (.6,-.55) {Simple};
            } &
            \tikz[baseline=(label.base)]{
                \path
                    (0,.1)   node[dot] (a) {}
                    (.3,.8)  node[dot] (b) {}
                    (1,.8)   node[dot] (c) {}
                    (.7,.4)  node[dot] (d) {}
                    (1.2,0)  node[dot] (e) {}
                    (.5,-.2) node[dot] (f) {}
                ;
                \draw
                    (b) -- (c) -- (e) -- (d) -- (a) -- (f)
                    (a) edge[bend left] (b) edge[bend right] (b)
                    (e) edge[bend left] (f) edge[bend right] (f)
                ;
                \node (label) at (.6,-.55) {Multigraph};
            } &
            \tikz[baseline=(label.base)]{
                \path
                    (0,.2)   node[dot] (a) {}
                    (.25,.8) node[dot] (b) {}
                    (1,.8)   node[dot] (c) {}
                    (.6,.4)  node[dot] (d) {}
                    (1.2,.1) node[dot] (e) {}
                    (.5,-.2) node[dot] (f) {}
                ;
                \draw
                    (b) -- (d) -- (f) -- (e)
                    (a) edge[bend left] (f) edge[bend right] (f)
                    (c) edge[bend left] (e) edge[bend right] (e) edge (b)
                    (e) edge[out=30,in=-50,loop] ()
                    (a) edge[out=130,in=50,loop] ()
                ;
                \node (label) at (.7,-.5) {Pseudograph};
            } \\[4pt]

            \tikz[baseline=(label.base)]{
                \node (label) {Null};
            } &
            \tikz[baseline=(label.base)]{
                \path
                    (0,0)    node[dot] (a) {}
                    (.4,.5)  node[dot] (b) {}
                    (.4,-.2) node[dot] (c) {}
                    (.7,.2)  node[dot] (d) {}
                ;
                \node (label) at (.35,-.55) {Empty};
            } &
            \tikz[baseline=(label.base)]{
                \path
                    (0,0)    node[dot] (a) {}
                    (.4,.6)  node[dot] (b) {}
                    (1,.3)   node[dot] (c) {}
                    (.7,-.2) node[dot] (d) {}
                ;
                \draw
                    (a) edge (b) edge (c) edge (d)
                    (b) edge (c) edge (d)
                    (c) edge (d)
                ;
                \node (label) at (.5,-.55) {Complete};
            }
        \end{adjustbox}
        \vspace{-\intextsep}
    \end{wrapfigure}

    \item \textbf{Multi-edges}\Href{https://en.wikipedia.org/wiki/Multi-edge} are edges that have the same end nodes.
    \item \textbf{Loop}\Href{https://en.wikipedia.org/wiki/Loop_(graph_theory)} is an edge that connects a vertex to itself.
    \item \textbf{Simple graph}\Href{https://en.wikipedia.org/wiki/Graph_theory\#Graph} is a graph without multi-edges and loops.
    \item \textbf{Multigraph}\Href{https://en.wikipedia.org/wiki/Multigraph} is a graph with multi-edges.
    \item \textbf{Pseudograph}\Href{https://en.wikipedia.org/wiki/Pseudograph} is a multigraph with loops.
    % \item \textbf{Hypergraph}\Href{https://en.wikipedia.org/wiki/Hypergraph} is a generalization of a graph in which \\ an edge can join any number of vertices.

    \item \textbf{Null graph}\Href{https://en.wikipedia.org/wiki/Null_graph} is a \enquote{graph} without vertices.
    \item \textbf{Trivial graph} is a graph consisting of a single vertex.
    \item \textbf{Empty (edgeless) graph}\Href{https://en.wikipedia.org/wiki/Edgeless_graph} is a graph without edges.
    \end{minipage}

    \item \textbf{Complete graph}\Href{https://en.wikipedia.org/wiki/Complete_graph} $K_n$ is a simple graph in which every pair of distinct vertices is connected by an edge.

    \item \textbf{Weighted graph}\Href{https://en.wikipedia.org/wiki/Weighted_graph} $G = \Tuple{V, E, w}$ is a graph in which each edge has an associated numerical value (the \emph{weight}) represented by the \textbf{weight function} $w: E \to \mathrm{Num}$.

    \begin{minipage}{\linewidth}

    \setlength{\intextsep}{0pt}%
    % \setlength{\columnsep}{0pt}%
    \begin{wrapfigure}{r}{0pt}
        \tikzset{
            mylabel/.style args={#1:#2}{
                append after command={
                    (\tikzlastnode.center) node [#1] {#2}
                }
            },
        }
        \NiceMatrixOptions{
            code-for-first-row = \color{gray},
            code-for-first-col = \color{gray},
        }
        \setlength{\tabcolsep}{2pt}
        \begin{tabular}{@{} cc @{}}
            \tikz[baseline]{
                \path
                    (0,0)     node (a) [dot,mylabel={left=1pt:a}] {}
                    (.8,.6)   node (b) [dot,mylabel={above right:b}] {}
                    (1.2,-.1) node (c) [dot,mylabel={above right:c}] {}
                    (.5,-.5)  node (d) [dot,mylabel={left=1pt:d}] {}
                ;
                \draw[inner sep=2pt]
                    (a) edge node[sloped,above]{$e_1$} (b)
                    (b) edge node[sloped,above]{$e_2$} (c)
                    (c) edge node[sloped,below]{$e_3$} (d)
                    (b) edge node[sloped,above]{$e_4$} (d)
                    (c) edge[out=0,in=-90,loop,>=To] node[right]{$e_5$} (c)
                ;
            }
            &
            \tikz[baseline]{
                \path
                    (0,0)     node (a) [dot,mylabel={left=1pt:a}] {}
                    (.8,.6)   node (b) [dot,mylabel={above right:b}] {}
                    (1.2,-.1) node (c) [dot,mylabel={above right:c}] {}
                    (.5,-.5)  node (d) [dot,mylabel={left=1pt:d}] {}
                ;
                \draw[->,>={Stealth[]}, inner sep=2pt]
                    (a) edge node[sloped,above]{$e_1$} (b)
                    (b) edge node[sloped,above]{$e_2$} (c)
                    (c) edge node[sloped,below]{$e_3$} (d)
                    (b) edge node[sloped,above]{$e_4$} (d)
                    (c) edge[out=0,in=-90,loop,>=To] node[right]{$e_5$} (c)
                ;
            } \\
            %
            \multicolumn{2}{c}{\textbf{Adjacency matrix:}} \\
            %
            \(\begin{bNiceMatrix}[
                light-syntax,
                small,
                first-row, first-col,
            ]
               {} a b c d ;
                a 0 1 0 0 ;
                b 1 0 1 1 ;
                c 0 1 0 1 ;
                d 0 1 1 0 ;
            \end{bNiceMatrix}\)
            &
            \begin{NiceMatrixBlock}[auto-columns-width]
            \(\begin{bNiceMatrix}[
                light-syntax,
                small, r,
                first-row, first-col,
            ]
                {} a  b  c  d ;
                a  0  1  0  0 ;
                b -1  0  1  1 ;
                c  0 -1  0  1 ;
                d  0 -1 -1  0 ;
            \end{bNiceMatrix}\)
            \end{NiceMatrixBlock} \\
            %
            \multicolumn{2}{c}{\textbf{Incidence matrix:}} \\
            %
            \(\begin{bNiceMatrix}[
                light-syntax,
                small,
                first-row, first-col,
            ]
                {} e_1 e_2 e_3 e_4 e_5 ;
                a    0   0   0   0   0 ;
                b    1   1   0   1   0 ;
                c    0   1   1   0   2 ;
                d    0   0   1   1   0 ;
            \end{bNiceMatrix}\)
            &
            \begin{NiceMatrixBlock}[auto-columns-width]
            \(\begin{bNiceMatrix}[
                light-syntax,
                small, r,
                first-row, first-col,
            ]
                {} e_1 e_2 e_3 e_4 e_5 ;
                a   -1   0   0   0   0 ;
                b    1  -1   0  -1   0 ;
                c    0   1  -1   0   2 ;
                d    0   0   1   1   0 ;
            \end{bNiceMatrix}\)
            \end{NiceMatrixBlock} \\
        \end{tabular}
    \end{wrapfigure}

    \item \textbf{Adjacency}\Href{https://en.wikipedia.org/wiki/Adjacent_(graph_theory)} is the relation between two vertices connected with an edge.
    \item \textbf{Adjacency matrix}\Href{https://en.wikipedia.org/wiki/Adjacency_matrix} is a square matrix $A_{V \times V}$ used to represent an adjacency relation of a finite graph.
    \begin{terms}
        \item For simple graphs, adjacency matrix is binary, \ie $A_{ij} \in \Bool$.
        % \hfill\href{https://en.wikipedia.org/wiki/Logical_matrix}{Boolean matrix}
        \item For multigraphs, adjacency matrix contains edge mutiplicities, \ie $A_{ij} \in \NaturalZero$.
    \end{terms}

    \item \textbf{Incidence}\Href{https://en.wikipedia.org/wiki/Incidence_(geometry)} is a relation between an edge and its endpoints.
    \item \textbf{Incidence matrix}\Href{https://en.wikipedia.org/wiki/Incidence_matrix} is a Boolean matrix $B_{V \times E}$ of an incidence relation.

    \end{minipage}

    \item \textbf{Degree}\Href{https://en.wikipedia.org/wiki/Degree_(graph_theory)} $\degree{v}$ the number of edges indident to $v$ (loops are counted twice).

    \begin{terms}
        \item $\minDegree{G} = \min\limits_{v \in V} \degree{v}$ is the \textbf{minimum degree}.
        \item $\maxDegree{G} = \max\limits_{v \in V} \degree{v}$ is the \textbf{maximum degree}.
        % TODO: in-degree and out-degree
        \item \textsc{Handshaking lemma}. $\displaystyle \sum_{v \in V} \degree{v} = 2 \card{E}$.
    \end{terms}

    \item A graph is called \textbf{$r$-regular} if all its vertices have the same degree: $\forall v \in V : \degree{v} = r$.

    \begin{minipage}{\linewidth}

    \setlength{\intextsep}{0pt}%
    % \setlength{\columnsep}{0pt}%
    \begin{wrapfigure}{r}{0pt}
        \setlength{\tabcolsep}{3pt}%
        \NiceMatrixOptions{
            notes/style = \arabic{#1},
            notes/code-before = \footnotesize \color{gray} \RaggedRight,
        }
        \begin{NiceTabular}{@{} rccl @{}}
            \toprule
            \thead{Term} & \thead{V}\tabularnote{Can vertices be repeated?} & \thead{E}\tabularnote{Can edges be repeated?} & \thead{Closed} \\
            Walk  & $+$ & $+$ & Closed walk \\
            Trail & $+$ & $-$ & Circuit \\
            Path  & $-$ & $-$ & Cycle \\
                    & $-$ & $+$ & (\emph{impossible}) \\
            \bottomrule
        \end{NiceTabular}
    \end{wrapfigure}

    \item \textbf{Walk}\Href{https://en.wikipedia.org/wiki/Path_(graph_theory)\#Walk,_trail,_and_path} is an alternating sequence of vertices and edges in an arbitrary graph traversal.
    \begin{terms}
        \item \textbf{Trail} is a walk with distinct edges.
        \item \textbf{Path} is a walk with distinct vertices (and therefore distinct edges).
        \item A walk is \textbf{closed} if it starts and ends at the same vertex. Otherwise, it is \textbf{open}.
        \item \textbf{Circuit} is a closed trail.
        \item \textbf{Cycle} is a closed path.
    \end{terms}

    \end{minipage}

    \item \textbf{Length} of a path (walk, trail) $l = u \rightsquigarrow v$ is the number of edges in it: $\card{l} = \card{E(l)}$.

    \item \textbf{Girth}\Href{https://en.wikipedia.org/wiki/Girth_(graph_theory)} is the length of the shortest cycle in the graph.

    \begin{minipage}{\linewidth}

    \setlength{\intextsep}{0pt}%
    \setlength{\columnsep}{0pt}%
    \begin{wrapfigure}{r}{0pt}
        \setlength{\tabcolsep}{8pt}%
        \begin{tabular}{@{} cc @{}}
            \tikz[baseline, on grid]{
                \node[dot] (a) {};
                % Center
                \node[fit=(a), draw=blue, fill=blue, fill opacity=0.1, inner xsep=4pt, inner ysep=4pt, yshift=2pt] (center) {};
                \node[text=blue] at (0,.5) {Center};
                % Centroid
                \node[fit=(a), draw=green!60!black, fill=green, fill opacity=0.1, inner xsep=8pt, inner ysep=4pt, yshift=-2pt] (centroid) {};
                \node[text=green!60!black] at (0,-.5) {Centroid};
            } &
            \tikz[baseline, on grid]{
                \node[dot] (a) {};
                \node[dot] (b) [right=.5 of a] {};
                \node[dot] (c) [right=.5 of b] {};
                \node[dot] (d) [position=140:.5 from a] {};
                \node[dot] (e) [position=-140:.5 from a] {};
                \draw (a) -- (b) -- (c);
                \draw (a) edge (d) edge (e);
                % Center
                \node[fit=(a)(b), draw=blue, fill=blue, fill opacity=0.1, inner xsep=4pt, inner ysep=4pt, yshift=2pt] (center) {};
                \node[text=blue] at (.25,.5) {Center};
                % Centroid
                \node[fit=(a), draw=green!60!black, fill=green, fill opacity=0.1, inner xsep=6pt, inner ysep=4pt, xshift=-1pt, yshift=-2pt] (centroid) {};
                \node[text=green!60!black] at (.3,-.5) {Centroid};
            } \\
            \tikz[baseline, on grid]{
                \node[dot] (a) {};
                \node[dot] (b) [right=.5 of a] {};
                \node[dot] (c) [right=.5 of b] {};
                \node[dot] (d) [right=.5 of c] {};
                \node[dot] (e) [position=40:.5 from d] {};
                \node[dot] (f) [position=-40:.5 from d] {};
                \draw (a) -- (b) -- (c) -- (d);
                \draw (d) edge (e) edge (f);
                % Center
                \node[fit=(c), draw=blue, fill=blue, fill opacity=0.1, inner xsep=4pt, inner ysep=4pt, yshift=2pt] (center) {};
                \node[text=blue] at (1,.5) {Center};
                % Centroid
                \node[fit=(c)(d), draw=green!60!black, fill=green, fill opacity=0.1, inner xsep=6pt, inner ysep=4pt, xshift=-1pt, yshift=-2pt] (centroid) {};
                \node[text=green!60!black] at (1,-.5) {Centroid};
            } &
            \tikz[baseline, on grid]{
                \node[dot] (a) {};
                \node[dot] (b) [right=.5 of a] {};
                \draw (a) -- (b);
                % Center
                \node[fit=(a)(b), draw=blue, fill=blue, fill opacity=0.1, inner xsep=4pt, inner ysep=4pt, yshift=2pt] (center) {};
                \node[text=blue] at (.25,.5) {Center};
                % Centroid
                \node[fit=(a)(b), draw=green!60!black, fill=green, fill opacity=0.1, inner xsep=8pt, inner ysep=4pt, yshift=-2pt] (centroid) {};
                \node[text=green!60!black] at (.25,-.5) {Centroid};
            } \\
        \end{tabular}
    \end{wrapfigure}

    \item \textbf{Distance}\Href{https://en.wikipedia.org/wiki/Distance_(graph_theory)} $\dist{u,v}$ between two vertices is the length of the shortest path $u \rightsquigarrow v$.
    \begin{terms}
        \item $\eccentricity{v} = \max\limits_{u \in V} \dist{v,u}$ is the \textbf{eccentricity} of the vertex $v$.
        \item $\graphRadius{G} = \min\limits_{v \in V} \eccentricity{v}$ is the \textbf{radius} of the graph $G$.
        \item $\graphDiameter{G} = \max\limits_{v \in V} \eccentricity{v}$ is the \textbf{diameter} of the graph $G$.
        \item $\graphCenter{G} = \Set{v \given \eccentricity{v} = \graphRadius{G}}$ is the \textbf{center} of the graph $G$.
    \end{terms}

    \end{minipage}

    \item \textbf{Clique}\Href{https://en.wikipedia.org/wiki/Clique_(graph_theory)} $Q \subseteq V$ is a set of vertices inducing a complete subgraph.

    \item \textbf{Stable set}\Href{https://en.wikipedia.org/wiki/Independent_set_(graph_theory)} $S \subseteq V$ is a set of independent (pairwise non-adjacent) vertices.

    \item \textbf{Matching}\Href{https://en.wikipedia.org/wiki/Matching_(graph_theory)} $M \subseteq E$ is a set of independent (pairwise non-adjacent) edges.

    \item \textbf{Vertex cover}\Href{https://en.wikipedia.org/wiki/Vertex_cover} $R \subseteq V$ is a set of vertices \enquote{covering} all edges.

    \item \textbf{Edge cover}\Href{https://en.wikipedia.org/wiki/Edge_cover} $F \subseteq E$ is a set of edges \enquote{covering} all vertices.

    \item \textbf{Subgraph}\Href{https://en.wikipedia.org/wiki/Glossary_of_graph_theory\#subgraph} of a graph $G = \Pair{V,E}$ is another graph $G' = \Pair{V',E'}$ such that $V' \subseteq V$, $E' \subseteq E$.

    \item \textbf{Induces subgraph}\Href{https://en.wikipedia.org/wiki/Induced_subgraph} of a graph $G = \Pair{V,E}$ is another graph $G'$, formed from a subset~$S$ of the vertices of the graph and \emph{all} the edges (from the original graph) connecting pairs of vertices in that subset.
    Formally, $G[S] = G' = \Pair{V',E'}$, where $S \subseteq V$, $V' = V \intersection S$, $E' = \Set{e \in E \given \exists v \in S: e \mathrel{I} v}$.

    \item \textbf{Vertex connectivity}\Href{https://en.wikipedia.org/wiki/Vertex_connectivity} $\vertexConnectivity{G}$ is the minimum number of vertices that has to be removed in order to make the graph disconnected or trivial.
    Equivalently, it is the largest~$k$ for which the graph~$G$ is $k$-vertex-connected.

    \item \textbf{$k$-vertex-connected graph}\Href{https://en.wikipedia.org/wiki/K-vertex-connected_graph} is a graph that remains connected after less than $k$ vertices are removed, \ie $\vertexConnectivity{G} \geq k$.

    \begin{items}
        \item Corollary of Menger's theorem: graph $G = \Pair{V,E}$ is $k$-vertex-connected if, for every pair of vertices $u,v \in V$, it is possible to find $k$ \emph{vertex-independent} (\emph{internally vertex-disjoint}) paths between $u$ and~$v$.
        \item $k$-vertex-connected graphs are also called simply \emph{$k$-connected}.
        \item 1-connected graph is called \emph{connected}, 2-connected is \emph{biconnected}, 3-connected is \emph{triconnected}.
    \end{items}

    \item \textbf{Edge connectivity}\Href{https://en.wikipedia.org/wiki/Edge_connectivity} $\edgeConnectivity{G}$ is the minimum number of edges that has to be removed in order to make the graph disconnected.
    Equivalently, it is the largest~$k$ for which the graph~$G$ is $k$-edge-connected.

    \item \textbf{$k$-edge-connected graph}\Href{https://en.wikipedia.org/wiki/K-edge-connected_graph} is a graph that remains connected after less than $k$ edges are removed, \ie $\edgeConnectivity{G} \geq k$.
    \begin{items}
        \item Corollary of Menger's theorem: graph $G = \Pair{V,E}$ is $k$-edge-connected if, for every pair of vertices $u,v \in V$, it is possible to find $k$ \emph{edge-disjoint} paths between $u$ and~$v$.
    \end{items}

    \begin{minipage}{\linewidth}

    \setlength{\intextsep}{0pt}%
    \setlength{\columnsep}{0pt}%
    \begin{wrapfigure}{r}{0pt}
        \tikz[]{
            \coordinate (origin);
            \node[dot] (a1) [position=36:.5 from origin] {};
            \node[dot] (a2) [position=-36:.5 from origin] {};
            \node[dot] (a3) [position=108:.5 from origin] {};
            \node[dot] (a4) [position=-108:.5 from origin] {};
            \node[dot] (a5) [position=180:.5 from origin] {};
            \node[dot] (x1) [right=.8 of a1] {};
            \node[dot] (x2) [right=.8 of a2] {};
            \node[dot] (x3) [right=.5 of x1] {};
            \node[dot] (x4) [right=.5 of x2] {};
            \draw (a1) edge (a3) edge (a4) edge (a5)
                -- (a2) edge (a4) edge (a5)
                -- (a3) edge (a5)
                -- (a4)
                -- (a5)
                -- cycle;
            \draw (x1) edge (x3) edge (x4)
                -- (x2) edge (x4)
                -- (x3)
                -- (x4)
                -- cycle;
            \draw (a1) edge (x1);
            \draw (a2) edge (x1) edge (x2);
            \node[below] at (current bounding box.south) {$\vertexConnectivity{G} = 2$, $\edgeConnectivity{G} = 3$, \\ $\minDegree{G} = 3$, $\maxDegree{G} = 6$};
        }
    \end{wrapfigure}

    \item \textsc{Whitney's theorem}. $\vertexConnectivity{G} \leq \edgeConnectivity{G} \leq \minDegree{G}$.

    \end{minipage}

    \item \textbf{Forest}\Href{https://en.wikipedia.org/wiki/Tree_(graph_theory)\#Forest} is an undirected acyclic graph.
    \item \textbf{Tree}\Href{https://en.wikipedia.org/wiki/Tree_(graph_theory)} is a connected undirected acyclic graph.

    % \item

    \item \textit{To be continued...}

    % \vfill
    % \emph{TODO}

    % \begin{otherlanguage}{russian}

    % \item Операции над графами: дополнение, объединение, пересечение
    % \item Граф пересечений
    % \item Рёберный граф (Line graph)
    % \item Двудольный граф (Bipartite graph)
    % \item Связность (connectivity)
    % \item Сильная связность
    % \item Отношение (вершинной/рёберной) k-связности (неэкв./экв.)
    % \item Теорема Менгере
    % \item Двусвязность
    % \item Точки сочленения (шарниры/cut vertex/articulation point)
    % \item Блоки (компоненты вершинной двусвязности)
    % \item Отношение вершинной двусвязности (экв.)
    % \item Мосты
    % \item Острова (компоненты рёберной двусвязности)
    % \item Отношение рёберной двусвязности (эквивалентно 2-рёб-связности)
    % \item Дерево
    % \item Вес вершины в дереве
    % \item Центроид дерева

    % \end{otherlanguage}

\endgroup


\end{document}
