\documentclass[a4paper,10pt]{article}
\usepackage{mypreamble}

%% Page setup
\geometry{
    margin=2cm,
    includehead,
    includefoot,
    headsep=8pt,
    footskip=16pt,
}
\pagestyle{fancy}
\MakeSingleHeader% {<l>}{<r>}
    {\TextCheatsheetEng: Graph Theory}%
    {\TextDiscreteMathEng, \IconSpring~Spring 2025}
\fancyfoot{}
\fancyfoot[L]{\tiny Build time: \DTMnow}
\fancyfoot[R]{\tiny Source code can be found at \url{https://github.com/Lipen/discrete-math-course}}
% \fancyfoot[C]{\thepage\ of \zpageref{LastPage}}

%% Add custom setup below

% \titlespacing{\type}{left}{above}{below}[right]
\titlespacing{\section}{0pt}{*1}{*0.5}
\titlespacing{\subsection}{0pt}{*1}{*0.5}

\newcommand{\op}[1]{\operatorname*{#1}}
\newcommand{\dist}[1]{\op{dist}(#1)}
\newcommand{\degree}[1]{\op{deg}(#1)}
\newcommand{\minDegree}[1]{\delta(#1)}
\newcommand{\maxDegree}[1]{\Delta(#1)}
\newcommand{\eccentricity}[1]{\varepsilon(#1)}
\newcommand{\graphRadius}[1]{\op{rad}(#1)}
\newcommand{\graphDiameter}[1]{\op{diam}(#1)}
\newcommand{\graphGirth}[1]{\op{girth}(#1)}
\newcommand{\graphCenter}[1]{\op{center}(#1)}
\newcommand{\graphCentroid}[1]{\op{centroid}(#1)}
% \newcommand{\stabilitynumber}[1]{\alpha(#1)}
% \newcommand{\matchingnumber}[1]{\alpha'(#1)}
\newcommand{\vertexConnectivity}[1]{\varkappa(#1)}
\newcommand{\edgeConnectivity}[1]{\lambda(#1)}
\newcommand{\blockGraph}[1]{\op{B}(#1)}

\declaretheoremstyle[
    spaceabove=6pt,
    spacebelow=6pt,
    postheadspace=0.5em,
    notefont=\normalfont\scshape,
]{mystyle}
\declaretheorem[style=mystyle]{theorem}

\tikzset{
    dot/.style = {
        draw,
        fill=black,
        shape=circle,
        minimum size=4pt,
        inner sep=0pt,
    },
    position/.style args={#1:#2 from #3}{
        at=(#3.#1), anchor=#1+180, shift=(#1:#2)
    },
    mylabel/.style args={#1:#2}{
        append after command={
            (\tikzlastnode.center) node [#1] {#2}
        }
    },
}

\colorlet{myred}{Red}
\colorlet{mygreen}{green!70!black}
\colorlet{myblue}{RoyalBlue}


\begin{document}

\selectlanguage{english}

\setcounter{section}{4}% +1 = actual
\section{Graph Theory Cheatsheet%
\texorpdfstring{\normalsize\hfill\href{https://en.wikipedia.org/wiki/Glossary_of_graph_theory}{Glossary}}{}}

\begin{terms}
    \item \textbf{Graph}\Href{https://en.wikipedia.org/wiki/Graph_(discrete_mathematics)} is an ordered pair $G = \Pair{V, E}$, where $V = \Set{v_1,\dotsc,v_n}$ is a set of vertices, and $E = \Set{e_1,\dotsc,e_m}$ is a set of edges.
    \begin{terms}
        \item Given a graph $G$, the notation $V(G)$ denotes the vertices of $G$.
        \item Given a graph $G$, the notation $E(G)$ denotes the edges of $G$.
        \item In fact, $V({\cdot})$ and $E({\cdot})$ functions allow to access \enquote{vertices} and \enquote{edges} of any object possessing them (\eg paths).
    \end{terms}

    \item \textbf{Order} of a graph $G$ is the number of vertices in it: $\card{V(G)}$.
    \item \textbf{Size} of a graph $G$ is the number of edges in it: $\card{E(G)}$.

    \item Two graphs are \textbf{equal} if their vertex sets and edge sets are equal: $G_1 = G_2$ iff $V_1 = V_2$ and $E_1 = E_2$.

    \item Two graphs $G_1 = \Pair{V_1, E_1}$ and $G_2 = \Pair{V_2, E_2}$ are called \textbf{isomorphic}\Href{https://en.wikipedia.org/wiki/Graph_isomorphism}, denoted $G_1 \isomorphic G_2$, if there exists an \emph{edge-preserving} bijection $f \colon V_1 \to V_2$, \ie any two vertices $u,v \in V_1$ are adjacent in~$G_1$ if and only if $f(u)$ and~$f(v)$ are adjacent in~$G_2$.
    This means that the graphs are structurally identical \emph{up to vertex renaming}.

    \item Simple \textbf{undirected}\Href{https://en.wikipedia.org/wiki/Undirected_graph} graphs have $E \subseteq V^{(2)}$, \ie each edge $e_i \in E$ between vertices $u$ and~$v$ is denoted by $\Set{u,v} \in V^{(2)}$.
    Such~\emph{undirected edges} are also called \emph{links} or \emph{lines}.

    \begin{terms}
        \item $A^{(k)} = \Set{\Set{x_1,\dotsc,x_k} \given x_1 \neq \dotsb \neq x_k \in A} = \Set{ S \given S \subseteq A, \card{S} = k }$ is the set of $k$-sized subsets of~$A$.
    \end{terms}

    \item Simple \textbf{directed}\Href{https://en.wikipedia.org/wiki/Directed_graph} graphs have $E \subseteq V^{2}$, \ie each edge $e_i \in E$ from vertex $u$ to~$v$ is denoted by an ordered pair~$\Pair{u,v} \in V^{2}$.
    Such \emph{directed edges} are also called \emph{arcs} or \emph{arrows}.

    \begin{terms}
        \item $A^k = A \times \dotsb \times A = \Set{\Tuple{x_1,\dotsc,x_k} \given x_1,\dotsc,x_k \in A}$ is the set of $k$-tuples (Cartesian $k$-power of~$A$).
    \end{terms}

    \begin{minipage}{\linewidth}

    % \setlength{\intextsep}{0pt}%
    % \setlength{\columnsep}{0pt}%
    \begin{wrapfigure}{r}{0pt}
        \setlength{\tabcolsep}{2pt}%
        \begin{adjustbox}{tabular={@{} ccc @{}}}
            \tikz[baseline=(label.base)]{
                \path
                    (0,.1)   node[dot] (a) {}
                    (.3,.8)  node[dot] (b) {}
                    (1,.8)   node[dot] (c) {}
                    (.7,.4)  node[dot] (d) {}
                    (1.2,0)  node[dot] (e) {}
                    (.5,-.2) node[dot] (f) {}
                ;
                \draw
                    (a) -- (b) -- (c) -- (e) -- (f) -- (a)
                    (d) edge (a) edge (e) edge (f)
                ;
                \node (label) at (.6,-.55) {Simple};
            } &
            \tikz[baseline=(label.base)]{
                \path
                    (0,.1)   node[dot] (a) {}
                    (.3,.8)  node[dot] (b) {}
                    (1,.8)   node[dot] (c) {}
                    (.7,.4)  node[dot] (d) {}
                    (1.2,0)  node[dot] (e) {}
                    (.5,-.2) node[dot] (f) {}
                ;
                \draw
                    (b) -- (c) -- (e) -- (d) -- (a) -- (f)
                    (a) edge[bend left] (b) edge[bend right] (b)
                    (e) edge[bend left] (f) edge[bend right] (f)
                ;
                \node (label) at (.6,-.55) {Multigraph};
            } &
            \tikz[baseline=(label.base)]{
                \path
                    (0,.2)   node[dot] (a) {}
                    (.25,.8) node[dot] (b) {}
                    (1,.8)   node[dot] (c) {}
                    (.6,.4)  node[dot] (d) {}
                    (1.2,.1) node[dot] (e) {}
                    (.5,-.2) node[dot] (f) {}
                ;
                \draw
                    (b) -- (d) -- (f) -- (e)
                    (a) edge[bend left] (f) edge[bend right] (f)
                    (c) edge[bend left] (e) edge[bend right] (e) edge (b)
                    (e) edge[out=30,in=-50,loop] ()
                    (a) edge[out=130,in=50,loop] ()
                ;
                \node (label) at (.7,-.5) {Pseudograph};
            } \\[4pt]

            \tikz[baseline=(label.base)]{
                \node (label) {Null};
            } &
            \tikz[baseline=(label.base)]{
                \path
                    (0,0)    node[dot] (a) {}
                    (.4,.5)  node[dot] (b) {}
                    (.4,-.2) node[dot] (c) {}
                    (.7,.2)  node[dot] (d) {}
                ;
                \node (label) at (.35,-.55) {Empty};
            } &
            \tikz[baseline=(label.base)]{
                \path
                    (0,0)    node[dot] (a) {}
                    (.4,.6)  node[dot] (b) {}
                    (1,.3)   node[dot] (c) {}
                    (.7,-.2) node[dot] (d) {}
                ;
                \draw
                    (a) edge (b) edge (c) edge (d)
                    (b) edge (c) edge (d)
                    (c) edge (d)
                ;
                \node (label) at (.5,-.55) {Complete};
            }
        \end{adjustbox}
        \vspace{-\intextsep}
    \end{wrapfigure}

    \item \textbf{Multi-edges}\Href{https://en.wikipedia.org/wiki/Multi-edge} are edges that have the same end nodes.
    \item \textbf{Loop}\Href{https://en.wikipedia.org/wiki/Loop_(graph_theory)} is an edge that connects a vertex to itself.
    \item \textbf{Simple graph}\Href{https://en.wikipedia.org/wiki/Graph_theory\#Graph} is a graph without multi-edges and loops.
    \item \textbf{Multigraph}\Href{https://en.wikipedia.org/wiki/Multigraph} is a graph with multi-edges.
    \item \textbf{Pseudograph}\Href{https://en.wikipedia.org/wiki/Pseudograph} is a multigraph with loops.
    % \item \textbf{Hypergraph}\Href{https://en.wikipedia.org/wiki/Hypergraph} is a generalization of a graph in which \\ an edge can join any number of vertices.

    \item \textbf{Null graph}\Href{https://en.wikipedia.org/wiki/Null_graph} is a \enquote{graph} without vertices.
    \item \textbf{Trivial (singleton) graph} is a graph consisting of a single vertex.
    \item \textbf{Empty (edgeless) graph}\Href{https://en.wikipedia.org/wiki/Edgeless_graph} is a graph without edges.
    \end{minipage}

    \item \textbf{Complete graph}\Href{https://en.wikipedia.org/wiki/Complete_graph} $K_n$ is a simple graph in which every pair of distinct vertices is connected by an edge.

    \item \textbf{Weighted graph}\Href{https://en.wikipedia.org/wiki/Weighted_graph} $G = \Tuple{V, E, w}$ is a graph in which each edge has an associated numerical value (the \emph{weight}) represented by the \textbf{weight function} $w: E \to \mathrm{Num}$.

    \item \textbf{Subgraph}\Href{https://en.wikipedia.org/wiki/Glossary_of_graph_theory\#subgraph} of a graph $G = \Pair{V,E}$ is another graph $G' = \Pair{V',E'}$ such that $V' \subseteq V$, $E' \subseteq E$. Designated as $G' \subseteq G$.

    \item \textbf{Spanning (partial) subgraph}\Href{https://en.wikipedia.org/wiki/Glossary_of_graph_theory\#spanning} is a subgraph that includes all vertices of a graph.

    \item \textbf{Induces subgraph}\Href{https://en.wikipedia.org/wiki/Induced_subgraph} of a graph $G = \Pair{V,E}$ is another graph $G'$ formed from a subset~$S$ of the vertices of the graph and \emph{all} the edges (from the original graph) connecting pairs of vertices in that subset.
    Formally, $G' = G[S] = \Pair{V',E'}$, where $S \subseteq V$, $V' = V \intersection S$, $E' = \Set{e \in E \given \exists v \in S: e \mathrel{I} v}$.

    \begin{minipage}{\linewidth}

    \setlength{\intextsep}{0pt}%
    % \setlength{\columnsep}{0pt}%
    \begin{wrapfigure}{r}{0pt}
        \NiceMatrixOptions{
            code-for-first-row = \color{gray},
            code-for-first-col = \color{gray},
        }
        \setlength{\tabcolsep}{2pt}
        \begin{tabular}{@{} cc @{}}
            \tikz[baseline]{
                \path
                    (0,0)     node (a) [dot,mylabel={left=1pt:a}] {}
                    (.8,.6)   node (b) [dot,mylabel={above right:b}] {}
                    (1.2,-.1) node (c) [dot,mylabel={above right:c}] {}
                    (.5,-.5)  node (d) [dot,mylabel={left=1pt:d}] {}
                ;
                \draw[inner sep=2pt]
                    (a) edge node[sloped,above]{$e_1$} (b)
                    (b) edge node[sloped,above]{$e_2$} (c)
                    (c) edge node[sloped,below]{$e_3$} (d)
                    (b) edge node[sloped,above]{$e_4$} (d)
                    (c) edge[out=0,in=-90,loop,>=To] node[right]{$e_5$} (c)
                ;
            }
            &
            \tikz[baseline]{
                \path
                    (0,0)     node (a) [dot,mylabel={left=1pt:a}] {}
                    (.8,.6)   node (b) [dot,mylabel={above right:b}] {}
                    (1.2,-.1) node (c) [dot,mylabel={above right:c}] {}
                    (.5,-.5)  node (d) [dot,mylabel={left=1pt:d}] {}
                ;
                \draw[->,>={Stealth[]}, inner sep=2pt]
                    (a) edge node[sloped,above]{$e_1$} (b)
                    (b) edge node[sloped,above]{$e_2$} (c)
                    (c) edge node[sloped,below]{$e_3$} (d)
                    (b) edge node[sloped,above]{$e_4$} (d)
                    (c) edge[out=0,in=-90,loop,>=To] node[right]{$e_5$} (c)
                ;
            } \\
            %
            \multicolumn{2}{c}{\textbf{Adjacency matrix:}} \\
            %
            \(\begin{bNiceMatrix}[
                light-syntax,
                small,
                first-row, first-col,
            ]
               {} a b c d ;
                a 0 1 0 0 ;
                b 1 0 1 1 ;
                c 0 1 1 1 ;
                d 0 1 1 0 ;
            \end{bNiceMatrix}\)
            &
            \begin{NiceMatrixBlock}[auto-columns-width]
            \(\begin{bNiceMatrix}[
                light-syntax,
                small, r,
                first-row, first-col,
            ]
                {} a  b  c  d ;
                a  0  1  0  0 ;
                b -1  0  1  1 ;
                c  0 -1  1  1 ;
                d  0 -1 -1  0 ;
            \end{bNiceMatrix}\)
            \end{NiceMatrixBlock} \\
            %
            \multicolumn{2}{c}{\textbf{Incidence matrix:}} \\
            %
            \(\begin{bNiceMatrix}[
                light-syntax,
                small,
                first-row, first-col,
            ]
                {} e_1 e_2 e_3 e_4 e_5 ;
                a    0   0   0   0   0 ;
                b    1   1   0   1   0 ;
                c    0   1   1   0   2 ;
                d    0   0   1   1   0 ;
            \end{bNiceMatrix}\)
            &
            \begin{NiceMatrixBlock}[auto-columns-width]
            \(\begin{bNiceMatrix}[
                light-syntax,
                small, r,
                first-row, first-col,
            ]
                {} e_1 e_2 e_3 e_4 e_5 ;
                a   -1   0   0   0   0 ;
                b    1  -1   0  -1   0 ;
                c    0   1  -1   0   2 ;
                d    0   0   1   1   0 ;
            \end{bNiceMatrix}\)
            \end{NiceMatrixBlock} \\
        \end{tabular}
    \end{wrapfigure}

    \item \textbf{Adjacency}\Href{https://en.wikipedia.org/wiki/Adjacent_(graph_theory)} is the relation between two vertices connected with an edge.
    \item \textbf{Adjacency matrix}\Href{https://en.wikipedia.org/wiki/Adjacency_matrix} is a square matrix $A_{V \times V}$ of an adjacency relation.
    \begin{terms}
        \item For simple graphs, adjacency matrix is binary, \ie $A_{ij} \in \Set{0,1}$.
        \item For directed graphs, $A_{ij} \in \Set{0,1,-1}$.
        \item For multigraphs, adjacency matrix contains edge multiplicities, \ie $A_{ij} \in \NaturalZero$.
    \end{terms}

    \item \textbf{Incidence}\Href{https://en.wikipedia.org/wiki/Incidence_(geometry)} is a relation between an edge and its endpoints.
    \item \textbf{Incidence matrix}\Href{https://en.wikipedia.org/wiki/Incidence_matrix} is a Boolean matrix $B_{V \times E}$ of an incidence relation.

    \end{minipage}

    \item \textbf{Degree}\Href{https://en.wikipedia.org/wiki/Degree_(graph_theory)} $\degree{v}$ the number of edges incident to $v$ (loops are counted twice).

    \begin{terms}
        \item $\minDegree{G} = \min\limits_{v \in V} \degree{v}$ is the \textbf{minimum degree}.
        \item $\maxDegree{G} = \max\limits_{v \in V} \degree{v}$ is the \textbf{maximum degree}.
        % TODO: in-degree and out-degree
        \item \textsc{Handshaking lemma}. $\displaystyle \sum_{v \in V} \degree{v} = 2 \card{E}$.
    \end{terms}

    \item A graph is called \textbf{$r$-regular}\Href{https://en.wikipedia.org/wiki/Regular_graph} if all its vertices have the same degree: $\forall v \in V : \degree{v} = r$.

    \item \textbf{Complement graph}\Href{https://en.wikipedia.org/wiki/Complement_graph} of a graph $G$ is a graph~$H$ on the same vertices such that two distinct vertices of~$H$ are adjacent iff they are non-adjacent in~$G$.

    \item \textbf{Intersection graph}\Href{https://en.wikipedia.org/wiki/Intersection_graph} of a family of sets $F = \Set{S_i}$ is a graph $G = \Omega(F) = \Pair{V,E}$ such that each vertex $v_i \in V$ denotes the set~$S_i$, \ie $V = F$, and the two vertices $v_i$ and~$v_j$ are adjacent whenever the corresponding sets $S_i$ and~$S_j$ have a non-empty intersection, \ie $E = \Set{ \Pair{v_i,v_j} \given i \neq j, S_i \intersection S_j \neq \emptyset }$.

    \needspace{2\baselineskip}

    \item \textbf{Line graph}\Href{https://en.wikipedia.org/wiki/Line_graph} of a graph $G = \Pair{V,E}$ is another graph $L(G) = \Omega(E)$ that represents the adjacencies between edges of~$G$. Each vertex of~$L(G)$ represents an edge of~$G$, and two vertices of~$L(G)$ are adjacent iff the corresponding edges share a common endpoint in~$G$ (\ie edges are \enquote{adjacent}/\enquote{incident}).

    \begin{minipage}{\linewidth}

    \setlength{\intextsep}{0pt}%
    % \setlength{\columnsep}{0pt}%
    \begin{wrapfigure}{r}{0pt}
        \setlength{\tabcolsep}{4pt}%
        \NiceMatrixOptions{
            notes/style = \arabic{#1},
            notes/code-before = \footnotesize \color{gray} \RaggedRight,
        }
        \begin{NiceTabular}{ cccl }
            % Note: "If a command \tabularnote{...} is exactly at the end of a cell (with no space at all after), the label of the note is composed in an overlapping position (towards the right)."
            \toprule \RowStyle{\bfseries}
            Term & V\tabularnote{Can vertices be repeated?}& E\tabularnote{Can edges be repeated?}& \enquote{Closed} term \\
            \midrule
            Walk  & $+$ & $+$ & Closed walk \\
            Trail & $+$ & $-$ & Circuit \\
            Path  & $-$ & $-$ & Cycle \\
                  & $-$ & $+$ & (\emph{impossible}) \\
            \bottomrule
        \end{NiceTabular}
    \end{wrapfigure}

    \item \textbf{Walk}\Href{https://en.wikipedia.org/wiki/Path_(graph_theory)\#Walk,_trail,_and_path} is an alternating sequence of vertices and edges: $l = v_{1} e_{1} v_{2} \dots e_{n-1} v_{n}$.
    \begin{terms}
        \item \textbf{Trail} is a walk with distinct edges.
        \item \textbf{Path} is a walk with distinct vertices (and therefore distinct edges).
        \item A walk is \textbf{closed} if it starts and ends at the same vertex. Otherwise, it is \textbf{open}.
        \item \textbf{Circuit} is a closed trail.
        \item \textbf{Cycle} is a closed path.
    \end{terms}

    \end{minipage}

    \item \textbf{Length} of a path (walk, trail) $l = u \rightsquigarrow v$ is the number of edges in it: $\card{l} = \card{E(l)}$.

    \item \textbf{Girth}\Href{https://en.wikipedia.org/wiki/Girth_(graph_theory)} is the length of the shortest cycle in the graph.

    \begin{minipage}{\linewidth}

    \setlength{\intextsep}{0pt}%
    \setlength{\columnsep}{0pt}%
    \begin{wrapfigure}{r}{0pt}
        \colorlet{mygreen}{green!60!black}%
        \setlength{\tabcolsep}{8pt}%
        \begin{tabular}{@{} cc @{}}
            \tikz[baseline, on grid]{
                \node[dot] (a) {};
                % Center
                \node[fit=(a), draw=blue, fill=blue, fill opacity=0.1, inner xsep=4pt, inner ysep=4pt, yshift=2pt] (center) {};
                \node[text=blue] at (0,.5) {Center};
                % Centroid
                \node[fit=(a), draw=mygreen, fill=green, fill opacity=0.1, inner xsep=8pt, inner ysep=4pt, yshift=-2pt] (centroid) {};
                \node[text=mygreen] at (0,-.5) {Centroid};
            } &
            \tikz[baseline, on grid]{
                \node[dot] (a) {};
                \node[dot] (b) [right=.5 of a] {};
                \node[dot] (c) [right=.5 of b] {};
                \node[dot] (d) [position=140:.5 from a] {};
                \node[dot] (e) [position=-140:.5 from a] {};
                \draw (a) -- (b) -- (c);
                \draw (a) edge (d) edge (e);
                % Center
                \node[fit=(a)(b), draw=blue, fill=blue, fill opacity=0.1, inner xsep=4pt, inner ysep=4pt, yshift=2pt] (center) {};
                \node[text=blue] at (.25,.5) {Center};
                % Centroid
                \node[fit=(a), draw=mygreen, fill=green, fill opacity=0.1, inner xsep=6pt, inner ysep=4pt, xshift=-1pt, yshift=-2pt] (centroid) {};
                \node[text=mygreen] at (.3,-.5) {Centroid};
            } \\
            \tikz[baseline, on grid]{
                \node[dot] (a) {};
                \node[dot] (b) [right=.5 of a] {};
                \node[dot] (c) [right=.5 of b] {};
                \node[dot] (d) [right=.5 of c] {};
                \node[dot] (e) [position=40:.5 from d] {};
                \node[dot] (f) [position=-40:.5 from d] {};
                \draw (a) -- (b) -- (c) -- (d);
                \draw (d) edge (e) edge (f);
                % Center
                \node[fit=(c), draw=blue, fill=blue, fill opacity=0.1, inner xsep=4pt, inner ysep=4pt, yshift=2pt] (center) {};
                \node[text=blue] at (1,.5) {Center};
                % Centroid
                \node[fit=(c)(d), draw=mygreen, fill=green, fill opacity=0.1, inner xsep=6pt, inner ysep=4pt, xshift=-1pt, yshift=-2pt] (centroid) {};
                \node[text=mygreen] at (1,-.5) {Centroid};
            } &
            \tikz[baseline, on grid]{
                \node[dot] (a) {};
                \node[dot] (b) [right=.5 of a] {};
                \draw (a) -- (b);
                % Center
                \node[fit=(a)(b), draw=blue, fill=blue, fill opacity=0.1, inner xsep=4pt, inner ysep=4pt, yshift=2pt] (center) {};
                \node[text=blue] at (.25,.5) {Center};
                % Centroid
                \node[fit=(a)(b), draw=mygreen, fill=green, fill opacity=0.1, inner xsep=8pt, inner ysep=4pt, yshift=-2pt] (centroid) {};
                \node[text=mygreen] at (.25,-.5) {Centroid};
            } \\
        \end{tabular}
    \end{wrapfigure}

    \item \textbf{Distance}\Href{https://en.wikipedia.org/wiki/Distance_(graph_theory)} $\dist{u,v}$ between two vertices is the length of the shortest path $u \rightsquigarrow v$.

    \end{minipage}

    \begin{terms}
        \item $\eccentricity{v} = \max\limits_{u \in V} \dist{v,u}$ is the \textbf{eccentricity} of the vertex $v$.
        \item $\graphRadius{G} = \min\limits_{v \in V} \eccentricity{v}$ is the \textbf{radius} of the graph $G$.
        \item $\graphDiameter{G} = \max\limits_{v \in V} \eccentricity{v}$ is the \textbf{diameter} of the graph $G$.
        \item $\graphCenter{G} = \Set{v \given \eccentricity{v} = \graphRadius{G}}$ is the \textbf{center} of the graph $G$.
    \end{terms}

    \item \textbf{Clique}\Href{https://en.wikipedia.org/wiki/Clique_(graph_theory)} $Q \subseteq V$ is a set of vertices inducing a complete subgraph.

    \item \textbf{Stable set}\Href{https://en.wikipedia.org/wiki/Independent_set_(graph_theory)} $S \subseteq V$ is a set of independent (pairwise non-adjacent) vertices.

    \medskip
    \begin{tikzpicture}[
        baseline,
        on grid,
        stable/.style={
            dot,
            minimum size=11pt,
            fill=mygreen,
            fill opacity=0.2,
            % label={center:\scriptsize\faCat}
            label={center:\scriptsize\faBiohazard}
        },
        notstable/.style={stable,fill=myred},
    ]
        \def\Radius{0.5}
        \def\HGap{2.5cm}
        \def\LabelYShift{-4pt}

        \begin{scope}
            \node[notstable] (v1) at (0:\Radius) {};
            \node[notstable] (v2) at (60:\Radius) {};
            \node[dot] (v3) at (120:\Radius) {};
            \node[dot] (v4) at (180:\Radius) {};
            \node[notstable] (v5) at (240:\Radius) {};
            \node[dot] (v6) at (300:\Radius) {};
            \node[dot] (v7) [left=\Radius of v3] {};
            \draw[-] (v1) -- (v2) -- (v3) -- (v4) -- (v5) -- (v6) -- (v1) (v3) -- (v7);
            \node[below,align=center] at (0,-\Radius) [yshift=\LabelYShift] {\textcolor{myred}{\textbf{not stable set}}};
        \end{scope}

        \begin{scope}[xshift=\HGap]
            \node[stable] (v1) at (0:\Radius) {};
            \node[dot] (v2) at (60:\Radius) {};
            \node[dot] (v3) at (120:\Radius) {};
            \node[stable] (v4) at (180:\Radius) {};
            \node[dot] (v5) at (240:\Radius) {};
            \node[dot] (v6) at (300:\Radius) {};
            \node[dot] (v7) [left=\Radius of v3] {};
            \draw[-] (v1) -- (v2) -- (v3) -- (v4) -- (v5) -- (v6) -- (v1) (v3) -- (v7);
            \node[below,align=center] at (0,-\Radius) [yshift=\LabelYShift] {\textcolor{mygreen}{\textbf{stable set}} \\ \textcolor{myred}{not maximal} \\ \textcolor{myred}{not maximum}};
        \end{scope}

        \begin{scope}[xshift=\HGap*2]
            \node[stable] (v1) at (0:\Radius) {};
            \node[dot] (v2) at (60:\Radius) {};
            \node[stable] (v3) at (120:\Radius) {};
            \node[dot] (v4) at (180:\Radius) {};
            \node[stable] (v5) at (240:\Radius) {};
            \node[dot] (v6) at (300:\Radius) {};
            \node[dot] (v7) [left=\Radius of v3] {};
            \draw[-] (v1) -- (v2) -- (v3) -- (v4) -- (v5) -- (v6) -- (v1) (v3) -- (v7);
            \node[below,align=center] at (0,-\Radius) [yshift=\LabelYShift] {\textcolor{mygreen}{\textbf{stable set}} \\ \textcolor{mygreen}{maximal} \\ \textcolor{myred}{not maximum}};
        \end{scope}

        \begin{scope}[xshift=\HGap*3]
            \node[dot] (v1) at (0:\Radius) {};
            \node[stable] (v2) at (60:\Radius) {};
            \node[dot] (v3) at (120:\Radius) {};
            \node[stable] (v4) at (180:\Radius) {};
            \node[dot] (v5) at (240:\Radius) {};
            \node[stable] (v6) at (300:\Radius) {};
            \node[stable] (v7) [left=\Radius of v3] {};
            \draw[-] (v1) -- (v2) -- (v3) -- (v4) -- (v5) -- (v6) -- (v1) (v3) -- (v7);
            \node[below,align=center] at (0,-\Radius) [yshift=\LabelYShift] {\textcolor{mygreen}{\textbf{stable set}} \\ \textcolor{mygreen}{maximal} \\ \textcolor{mygreen}{maximum}};
        \end{scope}
    \end{tikzpicture}

    \item \textbf{Matching}\Href{https://en.wikipedia.org/wiki/Matching_(graph_theory)} $M \subseteq E$ is a set of independent (pairwise non-adjacent) edges.

    \medskip
    \begin{tikzpicture}[
        baseline,
        on grid,
        matching/.style={mygreen,draw opacity=0.7,line width=3pt},
        notmatching/.style={matching,myred},
    ]
        \def\Side{0.6}
        \def\HGap{2.5cm}
        \def\LabelX{1.5*\Side}
        \def\LabelY{0}
        \def\LabelYShift{-4pt}

        \begin{scope}
            \node[dot] (v1) at (0,0) {};
            \node[dot] (v2) at (\Side,0) {};
            \node[dot] (v3) at (2*\Side,0) {};
            \node[dot] (v4) at (3*\Side,0) {};
            \node[dot] (v5) at (0.5*\Side,\Side) {};
            \node[dot] (v6) at (1.5*\Side,\Side) {};
            \draw (v1) -- (v2);
            \draw[notmatching] (v1) -- (v5);
            \draw[notmatching] (v2) -- (v3);
            \draw[notmatching] (v2) -- (v6);
            \draw (v3) -- (v4);
            \draw[notmatching] (v3) -- (v6);
            \draw (v5) -- (v6);
            \node[below,align=center] at (\LabelX,\LabelY) [yshift=\LabelYShift] {\textcolor{myred}{\textbf{not matching}}};
        \end{scope}

        \begin{scope}[xshift=1*\HGap]
            \node[dot] (v1) at (0,0) {};
            \node[dot] (v2) at (\Side,0) {};
            \node[dot] (v3) at (2*\Side,0) {};
            \node[dot] (v4) at (3*\Side,0) {};
            \node[dot] (v5) at (0.5*\Side,\Side) {};
            \node[dot] (v6) at (1.5*\Side,\Side) {};
            \draw[matching] (v1) -- (v2);
            \draw (v1) -- (v5);
            \draw (v2) -- (v3);
            \draw (v2) -- (v6);
            \draw (v3) -- (v4);
            \draw (v3) -- (v6);
            \draw (v5) -- (v6);
            \node[below,align=center] at (\LabelX,\LabelY) [yshift=\LabelYShift] {\textcolor{mygreen}{\textbf{matching}} \\ \textcolor{myred}{not maximal} \\ \textcolor{myred}{not maximum}};
        \end{scope}

        \begin{scope}[xshift=2*\HGap]
            \node[dot] (v1) at (0,0) {};
            \node[dot] (v2) at (\Side,0) {};
            \node[dot] (v3) at (2*\Side,0) {};
            \node[dot] (v4) at (3*\Side,0) {};
            \node[dot] (v5) at (0.5*\Side,\Side) {};
            \node[dot] (v6) at (1.5*\Side,\Side) {};
            \draw (v1) -- (v2);
            \draw (v1) -- (v5);
            \draw[matching] (v2) -- (v3);
            \draw (v2) -- (v6);
            \draw (v3) -- (v4);
            \draw (v3) -- (v6);
            \draw[matching] (v5) -- (v6);
            \node[below,align=center] at (\LabelX,\LabelY) [yshift=\LabelYShift] {\textcolor{mygreen}{\textbf{matching}} \\ \textcolor{mygreen}{maximal} \\ \textcolor{myred}{not maximum}};
        \end{scope}

        \begin{scope}[xshift=3*\HGap]
            \node[dot] (v1) at (0,0) {};
            \node[dot] (v2) at (\Side,0) {};
            \node[dot] (v3) at (2*\Side,0) {};
            \node[dot] (v4) at (3*\Side,0) {};
            \node[dot] (v5) at (0.5*\Side,\Side) {};
            \node[dot] (v6) at (1.5*\Side,\Side) {};
            \draw (v1) -- (v2);
            \draw[matching] (v1) -- (v5);
            \draw (v2) -- (v3);
            \draw[matching] (v2) -- (v6);
            \draw[matching] (v3) -- (v4);
            \draw (v3) -- (v6);
            \draw (v5) -- (v6);
            \node[below,align=center] at (\LabelX,\LabelY) [yshift=\LabelYShift] {\textcolor{mygreen}{\textbf{matching}} \\ \textcolor{mygreen}{maximal} \\ \textcolor{mygreen}{maximum}};
        \end{scope}
    \end{tikzpicture}

    \item \textbf{Perfect matching}\Href{https://en.wikipedia.org/wiki/Perfect_matching} is a matching that covers all vertices in the graph.
    \begin{terms}
        \item A perfect matching (if it exists) is always a minimum edge cover (\emph{but not vice-versa!}).
    \end{terms}

    \item \textbf{Vertex cover}\Href{https://en.wikipedia.org/wiki/Vertex_cover} $R \subseteq V$ is a set of vertices \enquote{covering} all edges.

    \medskip
    \begin{tikzpicture}[
        baseline,
        on grid,
        cover/.style={
            dot,
            minimum size=11pt,
            fill=mygreen,
            fill opacity=0.2,
            label={center:\scriptsize\faEye[regular]}
        },
        notcover/.style={cover,fill=myred},
        notcovered/.style={myred,draw opacity=0.5,line width=2pt},
    ]
        \def\Side{0.8}
        \def\HGap{2.5cm}
        \def\LabelX{\Side}
        \def\LabelY{0}
        \def\LabelYShift{-6pt}

        \begin{scope}
            \node[dot] (v1) at (0,0) {};
            \node[notcover] (v2) at (\Side,0) {};
            \node[dot] (v3) at (2*\Side,0) {};
            \node[notcover] (v4) at (0,\Side) {};
            \node[dot] (v5) at (\Side,\Side) {};
            \node[notcover] (v6) at (2*\Side,\Side) {};
            \draw (v1) -- (v4);
            \draw[notcovered] (v1) -- (v5);
            \draw (v2) -- (v5);
            \draw[notcovered] (v3) -- (v5);
            \draw (v4) -- (v5);
            \draw (v5) -- (v6);
            \node[below,align=center] at (\LabelX,\LabelY) [yshift=\LabelYShift] {\textcolor{myred}{\textbf{not vertex cover}}};
        \end{scope}

        \begin{scope}[xshift=1*\HGap]
            \node[cover] (v1) at (0,0) {};
            \node[cover] (v2) at (\Side,0) {};
            \node[dot] (v3) at (2*\Side,0) {};
            \node[dot] (v4) at (0,\Side) {};
            \node[cover] (v5) at (\Side,\Side) {};
            \node[cover] (v6) at (2*\Side,\Side) {};
            \draw (v1) -- (v4);
            \draw (v1) -- (v5);
            \draw (v2) -- (v5);
            \draw (v3) -- (v5);
            \draw (v4) -- (v5);
            \draw (v5) -- (v6);
            \node[below,align=center] at (\LabelX,\LabelY) [yshift=\LabelYShift] {\textcolor{mygreen}{\textbf{vertex cover}} \\ \textcolor{myred}{not minimal} \\ \textcolor{myred}{not minimum}};
        \end{scope}

        \begin{scope}[xshift=2*\HGap]
            \node[cover] (v1) at (0,0) {};
            \node[cover] (v2) at (\Side,0) {};
            \node[cover] (v3) at (2*\Side,0) {};
            \node[cover] (v4) at (0,\Side) {};
            \node[dot] (v5) at (\Side,\Side) {};
            \node[cover] (v6) at (2*\Side,\Side) {};
            \draw (v1) -- (v4);
            \draw (v1) -- (v5);
            \draw (v2) -- (v5);
            \draw (v3) -- (v5);
            \draw (v4) -- (v5);
            \draw (v5) -- (v6);
            \node[below,align=center] at (\LabelX,\LabelY) [yshift=\LabelYShift] {\textcolor{mygreen}{\textbf{vertex cover}} \\ \textcolor{mygreen}{minimal} \\ \textcolor{myred}{not minimum}};
        \end{scope}

        \begin{scope}[xshift=3*\HGap]
            \node[cover] (v1) at (0,0) {};
            \node[dot] (v2) at (\Side,0) {};
            \node[dot] (v3) at (2*\Side,0) {};
            \node[dot] (v4) at (0,\Side) {};
            \node[cover] (v5) at (\Side,\Side) {};
            \node[dot] (v6) at (2*\Side,\Side) {};
            \draw (v1) -- (v4);
            \draw (v1) -- (v5);
            \draw (v2) -- (v5);
            \draw (v3) -- (v5);
            \draw (v4) -- (v5);
            \draw (v5) -- (v6);
            \node[below,align=center] at (\LabelX,\LabelY) [yshift=\LabelYShift] {\textcolor{mygreen}{\textbf{vertex cover}} \\ \textcolor{mygreen}{minimal} \\ \textcolor{mygreen}{minimum}};
        \end{scope}
    \end{tikzpicture}

    \item \textbf{Edge cover}\Href{https://en.wikipedia.org/wiki/Edge_cover} $F \subseteq E$ is a set of edges \enquote{covering} all vertices.

    \medskip
    \begin{tikzpicture}[
        baseline,
        on grid,
        cover/.style={mygreen,draw opacity=0.7,line width=3pt},
        notcover/.style={cover,myred},
    ]
        \def\Side{0.6}
        \def\HGap{2.5cm}
        \def\LabelX{0.5*\Side}
        \def\LabelY{0}
        \def\LabelYShift{-4pt}

        \begin{scope}
            \node[dot] (v1) at (0,0) {};
            \node[dot] (v2) at (\Side,0) {};
            \node[dot] (v3) at (0,\Side) {};
            \node[dot] (v4) at (\Side,\Side) {};
            \node[dot] (v5) at (2*\Side,\Side) {};
            \node[dot] (v6) at (150:\Side) {};
            \draw (v1) -- (v2);
            \draw[notcover] (v1) -- (v3);
            \draw (v1) -- (v6);
            \draw[notcover] (v2) -- (v4);
            \draw (v3) -- (v4);
            \draw (v3) -- (v6);
            \draw (v4) -- (v5);
            \node[below,align=center] at (\LabelX,\LabelY) [yshift=\LabelYShift] {\textcolor{myred}{\textbf{not edge cover}}};
        \end{scope}

        \begin{scope}[xshift=1*\HGap]
            \node[dot] (v1) at (0,0) {};
            \node[dot] (v2) at (\Side,0) {};
            \node[dot] (v3) at (0,\Side) {};
            \node[dot] (v4) at (\Side,\Side) {};
            \node[dot] (v5) at (2*\Side,\Side) {};
            \node[dot] (v6) at (150:\Side) {};
            \draw[cover] (v1) -- (v2);
            \draw (v1) -- (v3);
            \draw (v1) -- (v6);
            \draw[cover] (v2) -- (v4);
            \draw (v3) -- (v4);
            \draw[cover] (v3) -- (v6);
            \draw[cover] (v4) -- (v5);
            \node[below,align=center] at (\LabelX,\LabelY) [yshift=\LabelYShift] {\textcolor{mygreen}{\textbf{edge cover}} \\ \textcolor{myred}{not minimal} \\ \textcolor{myred}{not minimum}};
        \end{scope}

        \begin{scope}[xshift=2*\HGap]
            \node[dot] (v1) at (0,0) {};
            \node[dot] (v2) at (\Side,0) {};
            \node[dot] (v3) at (0,\Side) {};
            \node[dot] (v4) at (\Side,\Side) {};
            \node[dot] (v5) at (2*\Side,\Side) {};
            \node[dot] (v6) at (150:\Side) {};
            \draw (v1) -- (v2);
            \draw[cover] (v1) -- (v3);
            \draw (v1) -- (v6);
            \draw[cover] (v2) -- (v4);
            \draw (v3) -- (v4);
            \draw[cover] (v3) -- (v6);
            \draw[cover] (v4) -- (v5);
            \node[below,align=center] at (\LabelX,\LabelY) [yshift=\LabelYShift] {\textcolor{mygreen}{\textbf{edge cover}} \\ \textcolor{mygreen}{minimal} \\ \textcolor{myred}{not minimum}};
        \end{scope}

        \begin{scope}[xshift=3*\HGap]
            \node[dot] (v1) at (0,0) {};
            \node[dot] (v2) at (\Side,0) {};
            \node[dot] (v3) at (0,\Side) {};
            \node[dot] (v4) at (\Side,\Side) {};
            \node[dot] (v5) at (2*\Side,\Side) {};
            \node[dot] (v6) at (150:\Side) {};
            \draw[cover] (v1) -- (v2);
            \draw (v1) -- (v3);
            \draw (v1) -- (v6);
            \draw (v2) -- (v4);
            \draw (v3) -- (v4);
            \draw[cover] (v3) -- (v6);
            \draw[cover] (v4) -- (v5);
            \node[below,align=center] at (\LabelX,\LabelY) [yshift=\LabelYShift] {\textcolor{mygreen}{\textbf{edge cover}} \\ \textcolor{mygreen}{minimal} \\ \textcolor{mygreen}{minimum}};
        \end{scope}
    \end{tikzpicture}

    \newpage
    % TODO: section

    \item \textbf{Cut vertex} (\textbf{articulation point})\Href{https://mathworld.wolfram.com/ArticulationVertex.html} is a vertex whose removal increases the number of connected components.

    \item \textbf{Bridge}\Href{https://en.wikipedia.org/wiki/Bridge_(graph_theory)} is an edge whose removal increases the number of connected components.

    \item \textbf{Biconnected graph}\Href{https://en.wikipedia.org/wiki/Biconnected_graph} is a connected \enquote{nonseparable} graph, which means that the removal of any vertex does not make the graph disconnected. Alternatively, this is a graph without \emph{cut vertices}.

    \item \textbf{Biconnectivity} can be defined as a relation on edges $R \subseteq E^2$:
    \begin{terms}
        \item Two edges are called \emph{biconnected} if there exist two \emph{vertex-disjoint} paths between the ends of these edges.
        \item Trivially, this relation is an equivalence relation.
        \item Equivalence classes of this relation are called \textbf{biconnected components}\Href{https://en.wikipedia.org/wiki/Biconnected_component}, also known as \textbf{blocks}.
    \end{terms}

    \item \textbf{Edge biconnectivity} can be defined as a relation on vertices $R \subseteq V^2$:
    \begin{terms}
        \item Two vertices are called \emph{edge-biconnected} if there exist two \emph{edge-disjoint} paths between them.
        \item Trivially, this relation is an equivalence relation.
        \item Equivalence classes of this relation are called \textbf{edge-biconnected components} (or \emph{2-edge-connected components}).
    \end{terms}

    \item \textbf{Vertex connectivity}\Href{https://en.wikipedia.org/wiki/Vertex_connectivity} $\vertexConnectivity{G}$ is the minimum number of vertices that has to be removed in order to make the graph disconnected or trivial (singleton).
    Equivalently, it is the largest~$k$ for which the graph~$G$ is $k$-vertex-connected.

    \item \textbf{$k$-vertex-connected graph}\Href{https://en.wikipedia.org/wiki/K-vertex-connected_graph} is a graph that remains connected after less than $k$ vertices are removed, \ie $\vertexConnectivity{G} \geq k$.
    \begin{terms}
        \item Corollary of Menger's theorem: graph $G = \Pair{V,E}$ is $k$-vertex-connected if, for every pair of vertices $u,v \in V$, it is possible to find $k$ \emph{vertex-independent} (\emph{internally vertex-disjoint}) paths between $u$ and~$v$.

        \item $k$-vertex-connected graphs are also called simply \emph{$k$-connected}.

        \item 1-connected graphs are called \emph{connected}, 2-connected are \emph{biconnected}, 3-connected are \emph{triconnected}, \textit{etc}.

        \item Note the \enquote{exceptions}:
        \begin{terms}
            \item Singleton graph $K_1$ has $\vertexConnectivity{K_1} = 0$, so it is \textbf{not} \emph{1-connected}, but still considered \emph{connected}.

            \item Graph $K_2$ has $\vertexConnectivity{K_2} = 1$, so it is \textbf{not} \emph{2-connected}, but considered \emph{biconnected}, so it can be a block.

            % TODO: what about K_3? Is it considered triconnected?
        \end{terms}
    \end{terms}

    \item \textbf{Edge connectivity}\Href{https://en.wikipedia.org/wiki/Edge_connectivity} $\edgeConnectivity{G}$ is the minimum number of edges that has to be removed in order to make the graph disconnected or trivial (singleton).
    Equivalently, it is the largest~$k$ for which the graph~$G$ is $k$-edge-connected.

    \item \textbf{$k$-edge-connected graph}\Href{https://en.wikipedia.org/wiki/K-edge-connected_graph} is a graph that remains connected after less than $k$ edges are removed, \ie $\edgeConnectivity{G} \geq k$.
    \begin{terms}
        \item Corollary of Menger's theorem: graph $G = \Pair{V,E}$ is $k$-edge-connected if, for every pair of vertices $u,v \in V$, it is possible to find $k$ \emph{edge-disjoint} paths between $u$ and~$v$.

        \item 2-edge-connected are called \emph{edge-biconnected}, 3-edge-connected are \emph{edge-triconnected}, \textit{etc}.

        \item Note the \enquote{exception}:
        \begin{terms}
            \item Singleton graph $K_1$ has $\edgeConnectivity{K_1} = 0$, so it is \textbf{not} \emph{2-edge-connected}, but considered \emph{edge-biconnected}, so it can be a \emph{2-edge-connected component}.
        \end{terms}
    \end{terms}

    \begin{minipage}{\linewidth}

    \setlength{\intextsep}{0pt}%
    % \setlength{\columnsep}{0pt}%
    \begin{wrapfigure}{r}{0pt}
        \tikz[]{
            \coordinate (origin);
            \node[dot] (a1) [position=36:.5 from origin] {};
            \node[dot] (a2) [position=-36:.5 from origin] {};
            \node[dot] (a3) [position=108:.5 from origin] {};
            \node[dot] (a4) [position=-108:.5 from origin] {};
            \node[dot] (a5) [position=180:.5 from origin] {};
            \node[dot] (x1) [right=.8 of a1] {};
            \node[dot] (x2) [right=.8 of a2] {};
            \node[dot] (x3) [right=.5 of x1] {};
            \node[dot] (x4) [right=.5 of x2] {};
            \draw (a1) edge (a3) edge (a4) edge (a5)
                -- (a2) edge (a4) edge (a5)
                -- (a3) edge (a5)
                -- (a4)
                -- (a5)
                -- cycle;
            \draw (x1) edge (x3) edge (x4)
                -- (x2) edge (x4)
                -- (x3)
                -- (x4)
                -- cycle;
            \draw (a1) edge (x1);
            \draw (a2) edge (x1) edge (x2);
            \node[below] at (current bounding box.south) {$\vertexConnectivity{G} = 2$, $\edgeConnectivity{G} = 3$, \\ $\minDegree{G} = 3$, $\maxDegree{G} = 6$};
        }
    \end{wrapfigure}

    \vspace{2pt}

    \item \textsc{Whitney's theorem}. For any graph $G$, $\vertexConnectivity{G} \leq \edgeConnectivity{G} \leq \minDegree{G}$.

    \end{minipage}

    % \begin{tabular}{ccc}
    %     \thead{Graph} & \thead{Vertex} & \thead{Edge} \\

    %     \begin{tikzpicture}
    %         \node[dot, label=below:$v_1$] {};
    %     \end{tikzpicture}

    % \end{tabular}


    \newpage
    % TODO: section

    \item \textbf{Tree}\Href{https://en.wikipedia.org/wiki/Tree_(graph_theory)} is a connected undirected acyclic graph.
    \item \textbf{Forest}\Href{https://en.wikipedia.org/wiki/Tree_(graph_theory)\#Forest} is an undirected acyclic graph, \ie a disjoint union of trees.

    \item An \textbf{unrooted tree} (\textbf{free tree}) is a tree without any designated \emph{root}.
    \item A \textbf{rooted tree} is a tree in which one vertex has been designated the \emph{root}.
    \begin{terms}
        \item In a rooted tree, the \textbf{parent} of a vertex $v$ is the vertex connected to $v$ on the path to the root.
        \item A \textbf{child} of a vertex $v$ is a vertex of which $v$ is the parent.
        \item A \textbf{sibling} to a vertex $v$ is any other vertex on the tree which has the same parent as $v$.

        % \item An \textbf{ascendant} of a vertex $v$ is any vertex which is either the parent of $v$ or is (recursively) the ascendant of the parent of $v$.
        % \item A \textbf{descendant} of a vertex $v$ is any vertex which is either the child of $v$ or is (recursively) the descendant of any of the children of $v$.

        \item A \textbf{leaf} is a vertex with no children. Equivalently, \textbf{leaf} is a \emph{pendant vertex}, \ie $\degree{v} = 1$.
        \item An \textbf{internal vertex} is a vertex that is not a leaf.

        \item A \textbf{$k$-ary tree} is a rooted tree in which each vertex has at most $k$ children. \emph{2-ary trees} are called \textbf{binary trees}.
    \end{terms}

    \item A \textbf{labeled tree}\Href{https://en.wikipedia.org/wiki/Labeled_tree} is a tree in which each vertex is given a unique \emph{label}, \eg $1, 2, \dotsc, n$.

    \item \textsc{Cayley's formula}\Href{https://en.wikipedia.org/wiki/Cayley's_formula}. Number of labeled trees on $n$ vertices is $n^{n-2}$.

    % Group for Prufer code definition
    \begingroup

    \colorlet{color-min}{red}
    \colorlet{color-parent}{blue}
    \colorlet{color-removed}{lightgray}
    \colorlet{color-last}{green!70!black}

    \item \textbf{Pr\"{u}fer code}\Href{https://en.wikipedia.org/wiki/Prufer_sequence} is a unique sequence of labels $\Set{1,\dotsc,n}$ of length $(n-2)$ associated with the labeled tree on $n$ vertices.
    \begin{terms}
        \item \textbf{\textsc{Encoding}} (iterative algorithm for converting tree $T$ labeled with $\Set{1,\dotsc,n}$ into a Pr\"{u}fer sequence $K$):
        \begin{terms}
            \item On each iteration, remove the leaf with \textcolor{color-min}{\emph{the smallest label}}, and extend $K$ with \textcolor{color-parent}{\emph{a single neighbour}} of this leaf.
            \item After $(n-2)$ iterations, the tree will be left with \textcolor{color-last}{\emph{two adjacent}} vertices \--- there is no need to encode them, because there is only one unique tree on 2 vertices, which requires 0 bits of information to encode.
        \end{terms}

        % Group for Prufer code pictures
        \begingroup

        \def\Dist{.5}
        \def\LabelY{-0.9}

        \tikzstyle{prufer}=[
            dot-removed/.style={dot,color-removed},
            dot-parent/.style={dot,color-parent},
            dot-min/.style={dot,color-min},
            dot-last/.style={dot,color-last},
            edge-removed/.style={color-removed,dashed},
            edge-min/.style={color-min,very thick},
            edge-last/.style={color-last,very thick},
        ]

        \tikz[baseline,prufer]{
            \node[dot] (2) [label={2}] {};
            \node[dot] (3) [label={3}, position=30:{\Dist} from 2] {};
            \node[dot] (7) [label={7}, position=-60:{\Dist} from 2] {};
            \node[dot] (6) [label={6}, position=-60:{\Dist} from 3] {};
            \node[dot] (5) [label={5}, position=-120:{\Dist} from 2] {};
            \node[dot] (1) [label={1}, position=180:{\Dist} from 5] {};
            \node[dot] (4) [label={4}, position=120:{\Dist} from 5] {};
            \draw (1) -- (5);
            \draw (4) -- (5);
            \draw (5) -- (2);
            \draw (2) -- (7);
            \draw (2) -- (3);
            \draw (3) -- (6);
            \node at (0,\LabelY) {$K = \emptyset$};
        }%
        \hfill%
        \tikz[baseline,prufer]{
            \node[dot] (2) [label={2}] {};
            \node[dot] (3) [label={3}, position=30:{\Dist} from 2] {};
            \node[dot] (7) [label={7}, position=-60:{\Dist} from 2] {};
            \node[dot] (6) [label={6}, position=-60:{\Dist} from 3] {};
            \node[dot-parent] (5) [label={[text=color-parent]5}, position=-120:{\Dist} from 2] {};
            \node[dot-min] (1) [label={[text=color-min]1}, position=180:{\Dist} from 5] {};
            \node[dot] (4) [label={4}, position=120:{\Dist} from 5] {};
            \draw[edge-min] (1) -- (5);
            \draw (4) -- (5);
            \draw (5) -- (2);
            \draw (2) -- (7);
            \draw (2) -- (3);
            \draw (3) -- (6);
            \node at (0,\LabelY) {$K = \mathcolor{blue}{5}$};
        }%
        \hfill%
        \tikz[baseline,prufer]{
            \node[dot] (2) [label={2}] {};
            \node[dot] (3) [label={3}, position=30:{\Dist} from 2] {};
            \node[dot] (7) [label={7}, position=-60:{\Dist} from 2] {};
            \node[dot] (6) [label={6}, position=-60:{\Dist} from 3] {};
            \node[dot-parent] (5) [label={[text=color-parent]5}, position=-120:{\Dist} from 2] {};
            \node[dot-removed] (1) [label={[text=color-removed]1}, position=180:{\Dist} from 5] {};
            \node[dot-min] (4) [label={[text=color-min]4}, position=120:{\Dist} from 5] {};
            \draw[edge-removed] (1) -- (5);
            \draw[edge-min] (4) -- (5);
            \draw (5) -- (2);
            \draw (2) -- (7);
            \draw (2) -- (3);
            \draw (3) -- (6);
            \node at (0,\LabelY) {$K = 5\mathcolor{blue}{5}$};
        }%
        \hfill%
        \tikz[baseline,prufer]{
            \node[dot-parent] (2) [label={[text=color-parent]2}] {};
            \node[dot] (3) [label={3}, position=30:{\Dist} from 2] {};
            \node[dot] (7) [label={7}, position=-60:{\Dist} from 2] {};
            \node[dot] (6) [label={6}, position=-60:{\Dist} from 3] {};
            \node[dot-min] (5) [label={[text=color-min]5}, position=-120:{\Dist} from 2] {};
            \node[dot-removed] (1) [label={[text=color-removed]1}, position=180:{\Dist} from 5] {};
            \node[dot-removed] (4) [label={[text=color-removed]4}, position=120:{\Dist} from 5] {};
            \draw[edge-removed] (1) -- (5);
            \draw[edge-removed] (4) -- (5);
            \draw[edge-min] (5) -- (2);
            \draw (2) -- (7);
            \draw (2) -- (3);
            \draw (3) -- (6);
            \node at (0,\LabelY) {$K = 55\mathcolor{blue}{2}$};
        }%
        \hfill%
        \tikz[baseline,prufer]{
            \node[dot] (2) [label={2}] {};
            \node[dot-parent] (3) [label={[text=color-parent]3}, position=30:{\Dist} from 2] {};
            \node[dot] (7) [label={7}, position=-60:{\Dist} from 2] {};
            \node[dot-min] (6) [label={[text=color-min]6}, position=-60:{\Dist} from 3] {};
            \node[dot-removed] (5) [label={[text=color-removed]5}, position=-120:{\Dist} from 2] {};
            \node[dot-removed] (1) [label={[text=color-removed]1}, position=180:{\Dist} from 5] {};
            \node[dot-removed] (4) [label={[text=color-removed]4}, position=120:{\Dist} from 5] {};
            \draw[edge-removed] (1) -- (5);
            \draw[edge-removed] (4) -- (5);
            \draw[edge-removed] (5) -- (2);
            \draw (2) -- (7);
            \draw (2) -- (3);
            \draw[edge-min] (3) -- (6);
            \node at (0,\LabelY) {$K = 552\mathcolor{blue}{3}$};
        }%
        \hfill%
        \tikz[baseline,prufer]{
            \node[dot-parent] (2) [label={[text=color-parent]2}] {};
            \node[dot-min] (3) [label={[text=color-min]3}, position=30:{\Dist} from 2] {};
            \node[dot] (7) [label={7}, position=-60:{\Dist} from 2] {};
            \node[dot-removed] (6) [label={[text=color-removed]6}, position=-60:{\Dist} from 3] {};
            \node[dot-removed] (5) [label={[text=color-removed]5}, position=-120:{\Dist} from 2] {};
            \node[dot-removed] (1) [label={[text=color-removed]1}, position=180:{\Dist} from 5] {};
            \node[dot-removed] (4) [label={[text=color-removed]4}, position=120:{\Dist} from 5] {};
            \draw[edge-removed] (1) -- (5);
            \draw[edge-removed] (4) -- (5);
            \draw[edge-removed] (5) -- (2);
            \draw (2) -- (7);
            \draw[edge-min] (2) -- (3);
            \draw[edge-removed] (3) -- (6);
            \node at (0,\LabelY) {$K = 5523\mathcolor{blue}{2}$};
        }%
        \hfill%
        \tikz[baseline,prufer]{
            \node[dot-last] (2) [label={[text=color-last]2}] {};
            \node[dot-removed] (3) [label={[text=color-removed]3}, position=30:{\Dist} from 2] {};
            \node[dot-last] (7) [label={[text=color-last]7}, position=-60:{\Dist} from 2] {};
            \node[dot-removed] (6) [label={[text=color-removed]6}, position=-60:{\Dist} from 3] {};
            \node[dot-removed] (5) [label={[text=color-removed]5}, position=-120:{\Dist} from 2] {};
            \node[dot-removed] (1) [label={[text=color-removed]1}, position=180:{\Dist} from 5] {};
            \node[dot-removed] (4) [label={[text=color-removed]4}, position=120:{\Dist} from 5] {};
            \draw[edge-removed] (1) -- (5);
            \draw[edge-removed] (4) -- (5);
            \draw[edge-removed] (5) -- (2);
            \draw[edge-last] (2) -- (7);
            \draw[edge-removed] (2) -- (3);
            \draw[edge-removed] (3) -- (6);
            \node at (0,\LabelY) {$K = \underline{55232}$};
        }

        \endgroup

        \item \textbf{\textsc{Decoding}} (iterative algorithm for converting a Pr\"{u}fer sequence $K$ into a tree $T$):
        \begin{terms}
            \item Given a Pr\"{u}fer code $K$ of length $(n-2)$, construct a set of \enquote{leaves} $W = \Set{1,\dotsc,n} \setminus K$.
            \item On each iteration:
            % TODO: fix margin
            \begin{enumerate}[label=(\arabic*)]
                \item Pop the \emph{first} element of $K$ (denote it as $k$) and the \emph{minimum} label in $W$ (denote it as $w$).
                \item Connect $k$ and $w$ with an edge $\Pair{k, w}$ in the tree $T$.
                \item If $k \notin K$, then extend the set of \enquote{leaves} $W \coloneqq W \union \Set{k}$.
            \end{enumerate}
            \item After $(n-2)$ iterations, the sequence $K$ will be empty, and the set $W$ will contain exactly two vertices \--- connect them with an edge.
        \end{terms}

        % TODO: visualization of decoding?
    \end{terms}
    \endgroup

    % \item \textit{To be continued...}

    % \vfill
    % \emph{TODO}

    % \begin{otherlanguage}{russian}

    % \item Операции над графами: объединение, пересечение
    % \item Двудольный граф (Bipartite graph)
    % \item Связность (connectivity)
    % \item Сильная связность
    % \item Отношение (вершинной/рёберной) k-связности (неэкв./экв.)
    % \item Теорема Менгере
    % \item Двусвязность
    % \item Точки сочленения (шарниры/cut vertex/articulation point)
    % \item Блоки (компоненты вершинной двусвязности)
    % \item Отношение вершинной двусвязности (экв.)
    % \item Мосты
    % \item Острова (компоненты рёберной двусвязности)
    % \item Отношение рёберной двусвязности (эквивалентно 2-рёб-связности)
    % \item Вес вершины в дереве
    % \item Центроид дерева

    % \end{otherlanguage}

\end{terms}

\end{document}
