\documentclass[a4paper,10pt]{article}
\usepackage{mypreamble}

%% Page setup
\geometry{
    margin=2cm,
    includehead,
    % includefoot,
    headsep=4mm,
}
\pagestyle{fancy}
\fancyfoot{}
\MakeSingleHeader% {<l>}{<r>}
    {\TextCheatsheetEng: Set Theory}
    {\TextDiscreteMathEng, \IconFall~Fall 2021}
% \fancyfoot[C]{\thepage\ of \zpageref{LastPage}}
% \fancyhead[C]{\includegraphics[height=14pt,raise=-2pt,set height=0pt,set depth=0pt]{img/set-theory.pdf}}

%% Add custom setup below

% \titlespacing{\type}{left}{above}{below}[right]
\titlespacing{\section}{0pt}{*1}{*0.5}
\titlespacing{\subsection}{0pt}{*1}{*0.5}

\setlength{\tabcolsep}{0pt}

\usepackage{venndiagram}
\usepackage{witharrows}
\usepackage{pdfcomment}

% %% Show page layout
% \usepackage{showframe}
% \renewcommand\ShowFrameLinethickness{0.1pt}
% \renewcommand*\ShowFrameColor{\color{lightgray}}


\begin{document}

\section{Set Theory Cheatsheet}

\subsection{Терминология и обозначения}

\begin{terms}
    \item \textbf{Множество} -- \emph{неупорядоченный} набор \emph{уникальных} элементов.
    \hfill\href{https://en.wikipedia.org/wiki/Set_(mathematics)}{Set}

    \item Множество может быть задано с помощью:
    \hfill\href{https://en.wikipedia.org/wiki/Set-builder_notation}{Set-builder notation}
    \begin{terms}
        \item перечисления элементов: $\Set{a_1, a_2, \dotsc, a_n}$ -- множество, состоящее из $n$ элементов $a_1, a_2, \dotsc, a_n$.
        \hfill\href{https://en.wikipedia.org/wiki/Urelement}{Urelement}
        \begin{terms}
            \item Например, $\Set{\square, \Cat, 42}$ -- множество, содержащее квадрат, кошку (или кота) и число 42.
            \hfill\href{https://ru.wikipedia.org/wiki/The_Ultimate_Question_of_Life,_the_Universe,_and_Everything}{42}
        \end{terms}

        \item характеристического свойства: $\Set{x \given P(x)}$ -- множество элементов, обладающих \textbf{свойством} $P$.
        \hfill\href{https://en.wikipedia.org/wiki/Predicate_(mathematical_logic)}{Predicate}
        \begin{terms}
            \item Например, $\Set{x \in \Natural \given \text{$x$ -- простое}}$ -- множество простых чисел.
            \hfill\href{https://en.wikipedia.org/wiki/Prime_number}{Prime number}
        \end{terms}
    \end{terms}

    \item $\emptyset = \Set{}$ -- \textbf{пустое} множество.
    \hfill\href{https://en.wikipedia.org/wiki/Empty_set}{Empty set}

    \item $\universalset$ -- \textbf{универсальное} множество (\textbf{универсум}).
    \hfill\href{https://en.wikipedia.org/wiki/Universal_set}{Universal set}

    \item $x \in A$ -- элемент $x$ \textbf{принадлежит} множеству $A$.
    \hfill\href{https://en.wikipedia.org/wiki/Element_(mathematics)}{Element}
    \begin{terms}
        \item \begin{tabular}{\ML{3cm} \ML{3cm} \ML{3cm} L}
            \pdftooltip%
                {1 \in \Set{1,2,3}}%
                {Почему? Потому что правое множество действительно содержит элемент \enquote{1}.}
            & \pdftooltip%
                {\square \in \Set{\triangle, \square, \circle}}%
                {Почему? Потому что правое множество дейсвительно содержит элемент \enquote{квадрат}.}
            & \pdftooltip%
                {1.25 \in \Rational}%
                {Почему? Потому что 1.25 действительно является рациональным числом (5/4).}
            & \pdftooltip%
                {2 \in \Set{x \in \Natural \given \text{$x$ -- простое}}}%
                {Почему? Потому что число 2 действительно считается простым числом.}
        \end{tabular}
    \end{terms}

    \item $x \notin A$ -- элемент $x$ \textbf{не принадлежит} множеству $A$.
    \begin{terms}
        \item \begin{tabular}{\ML{3cm} \ML{3cm} \ML{3cm} L}
            \pdftooltip%
                {9 \notin \Set{1,2,3}}%
                {Почему? Потому что правое множество действительно не содержит элемента \enquote{9}.}
            & \pdftooltip%
                {\Cat \notin \Set{\square, 42, \Set{\Cat}}}%
                {Почему? Потому что правое множество не содержит элемента \enquote{cat}.\textLF Правое множество содержит элемент \enquote{множество, содержащее \enquote{cat}}. Однако, здесь это не имеет значения.}
            & \pdftooltip%
                {\pi \notin \Rational}%
                {Почему? Потому что число Пи действительно не является рациональным.}
            & \pdftooltip%
                {42 \notin \Set{x \in \Natural \given \text{$x$ -- простое}}}%
                {Почему? Потому что число 42 действительно не является простым.}
        \end{tabular}
    \end{terms}

    \item $A \subseteq B$ -- множество $A$ является \textbf{подмножеством} множества $B$, т.е. $\forall x : x \in A \implies x \in B$.
    \hfill\href{https://en.wikipedia.org/wiki/Subset}{Subset}
    \begin{terms}
        \item \begin{tabular}{\ML{3cm} \ML{3cm} \ML{3cm} L}
            \pdftooltip%
                {\Set{a, b} \subseteq \Set{a, b, c}}%
                {Почему? Потому что элементы \enquote{a} и \enquote{b} (из левого множества) содержатся в правом множестве.}
            & \pdftooltip%
                {\Set{\Set{42}} \subseteq \Set{\Set{42}}}%
                {Почему? Потому что множества в левой и правой части равны.}
            & \pdftooltip%
                {\Set{\square} \nsubseteq \Set{a, \circle, 9}}%
                {Почему? Потмоу что элемент \enquote{квадрат} (из левого множества) не содержится в правом множестве.}
            & \pdftooltip%
                {\Set{5} \nsubseteq \Set{7, \Set{5}}}%
                {Почему? Потому что элемент \enquote{5} (из левого множества) не содержится в правом множестве как непосредственный элемент.}
        \end{tabular}
    \end{terms}

    \item $A \subset B$ -- множество $A$ является \textbf{строгим подмножеством} множества $B$, т.е. $A \subseteq B$ и $A \neq B$.
    \hfill\href{https://en.wikipedia.org/wiki/Strict_subset}{Strict subset}
    \begin{terms}
        \item \begin{tabular}{\ML{3cm} \ML{3cm} \ML{3cm} L}
            \pdftooltip%
                {\Set{c} \subset \Set{a, b, c}}%
                {Почему? Потому что элемент \enquote{c} содержится в правом множестве, а оба множества не равны.}
            & \pdftooltip%
                {\Set{42} \nsubset \Set{42}}%
                {Почему? Потому что, хоть элемент \enquote{42} (из левого множества) и содержится в правом множестве, но оба множества равны, что нарушает отношение \enquote{строгое подмножество}.}
            & \pdftooltip%
                {\Set{\Cat} \nsubset \Set{a, \circle, 9}}%
                {Почему? Потому что элемент \enquote{cat} (из левого множества) не содержится в правом множестве.}
            & \pdftooltip%
                {\Set{5} \nsubset \Set{7, \Set{5}}}%
                {Почему? Потому что элемент \enquote{5} (из левого множества) не содержится в правом множестве как непосредственный элемент.}
        \end{tabular}
    \end{terms}

    \item $A = B$ -- множества $A$ и $B$ содержат одинаковые элементы, т.е. $\forall x : x \in A \iff x \in B$.
    \hfill\href{https://en.wikipedia.org/wiki/Axiom_of_extensionality}{Extensionality}
    \begin{terms}
        \item \begin{tabular}{\ML{4.5cm} \ML{4.5cm} L}
            \pdftooltip%
                {\Set{\triangle, a, \Set{5}} = \Set{a, \Set{5}, \triangle}}%
                {Почему? Потому что множество -- \emph{неупорядоченный} набор элементов.\textLF В данном примере оба множества содержат одинаковый набор элементов.}
            & \pdftooltip%
                {\Set{2, \Set{\square, \square, \square}, 2} = \Set{2, \Set{\square}}}%
                {Почему? Потому что множество -- набор \emph{уникальных} элементов.\textLF В левой части некоторые элементы повторяются, однако, задаётся именно \enquote{множество} (т.к. используются фигурные скобки), поэтому на самом деле в обоих частях задаётся одно и то же множество ровно из двух элементов -- \enquote{числа 2} и \enquote{множества, содержащего квадрат}.}
            & \pdftooltip%
                {\Set{6, \emptyset} \neq \Set{6}}%
                {Почему? Потому что левое множество содержит два элемента: \enquote{число 6} и \enquote{пустое множество}, а правое -- только один: \enquote{число 6}.}
        \end{tabular}
    \end{terms}
\end{terms}


\subsection{Операции над множествами}

\begin{terms}
    \item $\card{A}$ -- \textbf{мощность} множества $A$ (число элементов).
    \hfill\href{https://en.wikipedia.org/wiki/Cardinality}{Cardinality}
    \begin{terms}
        \item \begin{tabular}{\ML{3cm} \ML{3cm} \ML{4cm} L}
            \pdftooltip%
                {\card{\Set{4,\square,d}} = 3}%
                {Почему? Потому что в заданном множестве действительно три элемента.}
            & \pdftooltip%
                {\card{\Set{1,9,9,9,1}} = 2}%
                {Почему? Потому что в заданном множестве (наборе уникальных элементов) действительно два различных элемента.}
            & \pdftooltip%
                {\card{\Set{\Set{a,b,c}, \Set{3,5,9}}} = 2}%
                {Почему? Потому что в заданном множестве действительно два элемента.\textLF Внутренняя структура этих элементов не оказывает влияния на мощность множества.}
            & \pdftooltip%
                {\card{\Set{1,\Set{2,3,4,\Set{5}}}} = 2}%
                {Почему? Потому что в заданном множестве действительно два элемента.\textLF Внутренняя структура этих элементов не оказывает влияния на мощность множества.}
        \end{tabular}
    \end{terms}

    \item $2^{A} = \powerset{A} = \Set{X \given X \subseteq A}$ -- \textbf{булеан} множества $A$ (множество всех подмножеств).
    \hfill\href{https://en.wikipedia.org/wiki/Power_set}{Powerset}
    \begin{terms}
        \item $\powerset{\Set{1,\square,\emptyset}} = \Set{\emptyset, ~\Set{1}, \Set{\square}, \Set{\emptyset}, ~\Set{1,\square}, \Set{1,\emptyset}, \Set{\square,\emptyset}, ~\Set{1,\square,3}}$
    \end{terms}

    \item $A \intersection B = \Set{x \given x \in A \land x \in B}$ -- \textbf{пересечение} множеств $A$ и $B$.
    \hfill\href{https://en.wikipedia.org/wiki/Intersection_(set_theory)}{Intersection}

    \item $A \union B = \Set{x \given x \in A \lor x \in B}$ -- \textbf{объединение} множеств $A$ и $B$.
    \hfill\href{https://en.wikipedia.org/wiki/Union_(set_theory)}{Union}

    \item $A \setminus B = A \intersection \overline{B} = \Set{x \given x \in A \;\land\; x \notin B}$ -- \textbf{разность} множеств $A$ и $B$ (дополнение $A$ до $B$).
    \hfill\href{https://en.wikipedia.org/wiki/Complement_(set_theory)}{Set difference}

    \item $\overline{A} = \universalset \setminus A = \Set{x \in \universalset \given x \notin A}$ -- \textbf{дополнение} (до универсума) множества $A$.
    \hfill\href{https://en.wikipedia.org/wiki/Complement_(set_theory)}{Complement}

    \item $A \symdiff B = (A \setminus B) \union (B \setminus A) = (A \union B) \setminus (A \intersection B)$ -- \textbf{симметрическая разность} множеств $A$~и~$B$.~
    \hfill\href{https://en.wikipedia.org/wiki/Symmetric_difference}{Symmetric difference}

    \item $A \times B = \Set{\Pair{a, b} \given a \in A,\, b \in B}$ -- \textbf{декартово произведение} множеств $A$ и $B$.
    \hfill\href{https://en.wikipedia.org/wiki/Cartesian_product}{Cartesian product}

    \item $\underbrace{A_1 \times \ldots \times A_n}_{n \text{ sets}} = \Set{\underbracket{\Tuple{a_1, \ldots, a_n}}_{n\text{-tuple}} \given a_i \in A_i,\, i \in [1; n]}$ -- \textbf{$n$-арное декартово произведение} множеств $A_1, \ldots, A_n$.

    \item $A^n = \underbrace{A \times \ldots \times A}_{n \text{ раз}} = \Set{\underbracket{\Tuple{a_1, \ldots, a_n}}_{n\text{-кортеж}} \given a_i \in A,\, i \in [1; n]}$\vspace{-4pt} -- \textbf{декартова степень} множества $A$.
    \hfill\href{https://en.wikipedia.org/wiki/Tuple}{Tuple}

    % \item $$
\end{terms}


\subsection{Некоторые свойства и законы}

\begin{multicols}{2}
\begin{terms}
    \item Свойства операций над множествами ($\forall A$):

    \begin{tabular}{L @{\hspace{1em}} L @{\hspace{1em}} L}
        A \union \emptyset = A
        & A \intersection \emptyset = \emptyset
        & A \symdiff \emptyset = A \\
        %
        A \union \universalset = \universalset
        & A \intersection \universalset = A
        & A \symdiff \universalset = \overline{A} \\
        %
        A \union A = A
        & A \intersection A = A
        & A \symdiff A = \emptyset \\
        %
        A \union \overline{A} = \universalset
        & A \intersection \overline{A} = \emptyset
        & A \symdiff \overline{A} = \universalset \\
        %
        % \smash{\overline{\overline{A}}} = A
        % \doverlinesmashed{A} = A
        \doverline{A} = A
        & \overline{\emptyset} = \universalset
        & \overline{\universalset} = \emptyset \\
        %
        \card{\emptyset} = 0
        & \card{2^{A}} = 2^{\card{A}}
        & \card{A^{n}} = \card{A}^n \\
        %
        \card{\Natural} = \card{\Rational} = \aleph_0
        & \card{\Real} = \mathfrak{c} = \card{2^{\Natural}} = \beth_1
        & \card{A \times B} = \card{A} \cdot \card{B} \\
        %
        \emptyset \subseteq A
        & 2^{\emptyset} = \Set{\emptyset}
        & A^0 = \Set{()} \\
    \end{tabular}

    \columnbreak

    \item Законы де Моргана:
    \hfill\href{https://en.wikipedia.org/wiki/De_Morgan's_laws}{De Morgan's laws}
    \begin{terms}
        \item $\overline{A \union B} = \overline{A} \intersection \overline{B}$
        \item $\overline{A \intersection B} = \overline{A} \union \overline{B}$
    \end{terms}

    \item Законы поглощения:
    \hfill\href{https://en.wikipedia.org/wiki/Absorption_law}{Absorption law}
    \begin{terms}
        % \item $A \union (A \intersection B) = A \intersection (A \union B) = A$
        \item $A \union (A \intersection B) = A$
        \item $A \intersection (A \union B) = A$
    \end{terms}

    \item Мистические законы:
    \begin{terms}
        \item $A \union (\overline{A} \intersection B) = A \union B$
        \item $A \intersection (\overline{A} \union B) = A \intersection B$
    \end{terms}

    % \columnbreak

    % \item Транзитивные законы:
    % \hfill\href{https://en.wikipedia.org/wiki/Transitive_relation}{Transitive relation}
    % \begin{terms}
    %     \item $(A \subset B) \land (B \subset C) \implies (A \subset C)$
    %     \item $(A \subseteq B) \land (B \subseteq C) \implies (A \subseteq C)$
    %     \item $(A = B) \land (B = C) \implies (A = C)$
    % \end{terms}

    % % \columnbreak

    % \item Коммутативные законы:
    % \hfill\href{https://en.wikipedia.org/wiki/Commutative_property}{Commutativity}
    % \begin{terms}
    %     \item $A \union B = B \union A$
    %     \item $A \intersection B = B \intersection A$
    %     \item $A \symdiff B = B \symdiff A$
    %     % \item $A \setminus B \neq B \setminus A$
    %     % \item $A \times B \neq B \times A$
    % \end{terms}

    % \item Ассоциативные законы:
    % \hfill\href{https://en.wikipedia.org/wiki/Associative_property}{Associativity}
    % \begin{terms}
    %     \item $A \union (B \union C) = (A \union B) \union C$
    %     \item $A \intersection (B \intersection C) = (A \intersection B) \intersection C$
    %     \item $A \symdiff (B \symdiff C) = (A \symdiff B) \symdiff C$
    %     % \item $A \setminus (B \setminus C) \neq (A \setminus B) \setminus C$ \quad
    %     % \item $A \times (B \times C) \neq (A \times B) \times C \neq A \times B \times C$
    % \end{terms}

    % \item Дистрибутивные законы:
    % \hfill\href{https://en.wikipedia.org/wiki/Distributive_property}{Distributivity}
    % \begin{terms}
    %     \item $A \union (B \intersection C) = (A \union B) \intersection (A \union C)$
    %     \item $A \intersection (B \union C) = (A \intersection B) \union (A \intersection C)$
    %     \item $A \intersection (B \symdiff C) = (A \intersection B) \symdiff (A \intersection C)$
    %     \item $A \times (B \union C) = (A \times B) \union (A \times C)$
    %     \item $A \times (B \intersection C) = (A \times B) \intersection (A \times C)$
    %     \item $A \times (B \setminus C) = (A \times B) \setminus (A \times C)$
    % \end{terms}
\end{terms}
\end{multicols}


%% Ensure page break
\newpage


\subsection{Диаграммы Венна%
\texorpdfstring{\hfill\normalfont\href{https://en.wikipedia.org/wiki/Venn_diagram}{Venn diagram}}{}}

% Definitions
\def\firstcircle{(0,0) circle[radius=1]}
\def\secondcircle{(1.2,0) circle[radius=1]}
\newcommand\drawcircles[1][]{% [<style>]
    \draw[outline,#1]
        \firstcircle  node[left]  {$A$}
        \secondcircle node[right] {$B$}; }
\newcommand\drawuniverse[1][]{% [<style>]
    \draw[universe,#1] (-1.1,-1.1) rectangle (2.3,1.6); }
\newcommand\drawlabel[1]{% {<label>}
    \node[above] at (current bounding box.north) {#1}; }

\tikzset{
    myvennstyle/.style={
        scale=0.95,
        universe/.style={draw=blue!70},
        outline/.style={draw=blue!50, thick},
        filled/.style={fill=blue!20},
    },
}

\begin{figure}[H]
\centering%
\begin{tikzpicture}[myvennstyle]
    \begin{scope}
        \clip \firstcircle;
        \fill[filled] \secondcircle;
    \end{scope}
    \drawcircles
    \drawlabel{$A \intersection B$}
    \drawuniverse
\end{tikzpicture}%
\hfill%
\begin{tikzpicture}[myvennstyle]
    \drawcircles[filled]
    \drawlabel{$A \union B$}
    \drawuniverse
\end{tikzpicture}%
\hfill%
\begin{tikzpicture}[myvennstyle]
    \drawcircles[filled,even odd rule]
    \drawlabel{$A \symdiff B$}
    \drawuniverse
\end{tikzpicture}%
\hfill%
\begin{tikzpicture}[myvennstyle]
    \begin{scope}
        \clip \firstcircle;
        \draw[filled,even odd rule]
            \firstcircle
            \secondcircle;
    \end{scope}
    \drawcircles
    \drawlabel{$A \setminus B$}
    \drawuniverse
\end{tikzpicture}%
\hfill%
\begin{tikzpicture}[myvennstyle]
    \begin{scope}
        \clip \secondcircle;
        \draw[filled,even odd rule]
            \firstcircle
            \secondcircle;
    \end{scope}
    \drawcircles
    \drawlabel{$B \setminus A$}
    \drawuniverse
\end{tikzpicture}%
\end{figure}

\vspace{4pt plus 6pt}

\begin{adjustbox}{valign=t, minipage=0.2\textwidth}
    \begin{venndiagram3sets}[
        tikzoptions={
            scale=0.9,
        },
        labelOnlyA={$1$},
        labelOnlyB={$2$},
        labelOnlyC={$3$},
        labelOnlyAB={$4$},
        labelOnlyBC={$5$},
        labelOnlyAC={$6$},
        labelABC={$7$},
        labelNotABC={$8$},
        radius=1cm,
        overlap=0.8cm,
        vgap=0.2cm,
        hgap=0.2cm
    ]
        \begin{scope}[blend mode=soft light]
            \setkeys{venn}{shade=red!20}
            \fillA
            \setkeys{venn}{shade=green!20}
            \fillB
            \setkeys{venn}{shade=blue!20}
            \fillC
        \end{scope}
    \end{venndiagram3sets}
\\[16pt]
    \begin{venndiagram3sets}[
        tikzoptions={
            remember picture,
            scale=0.9,
        },
        labelOnlyA={$1$},
        labelOnlyB={$2$},
        labelOnlyC={$3$},
        labelOnlyAB={$4$},
        labelOnlyBC={$5$},
        labelOnlyAC={$6$},
        labelABC={$7$},
        labelNotABC={$8$},
        radius=1cm,
        overlap=0.8cm,
        vgap=0.2cm,
        hgap=0.2cm
    ]
        \setkeys{venn}{shade=green!20}
        \fillOnlyA
        \fillACapBNotC
        \fillACapCNotB
        \fillNotABC
        \setpostvennhook{
            \coordinate (venn1) at (current bounding box.east);
        }
    \end{venndiagram3sets}
\\[16pt]
    \begin{venndiagram3sets}[
        tikzoptions={
            remember picture,
            scale=0.9,
        },
        labelOnlyA={$1$},
        labelOnlyB={$2$},
        labelOnlyC={$3$},
        labelOnlyAB={$4$},
        labelOnlyBC={$5$},
        labelOnlyAC={$6$},
        labelABC={$7$},
        labelNotABC={$8$},
        radius=1cm,
        overlap=0.8cm,
        vgap=0.2cm,
        hgap=0.2cm
    ]
        \setkeys{venn}{shade=green!20}
        \fillOnlyA
        \fillBCapCNotA
        \fillACapCNotB
        \setpostvennhook{
            \coordinate (venn2) at (current bounding box.east);
        }
    \end{venndiagram3sets}
\end{adjustbox}%
\hfill%
\begin{adjustbox}{valign=t, minipage=0.8\textwidth}
    На предоставленной слева диаграмме Венна для трёх множеств $A$, $B$, $C$ и универсума~$\universalset$ области отмечены номерами.
    Для заданного списка областей нарисуйте диаграмму Венна и составьте соответствующую формулу, используя термы $A$, $B$, $C$, $\overline{A}$, $\overline{B}$, $\overline{C}$ и операторы $\union$,~$\intersection$.

    \begin{enumerate}[leftmargin=1pc, itemsep=4pt]
        \item \(\begin{WithArrows}[more-columns,command-name=Explain,group]
            \smash[b]{\underbrace{\mathcal{S}(1,4,6,8)}_{\tikz[remember picture, overlay] \draw[-latex, shorten >=2pt] (0,1ex) to[bend left] (venn1);}}
            &= \mathcal{S}(1,4,6) + \mathcal{S}(8) = {}
            ~\dslash\; \text{\href{https://bit.ly/2nWtPaq}{Wolfram}} \;\dslash \\
            %
            &= \begin{aligned}[t]
                \dslash\;& \mathcal{S}(1,4,6) = \text{$A$ without $ABC$}, \\
                & \mathcal{S}(8) = \text{outside of $(A+B+C)$} \;\dslash {}={} \\
            \end{aligned} \\
            %
            &= (A - ABC) + \overline{A+B+C} = {}
            \Explain{правило: $A - B = A \overline{B}$)} \\
            %
            &= A\overline{ABC} + \overline{A+B+C} = {}
            \Explain{два де Моргана} \\
            %
            &= A \cdot (\overline{A} + \overline{B} + \overline{C}) + \overline{A} \cdot \overline{B} \cdot \overline{C} = {}
            \Explain{раскрываем скобку, сокращаем $A\overline{A} = \emptyset$} \\
            %
            &= \cancelto{\smash{\emptyset}}{A\overline{A}} + A\overline{B} + A\overline{C} + \overline{A} \cdot \overline{B} \cdot \overline{C} = {}
            \Explain{выносим $\overline{B}$ за скобку} \\
            %
            &= \overline{B} \cdot (A + \overline{A} \cdot \overline{C}) + A\overline{C} = {}
            \Explain{закон поглощения: $A + \overline{A}B = A + B$} \\
            %
            &= \overline{B} \cdot (A + \overline{C}) + A\overline{C} = {}
            \Explain{раскрываем скобку} \\
            %
            &= \uwave{A \cdot \overline{B} + \overline{B} \cdot \overline{C} + A \cdot \overline{C}}
        \end{WithArrows}\)

        \item \(\begin{WithArrows}[more-columns,command-name=Explain,group]
            \smash[b]{\underbrace{\mathcal{S}(1,5,6)}_{\tikz[remember picture, overlay] \draw[-latex, shorten >=2pt] (0,1ex) to[bend left] (venn2);}}
            &= \mathcal{S}(1,6) + \mathcal{S}(5) = {}
            ~\dslash\; \text{\href{https://bit.ly/2o3gmNT}{Wolfram}} \;\dslash \\
            %
            &= \begin{aligned}[t]
                \dslash\;& \mathcal{S}(1,6) = \text{$A$ without $AB$}, \\
                & \mathcal{S}(5) = \text{$BC$ without $ABC$} \;\dslash {}={} \\
            \end{aligned} \\
            %
            &= (A - AB) + (BC - ABC) = {}
            \Explain{правило: $A - B = A \overline{B}$)} \\
            %
            &= A\overline{AB} + BC\overline{ABC} = {}
            \Explain{два де Моргана} \\
            %
            &= A \cdot (\overline{A} + \overline{B}) + BC \cdot (\overline{A} + \overline{B} + \overline{C}) = {}
            \Explain{раскрываем скобку} \\
            %
            &= \cancelto{\smash{\emptyset}}{A\overline{A}} + A\overline{B} + \overline{A}BC + \cancelto{\smash{\emptyset}}{BC\overline{B}} + \cancelto{\smash{\emptyset}}{BC\overline{C}} = {}
            \Explain{сокращаем $A\overline{A} = \emptyset$} \\
            %
            &= \uwave{A\overline{B} + \overline{A}BC}
        \end{WithArrows}\)
    \end{enumerate}
\end{adjustbox}


% \subsection{Диаграммы (круги) Эйлера%
% \texorpdfstring{\hfill\normalfont\href{https://en.wikipedia.org/wiki/Euler_diagram}{Euler diagram}}{}}

% \tikzset{
%     myeulerstyle/.style={
%         scale=0.7,
%         colorA/.style={blue},
%         colorB/.style={green},
%         colorC/.style={red},
%         % colorU/.style={black},
%         colorU/.style={},
%         colorAfill/.style={fill=blue!10},
%         colorBfill/.style={fill=green!10},
%         colorCfill/.style={fill=red!10},
%         % colorUfill/.style={fill=gray!10},
%         colorUfill/.style={},
%     }
% }

% \adjustbox{valign=t}{\begin{minipage}{0.2\textwidth}%
%     \begin{tikzpicture}[myeulerstyle]
%         \coordinate (botleft) at (-1.7,-2.5);
%         \coordinate (topright) at (3.4,2.5);
%         \draw[colorUfill] (botleft) rectangle (topright);
%         \draw[colorCfill] (0.8,0) circle[radius=2.2];
%         \draw[colorBfill] (0.4,0) circle[radius=1.6];
%         \draw[colorAfill] (0,0) circle[radius=1];
%         \node at (0,0) {$A$};
%         \node[right] at (1,0) {$B$};
%         \node[right] at (2,0) {$C$};
%         \node[below left] at (topright) {$\universalset$};
%         \drawlabel{$A \subset B \subset C \subset \universalset$}
%     \end{tikzpicture}
% \end{minipage}}%
% \hfill%
% \begin{adjustbox}{valign=t,minipage=0.75\textwidth}
%     Множества $A \subset B \subset C \subset \universalset$ возможно изобразить на отрезке, что позволит графически упрощать выражения, содержащие операцию пересечения ($\intersection$):

%     \begin{adjustbox}{valign=t,minipage=0.32\textwidth}
%         \begin{tikzpicture}[
%             myeulerstyle,
%             myline/.style={thick},
%         ]
%             \def\YA{-0.4}
%             \def\YB{-1}
%             \def\YC{-1.6}
%             \def\YU{-2.2}
%             \foreach \x/\s in {0/{\emptyset},1/a,2/b,3/c,4/\mathfrak{u}} {
%                 \draw (\x,-0.1) -- (\x,0.1) node[above] {$\s$};
%             }
%             \draw[dashed]
%                 (0,0) -- +(0,\YU)
%                 ++(1,0) -- +(0,\YA)
%                 ++(1,0) -- +(0,\YB)
%                 ++(1,0) -- +(0,\YC)
%                 ++(1,0) -- +(0,\YU)
%             ;
%             \draw[very thick] (0,0) -- (4,0);
%             \draw[myline,colorA] (0,\YA) -- node[below] {$A$} (1,\YA);
%             \draw[myline,colorB] (0,\YB) -- node[below] {$B$} (2,\YB);
%             \draw[myline,colorC] (0,\YC) -- node[below] {$C$} (3,\YC);
%             \draw[myline,colorU] (0,\YU) -- node[below] {$\universalset$} (4,\YU);
%         \end{tikzpicture}
%     \end{adjustbox}%
%     \hfill%
%     \begin{adjustbox}{valign=t,minipage=0.32\textwidth}
%         \begin{tikzpicture}[
%             myeulerstyle,
%             myline/.style={thick},
%         ]
%             \def\YA{-0.4}
%             \def\YB{-0.9}
%             \def\YC{-1.4}
%             \def\YU{-1.9}
%             \foreach \x/\s in {0/{\emptyset},1/a,2/b,3/c,4/\mathfrak{u}} {
%                 \draw (\x,-0.1) -- (\x,0.1) node[above] {$\s$};
%             }
%             \draw[dashed]
%                 (0,0) -- +(0,\YC)
%                 ++(1,0) -- +(0,\YA)
%                 ++(1,0) -- +(0,\YB)
%                 ++(1,0) -- +(0,\YC)
%                 ++(1,0) -- +(0,\YB)
%             ;
%             \draw[very thick] (0,0) -- (4,0);
%             \draw[myline,colorA] (1,\YA) -- node[below,pos=0.15] {$\overline{A}$} (4,\YA);
%             \draw[myline,colorB] (2,\YB) -- node[below,pos=0.75] {$\overline{B}$} (4,\YB);
%             \draw[myline,colorC] (0,\YC) -- node[below] {$C$} (3,\YC);
%             \node[above] at (current bounding box.north) {$P = \overline{A} \intersection \overline{B} \intersection C = $};
%         \end{tikzpicture}
%     \end{adjustbox}%
%     \hfill%
%     \begin{adjustbox}{valign=t,minipage=0.32\textwidth}%
%         \begin{tikzpicture}[
%             %
%         ]
%             \draw (0,0) -- (1,1);
%         \end{tikzpicture}
%     \end{adjustbox}%
% \end{adjustbox}


\subsection{Декартово произведение множеств на плоскости \texorpdfstring{$\Real^2$}{R\string^2}%
\texorpdfstring{\hfill\normalfont\href{https://en.wikipedia.org/wiki/Real_coordinate_space}{$\Real^2$ coordinate space}}{}}

Декартово произведение двух множеств -- множество пар. Если представить, что такие пары -- элементы пространства~$\Real^2$ (точки на плоскости), то возможна следующая геометрическая интерпретация:

\tikzset{
    mycartesianstyle/.style={
        % scale=0.95,
        circ/.style={draw,solid,circle,inner sep=0pt,minimum size=4pt},
        filled/.style={fill=blue!20},
        alone/.style={circ,filled},
        included/.style={circ,fill=black},
        excluded/.style={circ,fill=white},
    }
}

\begin{figure}[!htb]
\centering
\hfill%
\begin{tikzpicture}[mycartesianstyle]
    % Grid
    \draw[help lines] (0,0) grid (5.7,4.7);
    % X axis
    \draw[-latex] (-0.3,0) -- (5.7,0) node[below] {$x$};
    \foreach \x in {1,2,3,4,5} {
        \draw (\x,0.1) -- (\x,-0.1) node[below] {\x};
    }
    % Y axis
    \draw[-latex] (0,-0.3) -- (0,4.7) node[left] {$y$};
    \foreach \x in {1,2,3,4} {
        \draw (0.1,\x) -- (-0.1,\x) node[left] {\x};
    }
    % Zero label
    \node[below left] (-0.1,-0.1) {0};
    % Sideway A vector
    \draw[thick,] (1,-0.7) node[included]{} -- node[below] {$A = [1; 4)$} (4,-0.7) node[excluded]{};
    % Sideway B vector
    \draw[thick] (-0.7,2) node[excluded]{} -- node[rotate=90,above,midway] {$B = (2; 4]$} (-0.7,4) node[included]{};
    % Filling
    \path[filled] (1,2) rectangle (4,4);
    % Edges
    \draw[thick]
        (1,2)
        --
        (1,4) node[included]{}
        --
        (4,4)
    ;
    \draw[thick,dashed]
        (1,2) node[excluded]{}
        --
        (4,2) node[excluded]{}
        -- node[right,fill=white] {$A \times B$}
        (4,4) node[excluded]{}
    ;
    % Additional C^2 point
    \draw[thick] (1,1) node[alone]{} node[right,xshift=2pt,fill=white] {$C^2 = \Set{\Pair{1,1}}$};
    % Caption
    \drawlabel{\smash[b]{$(A \times B) \union C^2 = \Bigl( [1; 4) \times (2; 4] \Bigr) \union \Set{1}^2$}};
\end{tikzpicture}%
\hfill%
\begin{tikzpicture}[mycartesianstyle]
    \def\Aleft{1}
    \def\Aright{5}
    \def\Bleft{1}
    \def\Bright{4}
    \def\Cleft{2}
    \def\Cright{3}
    \def\Dleft{2}
    \def\Dright{3}
    \def\Xside{-0.7}
    \def\Xsidetwo{-1.4}
    \def\Yside{-0.7}
    \def\Ysidetwo{-1}
    \def\axisovershoot{0.2}
    % Grid
    \draw[help lines] (0,0) grid (5.7,4.7);
    % X axis
    \draw[-latex] (-\axisovershoot,0) -- (5.7,0) node[below] {$x$};
    \foreach \x in {1,2,3,4,5} {
        \draw (\x,0.1) -- (\x,-0.1) node[below] {\x};
    }
    % Y axis
    \draw[-latex] (0,-\axisovershoot) -- (0,4.7) node[left] {$y$};
    \foreach \x in {1,2,3,4} {
        \draw (0.1,\x) -- (-0.1,\x) node[left] {\x};
    }
    % Zero label
    \node[below left] (-0.1,-0.1) {0};
    % Sideway A vector
    \draw[thick] (\Aleft,\Xside) node[excluded]{} -- node[below,pos=0.75] {$A = (\Aleft; \Aright]$} (\Aright,\Xside) node[included]{};
    % Sideway B vector
    \draw[thick] (\Xside,\Bleft) node[excluded]{} -- node[rotate=90,above] {$B = (\Bleft; \Bright]$} (\Xside,\Bright) node[included]{};
    % Sideway C vector
    \draw[thick,overlay] (\Cleft,\Ysidetwo) node[included]{} -- node[below] {$C = [\Cleft; \Cright]$} (\Cright,\Ysidetwo) node[included]{};
    % Sideway D vector
    \draw[thick] (\Xsidetwo,\Dleft) node[excluded]{} -- node[rotate=90,above] {$D = (\Dleft; \Dright)$} (\Xsidetwo,\Dright) node[excluded]{};
    % Filling
    \path[filled,even odd rule] (\Aleft,\Bleft) rectangle (\Aright,\Bright) (\Cleft,\Dleft) rectangle (\Cright,\Dright);
    % Outer edges
    \draw[thick]
        (\Aleft,\Bright)
        --
        (\Aright,\Bright) node[included]{}
        --
        (\Aright,\Bleft)
    ;
    \draw[thick,dashed]
        (\Aleft,\Bright) node[excluded]{}
        --
        (\Aleft,\Bleft) node[excluded]{}
        -- node[below,fill=white] {$A \times B$}
        (\Aright,\Bleft) node[excluded]{}
    ;
    % Inner edges
    \draw[thick]
        (\Cleft,\Dleft)
        --
        (\Cright,\Dleft)
        (\Cleft,\Dright)
        -- node[above] {$C \times D$}
        (\Cright,\Dright)
    ;
    \draw[thick,dashed]
        (\Cleft,\Dleft) node[included]{}
        --
        (\Cleft,\Dright) node[included]{}
        (\Cright,\Dleft) node[included]{}
        --
        (\Cright,\Dright) node[included]{}
    ;
    % Label
    \node[rotate=50] at (3.9,2.5) {$(A \times B) \setminus (C \times D)$};
    % Caption
    \drawlabel{\smash[b]{$(A \times B) \setminus (C \times D) = \Bigl( (\Aleft; \Aright] \times (\Bleft; \Bright] \Bigr) \setminus \Bigl( [\Cleft; \Cright] \times (\Dleft; \Dright) \Bigr)$}};
    % Dashed lines from sideway vectors onto the axes
    % \begin{scope}[on background layer]
    % \path[dashed,help lines] (\Aleft,\Yside) edge[bend left] (\Aleft,\Bleft);
    % \path[dashed,help lines] (\Aright,\Yside) edge[bend left] (\Aright,\Bleft);
    % \path[dashed,help lines] (\Xside,\Bleft) edge[bend left] (\Aleft,\Bleft);
    % \path[dashed,help lines] (\Xside,\Bright) edge[bend left] (\Aleft,\Bright);
    % \path[dashed,help lines] (\Cleft,\Ysidetwo) edge[bend left] (\Cleft,\Dleft);
    % \path[dashed,help lines] (\Cright,\Ysidetwo) edge[bend left] (\Cright,\Dleft);
    % \path[dashed,help lines] (\Xsidetwo,\Dleft) edge[bend right] (\Cleft,\Dleft);
    % \path[dashed,help lines] (\Xsidetwo,\Dright) edge[bend left] (\Cleft,\Dright);
    % \end{scope}
\end{tikzpicture}%
\hfill%
\null%
\end{figure}

\end{document}
