\documentclass[a4paper,10pt]{article}
\usepackage{mypreamble}

%% Page setup
\geometry{
    margin=2cm,
    includehead,
    includefoot,
    headsep=8pt,
    footskip=16pt,
}
\pagestyle{fancy}
\MakeSingleHeader% {<l>}{<r>}
    {\TextCheatsheetEng: Combinatorics}%
    {\TextDiscreteMathEng, \IconSpring~Spring 2024}
\fancyfoot{}
\fancyfoot[L]{\tiny Build time: \DTMnow}
\fancyfoot[R]{\tiny Source code can be found at \url{https://github.com/Lipen/discrete-math-course}}
% \fancyfoot[C]{\thepage\ of \zpageref{LastPage}}

%% Add custom setup below

% \titlespacing{\type}{left}{above}{below}[right]
\titlespacing{\section}{0pt}{*1}{*0.5}
\titlespacing{\subsection}{0pt}{*1}{*0.5}

\declaretheoremstyle[
    spaceabove=6pt,
    spacebelow=6pt,
    postheadspace=0.5em,
    notefont=\normalfont\scshape,
]{mystyle}
\declaretheorem[style=mystyle]{theorem}


\begin{document}
\selectlanguage{english}

\setcounter{section}{5}
\section{Combinatorics Cheatsheet}

\subsection{Overview}

\begin{terms}
    \item \textbf{Combinatorics} is a field of mathematics concerned with:
    \begin{terms}
        \item Arrangements of elements of a set into patterns satisfying specific rules, generally referred to as \emph{discrete structures}.
        \item The existence, enumeration, analysis, classification and optimization of discrete structures.
        \item Generalizations and specializations of relations between discrete structures.
    \end{terms}
\end{terms}

\subsection{Permutations and Combinations}

\subsubsection{Basic Counting Principles}

\noindent TODO: addition, multiplication, subtraction, bijection, pigeonhole, double counting

\subsubsection{Ordered Arrangements}

\noindent TODO: string, tuple, sequence, map (function), k-permutation, circular permutation

\begin{terms}
    \item ...
\end{terms}

\subsubsection{Unordered Arrangements}

\noindent TODO: subset, k-combination, multiset, k-comb of a multiset, k-perm of a multiset, binomial coefficient

\begin{terms}
    \item ...
\end{terms}

\subsubsection{Multinomial Coefficients}

\begin{terms}
    \item ...
\end{terms}

\subsubsection{The Twelvefold Way%
\texorpdfstring{\hfill\normalsize\href{https://en.wikipedia.org/wiki/Twelvefold_way}{Twelvefold way}}{}}

\begin{terms}
    \item ...
\end{terms}

\subsection{Inclusion--Exclusion Principle%
\texorpdfstring{\hfill\normalsize\href{https://en.wikipedia.org/wiki/Inclusion-exclusion_principle}{PIE}}{}}

\noindent TODO: PIE, M\"{o}bius Inversion

% TODO: \subsection{Generating Functions}
% TODO: \subsection{Partitions}

\end{document}
