\documentclass[a4paper,12pt]{article}
\usepackage{mypreamble}

%% Page setup
\geometry{
    margin=2cm,
    includehead,
    % includefoot,
    headsep=\baselineskip,
}
\pagestyle{fancy}
\fancyfoot{}
\MakeDoubleHeader% {<l1>}{<l2>}{<r1>}{<r2>}
    {\TextHomeworkEng~\#5}
    {Graph Theory}
    {\TextDiscreteMathEng}
    {\IconSpring~Spring 2024}

%% Add custom setup below

\newcommand{\MyGraph}{\mathcal{G}}
\newcommand{\MyGraphFull}{\MyGraph^{*}}
\newcommand{\op}[1]{\operatorname*{#1}}
\newcommand{\dist}[1]{\op{dist}(#1)}
\newcommand{\degree}[1]{\op{deg}(#1)}
\newcommand{\minDegree}[1]{\delta(#1)}
\newcommand{\maxDegree}[1]{\Delta(#1)}
\newcommand{\eccentricity}[1]{\varepsilon(#1)}
\newcommand{\graphRadius}[1]{\op{rad}(#1)}
\newcommand{\graphDiameter}[1]{\op{diam}(#1)}
\newcommand{\graphGirth}[1]{\op{girth}(#1)}
\newcommand{\graphCenter}[1]{\op{center}(#1)}
\newcommand{\graphCentroid}[1]{\op{centroid}(#1)}
% \newcommand{\stabilitynumber}[1]{\alpha(#1)}
% \newcommand{\matchingnumber}[1]{\alpha'(#1)}
\newcommand{\vertexConnectivity}[1]{\varkappa(#1)}
\newcommand{\edgeConnectivity}[1]{\lambda(#1)}
\newcommand{\blockGraph}[1]{\op{B}(#1)}

\declaretheoremstyle[
    spaceabove=6pt,
    spacebelow=6pt,
    postheadspace=0.5em,
    notefont=\normalfont\scshape,
]{mystyle}
\declaretheorem[style=mystyle]{theorem}

\usepackage[linesnumbered,ruled,vlined]{algorithm2e}

\tikzset{
    position/.style args={#1:#2 from #3}{
        at=(#3.#1), anchor=#1+180, shift=(#1:#2)
    },
}

\tikzstyle{mygraphstyle}=[
    auto,
    on grid,
    dot/.style={
        draw,
        fill=black,
        shape=circle,
        minimum size=4pt,
        inner sep=0pt,
        outer sep=0pt,
    },
    every label/.append style={
        text depth=0pt,
    },
]


\begin{document}
\selectlanguage{english}

\begin{tasks}

    \item For each of the following graphs, find $\vertexConnectivity{G}$, $\edgeConnectivity{G}$, $\minDegree{G}$, $\eccentricity{v}$ of each vertex $v \in V(G)$, $\graphRadius{G}$, $\graphDiameter{G}$, $\graphCenter{G}$.
    Find Euler path, Euler circuit, and Hamiltonian cycle, if they exist.
    In addition, find maximum clique $Q \subseteq V$, maximum stable set $S \subseteq V$, maximum matching $M \subseteq E$, minimum dominating set $D \subseteq V$, minimum vertex cover $R \subseteq V$, minimum edge cover $F \subseteq E$ of $G$.

    \begin{multicols}{2}
    \begin{subtasks}
        \item \tikz[mygraphstyle, baseline=(b.south)]{
            \node[dot] (b) [label=above:{b}] {};
            \node[dot] (a) [label=left:{a}, above left=8mm and 16mm of b] {};
            \node[dot] (c) [label=left:{c}, above right=8mm and 16mm of b] {};
            \node[dot] (e) [label=right:{e}, below left=8mm and 16mm of b] {};
            \node[dot] (d) [label=right:{d}, below right=8mm and 16mm of b] {};
            \draw (a) -- (b) -- (c) -- (d) -- (b) -- (e) -- (a);
        }

        \item \tikz[mygraphstyle, baseline=(c.south)]{
            \node[dot] (a) [label=above:{$a$}] {};
            \node[dot] (b) [label=above:{$b$}, right=12mm of a] {};
            \node[dot] (d) [label=right:{$d$}, below=8mm of a] {};
            \node[dot] (c) [label=left:{$c$}, left=8mm of d] {};
            \node[dot] (h) [label=right:{$h$}, below=12mm of d] {};
            \node[dot] (g) [label=left:{$g$}, left=8mm of h] {};
            \node[dot] (k) [label={[text height=1.5ex]below:{$k$}}, below=8mm of h] {};
            \node[dot] (e) [label=left:{$e$}, below=8mm of b] {};
            \node[dot] (f) [label=right:{$f$}, right=8mm of e] {};
            \node[dot] (i) [label=left:{$i$}, below=12mm of e] {};
            \node[dot] (j) [label=right:{$j$}, right=8mm of i] {};
            \node[dot] (m) [label={[text height=1.5ex]below:{$m$}}, below=8mm of i] {};
            \draw (a) -- (d) -- (h) -- (k) -- (g) -- (c) -- (a);
            \draw (b) -- (e) -- (i) -- (m) -- (j) -- (f) -- (b);
            \draw (a) -- (b);
            \draw (k) -- (m);
            \draw (c) -- (d);
            \draw (g) -- (h);
            \draw (e) -- (f);
            \draw (i) -- (j);
        }

        \item \tikz[mygraphstyle, baseline=(a.south)]{
            \node[dot] (a) [label=above:{$a$}] {};
            \node[dot] (c) [label=above:{$c$}, right=20mm of a] {};
            \node[dot] (g) [label=above left:{$g$}, right=10mm of c] {};
            \node[dot] (e) [label=above:{$e$}, right=20mm of g] {};
            \node[dot] (b) [label=right:{$b$}, above right=10mm and 10mm of a] {};
            \node[dot] (h) [label=right:{$h$}, below right=10mm and 10mm of a] {};
            \node[dot] (d) [label=right:{$d$}, above right=10mm and 10mm of g] {};
            \node[dot] (f) [label=right:{$f$}, below right=10mm and 10mm of g] {};
            \draw (a) -- (b) -- (c) -- (h) -- (a);
            \draw (b) -- (h);
            \draw (a) -- (c) -- (g) -- (e);
            \draw (g) -- (d) -- (e) -- (f) -- (g);
            \draw (d) -- (f);
            \draw (c) edge[out=45,in=180] (d);
            \draw (c) edge[out=-45,in=180] (f);
        }

        \item \tikz[mygraphstyle, baseline=(b.south)]{
            \coordinate (zero);
            \node[dot] (a) [label=above:{$a$}, position=90:14mm from zero] {};
            \node[dot] (d) [label=left:{$d$}, position=210:14mm from zero] {};
            \node[dot] (f) [label=right:{$f$}, position=-30:14mm from zero] {};
            \node[dot] (b) [label=left:{$b$}, position=150:{7mm-1pt} from zero] {};
            \node[dot] (c) [label=right:{$c$}, position=30:{7mm-1pt} from zero] {};
            \node[dot] (e) [label=below:{$e$}, position=270:{7mm-1pt} from zero] {};
            \draw (zero) circle (14mm+2pt);
            \draw (a) -- (b) -- (c) -- (e) -- (b) -- (d) -- (e) -- (f) -- (c) -- (a);
        }
    \end{subtasks}
    \end{multicols}


    \item A \textbf{precedence graph} is a directed graph where the vertices represent the program instructions and the edges represent the dependencies between instructions: there is an edge from one statement to a second statement if the second statement cannot be executed before the first statement.

    For~example, the instruction $b \coloneqq a + 1$ depends on the instruction $a \coloneqq 0$, so there would be an edge from the statement $S_1 = (a \coloneqq 0)$ to the statement $S_2 = (b \coloneqq a + 1)$.

    Construct a precedence graph for the following program:
    \par\hspace{2em}$S_1 :~ x \coloneqq 0$
    \par\hspace{2em}$S_3 :~ y \coloneqq 2$
    \par\hspace{2em}$S_2 :~ x \coloneqq x + 1$
    \par\hspace{2em}$S_4 :~ z \coloneqq y$
    \par\hspace{2em}$S_5 :~ x \coloneqq x + 2$
    \par\hspace{2em}$S_6 :~ y \coloneqq x + z$
    \par\hspace{2em}$S_7 :~ z \coloneqq 4$


    \item\label{task:weighted-graph} Find a shortest path between $a$ and $z$ in the given graph.

    \tikz[mygraphstyle] {
        \node[dot] (a) [label=left:{$a$}] {};
        \node[dot] (c) [label=below:{$c$}, right=12mm of a] {};
        \node[dot] (b) [label=above:{$b$}, above=12mm of c] {};
        \node[dot] (e) [label=above:{$e$}, right=18mm of b] {};
        \node[dot] (f) [label=above:{$f$}, below=12mm of e] {};
        \node[dot] (d) [label=below:{$d$}, below=12mm of c] {};
        \node[dot] (g) [label=below:{$g$}, below=12mm of f] {};
        \node[dot] (h) [label=above:{$h$}, above right=6mm and 18mm of e] {};
        \node[dot] (i) [label=above:{$i$}, above right=6mm and 18mm of f] {};
        \node[dot] (j) [label={[xshift=-2pt]below right:{$j$}}, below=12mm of i] {};
        \node[dot] (k) [label=below:{$k$}, below=12mm of j] {};
        \node[dot] (l) [label=above:{$l$}, below right=8mm and 18mm of h] {};
        \node[dot] (m) [label={[xshift=4pt]above left:{$m$}}, below=10mm of l] {};
        \node[dot] (n) [label={[xshift=3pt]above left:{$n$}}, below=10mm of m] {};
        \node[dot] (o) [label=above:{$o$}, right=36mm of h] {};
        \node[dot] (p) [label={[xshift=-4pt]above right:{$p$}}, below=12mm of o] {};
        \node[dot] (q) [label={[xshift=-4pt]above right:{$q$}}, below=12mm of p] {};
        \node[dot] (r) [label=below:{$r$}, below=12mm of q] {};
        \node[dot] (s) [label={[xshift=-2pt,yshift=-2pt]above right:{$s$}}, above right=6mm and 12mm of p] {};
        \node[dot] (t) [label={[xshift=-2pt,yshift=2pt]below right:{$t$}}, right=12mm of q] {};
        \node[dot] (z) [label=right:{$z$}, above right=9mm and 9mm of t] {};
        \draw (a) --node[sloped,above]{2} (b);
        \draw (a) --node[above,pos=0.7]{4} (c);
        \draw (a) --node[sloped,below]{1} (d);
        \draw (b) --node[above]{1} (e);
        \draw (b) --node[right,xshift=-2pt]{3} (c);
        \draw (c) --node[sloped,above]{2} (e);
        \draw (c) --node[above,pos=0.7]{2} (f);
        \draw (d) --node[below]{4} (g);
        \draw (d) --node[sloped,above]{5} (f);
        \draw (e) --node[sloped,above]{3} (h);
        \draw (f) --node[sloped,above,pos=0.4]{3} (h);
        \draw (f) --node[sloped,above,pos=0.6]{2} (i);
        \draw (f) --node[sloped,above]{4} (j);
        \draw (f) --node[right]{3} (g);
        \draw (g) --node[sloped,below]{2} (k);
        \draw (i) --node[right]{3} (j);
        \draw (i) --node[sloped,above]{3} (l);
        \draw (j) --node[left]{6} (k);
        \draw (k) --node[sloped,above]{4} (n);
        \draw (h) --node[sloped,above,pos=0.6]{1} (l);
        \draw (h) --node[above]{8} (o);
        \draw (l) --node[sloped,above,pos=0.4]{6} (o);
        \draw (m) --node[sloped,above]{4} (o);
        \draw (i) --node[sloped,above]{2} (m);
        \draw (i) --node[sloped,above]{2} (m);
        \draw (j) --node[sloped,above]{6} (m);
        \draw (j) --node[sloped,above]{3} (n);
        \draw (k) --node[sloped,above]{4} (n);
        \draw (k) --node[below]{2} (r);
        \draw (m) --node[right]{5} (n);
        \draw (m) --node[left,pos=0.75]{3} (l);
        \draw (n) --node[sloped,above]{2} (q);
        \draw (m) --node[sloped,above,pos=0.6]{2} (p);
        \draw (n) --node[sloped,above]{1} (r);
        \draw (o) --node[left]{2} (p);
        \draw (o) --node[sloped,above]{6} (s);
        \draw (p) --node[left]{1} (q);
        \draw (p) --node[sloped,above]{2} (s);
        \draw (p) --node[sloped,above]{1} (t);
        \draw (q) --node[right,pos=0.4]{8} (r);
        \draw (q) --node[sloped,above]{3} (t);
        \draw (r) --node[sloped,below]{5} (t);
        \draw (t) --node[sloped,below]{8} (z);
        \draw (s) --node[sloped,above]{2} (z);
    }


    \item Imagine that you have a three-liter jar and another five-liter jar.
    You can fill any jar with water, empty any jar, or transfer water from one jar to the other.
    Use a directed graph to demonstrate how you can end up with a jar containing exactly one litre of water.


    \item Draw $K_5$ and $K_{3,3}$ on the surface of a torus (a donut-shaped solid) without intersecting edges.


\newpage

    \item Floyd's algorithm (pseudocode given below) can be used to find the shortest path between any two vertices in a weighted connected simple graph.

    \begin{subtasks}
        \item Implement Floyd's algorithm in your favorite programming language and use it to find the distance between all pairs of vertices in the weighted graph given in task~\ref{task:weighted-graph}.
        \item Prove that Floyd's algorithm determines the shortest distance between all pairs of vertices in a weighted simple graph.
        \item Explain in detail (with examples and illustrations) the behavior of the Floyd's algorithm on a graph with negative cycles (a \emph{negative cycle} is a cycle whose edge weights sum to a negative value).
        \item Give a big-$\mathcal{O}$ estimate of the number of operations (comparisons and additions) used by Floyd's algorithm to determine the shortest distance between every pair of vertices in a weighted simple graph with $n$ vertices.
        \item Modify the algorithm to output the actual shortest path between any two given vertices, not just the distance between them.
    \end{subtasks}

    \begin{algorithm}
        \caption{Floyd's algorithm}
        \DontPrintSemicolon
        \KwData{weighted simple graph $G = \Pair{V, E, w}$ with vertices $V = \Set{v_1, \dots, v_n}$ and weights~$w(v_i, v_j)$, where $w(v_i, v_j) = \infty$ if $\Pair{v_i, v_j} \notin E$.}
        \KwResult{$d(v_i, v_j)$ is the length of a shortest path between $v_i$ and $v_j$.}
        \For{$i \coloneqq 1$ \KwTo $n$}{
            \For{$j \coloneqq 1$ \KwTo $n$}{
                $d(v_i, v_j) \coloneqq w(v_i, v_j)$ \;
            }
        }
        \For{$i \coloneqq 1$ \KwTo $n$}{
            \For{$j \coloneqq 1$ \KwTo $n$}{
                \For{$k \coloneqq 1$ \KwTo $n$}{
                    \If{$d(v_j, v_i) + d(v_i, v_k) < d(v_j, v_k)$}{
                        $d(v_j, v_k) \coloneqq d(v_j, v_i) + d(v_i, v_k)$ \;
                    }
                }
            }
        }
    \end{algorithm}


    \item\label{task:graceful-graphs} A tree with $n$ vertices is called \textbf{graceful} if its vertices can be labeled with the integers $1, 2, \dots, n$ in such a way that the absolute values of the difference of the labels of adjacent vertices are all different.
    Show that the following graphs are graceful.

    \begin{multicols}{2}
    \begin{subtasks}
        \item \tikz[mygraphstyle, baseline=-2pt]{
            \draw    (0,0) node[dot] {}
                -- ++(1,0) node[dot] {}
                -- ++(1,0) node[dot] {}
                -- ++(1,0) node[dot] {};
        }

        \item \tikz[mygraphstyle, baseline=-2pt]{
            \draw    (0,0) node[dot] {}
                -- ++(1,0) node[dot] {}
                -- ++(1,0) node[dot] (x) {}
                -- ++(1,0) node[dot] {}
                -- ++(1,0) node[dot] {};
            \draw (x) -- ++(0,-1) node[dot] {}
                      -- ++(0,-1) node[dot] {};
        }

        \item \tikz[mygraphstyle, baseline=-2pt]{
            \draw    (0,0) node[dot] {}
                -- ++(1,0) node[dot] (x) {}
                -- ++(1,0) node[dot] {}
                -- ++(1,0) node[dot] {};
            \draw (x) -- ++(0,-1) node[dot] {};
        }

        \item \tikz[mygraphstyle, baseline=-2pt]{
            \draw    (0,0) node[dot] {}
                -- ++(1,0) node[dot] (x) {}
                -- ++(1,0) node[dot] (y) {}
                -- ++(1,0) node[dot] {};
            \draw (x) -- ++(0,-1) node[dot] {};
            \draw (y) -- ++(0,-1) node[dot] {};
        }
    \end{subtasks}
    \end{multicols}


    \item A \textbf{caterpillar} is a tree that contains a simple path such that every vertex not contained in this path is adjacent to a vertex in the path.

    \begin{subtasks}
        \item Which of the graphs in task~\ref{task:graceful-graphs} are caterpillars?
        \item How many non-isomorphic caterpillars are there with six vertices?
        \item Prove or disprove that all caterpillars are graceful.
    \end{subtasks}


    \item Draw all pairwise non-isomorphic unlabeled unrooted trees on 6 vertices.


    \item Consider the following algorithm (let's call it \enquote{Algorithm S}) for finding a minimum spanning tree from a connected weighted simple graph $G = \Pair{V, E}$ by successively adding groups of edges.
    Suppose that the vertices in $V$ are ordered.
    Consider the lexicographic order on edges $\Pair{u, v} \in E$ with $u \prec v$.
    An edge $\Pair{u_1, v_1}$ precedes $\Pair{u_2, v_2}$ if $u_1$ precedes $u_2$ or if $u_1 = u_1$ and $v_1$ precedes $v_2$.

    The algorithm~S begins by simultaneously choosing the edge of least weight incident to each vertex.
    The first edge in the ordering is taken in the case of ties.
    This produces (you are going to prove it) a graph with no simple circuits, that is, a forest of trees.
    Next, simultaneously choose for each tree in the forest the shortest edge between a vertex in this tree and a vertex in a different tree.
    Again, the first edge in the ordering is chosen in the case of ties.
    This produces an acyclic graph containing fewer trees than before this step.
    Continue the process of simultaneously adding edges connecting trees until $n - 1$ edges have been chosen.
    At this stage a minimum spanning tree has been constructed.

    \begin{subtasks}
        \item Show that the addition of edge at each stage of algorithm S produces a forest.
        \item Express algorithm S in pseudocode.
        \item Use algorithm S to produce a minimum spanning tree for the weighted graph given in task~\ref{task:weighted-graph}.
    \end{subtasks}


    \item The \textbf{density} of an undirected graph $G$ is the number of edges
    of $G$ divided by the number of possible edges in an undirected
    graph with $|G|$ vertices.
    That is, the density of $G = \Pair{V, E}$ is $\frac{2 |E|}{|V| (|V| - 1)}$.
    A family of graphs $G_n$, $n = 1, 2, \dots$ is \textbf{sparse} if the limit of the density of $G_n$ is zero as $n$ grows without bound, while it is \textbf{dense} if this proportion approaches a positive real number.

    For each of these families of graphs, determine whether it is sparse, dense, or neither.

    \begin{multicols}{3}
    \begin{subtasks}
        \item $K_n$ (complete graph\Href{https://en.wikipedia.org/wiki/Complete_graph})
        \item $C_n$ (cycle graph\Href{https://en.wikipedia.org/wiki/Cycle_graph})
        \item $K_{n,n}$ (complete bipartite\Href{https://en.wikipedia.org/wiki/Complete_bipartite_graph})
        \item $K_{3,n}$ (complete bipartite\Href{https://en.wikipedia.org/wiki/Complete_bipartite_graph})
        \item $Q_n$ (hypercube graph\Href{https://en.wikipedia.org/wiki/Hypercube_graph})
        \item $W_n$ (wheel graph\Href{https://en.wikipedia.org/wiki/Wheel_graph})
    \end{subtasks}
    \end{multicols}


    \item Consider a graph of subway lines in Saint Petersburg in 2050\Href{https://web.archive.org/web/20240301072159/https://commons.wikimedia.org/wiki/File:Saint_Petersburg_metro_future_map_RUS.png}.
    Smoothen out all vertices, i.e. retain only transfer and final stations.
    Find all cut vertices and bridges, blocks and islands, maximum stable set, maximum matching, minimum dominating set, minimum vertex and edge covers.


    \item Find an error in the following inductive \enquote{proof} of the statement that every tree with $n$ vertices has a path of length $n - 1$.

    $\triangleright$
    Base: A tree with one vertex clearly has a path of length~0.
    Inductive step: Assume that a tree with $n$~vertices has a path of length $n - 1$, which has $u$ as its terminal vertex.
    Add a vertex~$v$ and the edge from $u$ to~$v$.
    The resulting tree has $n + 1$ vertices and has a path of length~$n$.
    \qed


    \item Prove \emph{rigorously} the following theorems:

    \begin{theorem}[Triangle Inequality]
        For any connected graph $G = \Pair{V, E}$:
        \[
            \forall x,y,z \in V : \dist{x,y} + \dist{y,z} \geq \dist{x,z}
        \]
    \end{theorem}

    \begin{theorem}
        Any connected graph $G$ has $\graphRadius{G} \leq \graphDiameter{G} \leq 2 \graphRadius{G}$.
    \end{theorem}

    % \begin{theorem}
    %     Every $r$-regular ($r > 0$) $(n,m)$-bipartite graph has $n = m$.
    % \end{theorem}

    \begin{theorem}[Tree]
        A connected graph $G = \Pair{V, E}$ is a tree (\ie acyclic graph) \emph{iff} $\card{E} = \card{V} - 1$.
    \end{theorem}

    % \begin{theorem}[Hall]
    %     For any bipartite graph $G = (X, Y, E)$, there exists $X$-perfect matching (set of disjoint edges covering all vertices in~$X$) \emph{iff} $\card{N(W)} \geq \card{W}$ for every $W \subseteq X$.
    % \end{theorem}

    \begin{theorem}[Whitney]
        For any graph $G$: $\vertexConnectivity{G} \leq \edgeConnectivity{G} \leq \minDegree{G}$.
    \end{theorem}

    \begin{theorem}[Chartrand]
        For a connected graph $G = \Pair{V, E}$: if $\minDegree{G} \geq \Floor{\card{V} / 2}$, then $\edgeConnectivity{G} = \minDegree{G}$.
    \end{theorem}

    % \begin{theorem}[Menger]
    %     For any pair of non-adjacent vertices $u$ and~$v$ in an undirected graph, the size of the minimum vertex cut is equal to the maximum number of pairwise internally vertex-disjoint paths from $u$ to~$v$.
    % \end{theorem}

    \begin{theorem}[Harary]
        Every block of a block graph\footnote{A block graph $H = \blockGraph{G}$ is an intersection graph of all blocks (biconnected components) of $G$, \ie each vertex $v \in V(H)$ corresponds to a block of $G$, and there is an edge $\Set{v,u} \in E(H)$ iff \enquote{blocks} $v$ and $u$ share a cut vertex.} is a clique.
    \end{theorem}

\end{tasks}
\end{document}
