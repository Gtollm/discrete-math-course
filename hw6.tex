\documentclass[a4paper,12pt]{article}
\usepackage{mypreamble}

%% Page setup
\geometry{
    margin=2cm,
    includehead,
    % includefoot,
    headsep=\baselineskip,
}
\pagestyle{fancy}
\fancyfoot{}
\MakeDoubleHeader% {<l1>}{<l2>}{<r1>}{<r2>}
    {\TextHomeworkEng~\#6}
    {Combinatorics}
    {\TextDiscreteMathEng}
    {\IconSpring~Spring 2022}

%% Add custom setup below

\newcommand*{\chancery}{\fontfamily{pzc}\selectfont}
\DeclareTextFontCommand{\textpzc}{\chancery}

\usepackage{epigraph}
\usepackage{varwidth}

\renewcommand{\epigraphsize}{\small}
\setlength{\beforeepigraphskip}{0pt}
\setlength{\afterepigraphskip}{-0.5\baselineskip}
\setlength{\epigraphwidth}{0.6\textwidth}% Maximum epigraph width
\setlength{\epigraphrule}{0.4pt}
\renewcommand{\textflush}{flushright}
\renewcommand{\sourceflush}{flushright}

% https://tex.stackexchange.com/a/96717/216486
\makeatletter
\newsavebox{\epi@textbox}
\newsavebox{\epi@sourcebox}
\newlength\epi@finalwidth
\renewcommand{\epigraph}[2]{%
  \vspace{\beforeepigraphskip}
  {\epigraphsize\begin{\epigraphflush}
  \epi@finalwidth=\z@
  \sbox\epi@textbox{%
    \varwidth{\epigraphwidth}
    \begin{\textflush}#1\end{\textflush}
    \endvarwidth
  }%
  \epi@finalwidth=\wd\epi@textbox
  \sbox\epi@sourcebox{%
    \varwidth{\epigraphwidth}
    \begin{\sourceflush}#2\end{\sourceflush}%
    \endvarwidth
  }%
  \ifdim\wd\epi@sourcebox>\epi@finalwidth
     \epi@finalwidth=\wd\epi@sourcebox
  \fi
  \leavevmode\vbox{
    \hb@xt@\epi@finalwidth{\hfil\box\epi@textbox}
    \vskip1.25ex
    \hrule height \epigraphrule
    \vskip.75ex
    \hb@xt@\epi@finalwidth{\hfil\box\epi@sourcebox}
  }%
  \end{\epigraphflush}
  \vspace{\afterepigraphskip}}}
\makeatother

\tikzset{
    position/.style args={#1:#2 from #3}{
        at=(#3.#1), anchor=#1+180, shift=(#1:#2)
    },
}

\usepackage{stringstrings}


\begin{document}
\selectlanguage{english}

\epigraph{\textpzc{Do whatever you want, but always explain what you are doing.}}{--- \textsc{Konstantin}, 2020}

\begin{tasks}
    \item Find the number of different 5-digit numbers using digits 1--9 under the given constraints.
    For each case, provide representative examples of (non-)complying numbers (\eg 12345 and 52814 are suitable for~(b), but 44521 and 935 are~not) and derive a generic\footnote{Here, \enquote{generic formula} means \enquote{depending on the input data}. In this particular example, $n = 9$ and $k = 5$, but the sought formula must also be valid for all other (adequate) values of~$n$ and~$k$.} formula.
    Try to express the formula using standard combinatorial terms, \eg $k$-combs $C_n^k$ and $k$-perms $P(n,k)$.
    \begin{subtasks}
        \item Digits \emph{can} be repeated.

        \item Digits \emph{cannot} be repeated.

        \item Digits \emph{can} be repeated and must be written in \emph{non-increasing}\footnote{A sequence $(x_n)$ is said to be \emph{strictly monotonically increasing} if each term is \emph{strictly greater} than the previous one, \ie $x_{i} < x_{i+1}$. A sequence $(x_n)$ is called \emph{non-increasing} if each term is \emph{less than or equal} to the previous one, \ie $x_{i} \geq x_{i+1}$.} order.

        \item Digits \emph{cannot} be repeated and must be written in \emph{strictly increasing} order.

        \item Digits \emph{can} be repeated, must be written in \emph{non-decreasing} order, and the 4th digit must be 6.
    \end{subtasks}


\tikzstyle{myboxstyle}=[
    scale=0.95, transform shape,
    baseline=4pt, % why? maybe in order to maintain depth
    % baseline=(box-label.base),
    myoutline/.style={
        thick,
        line join=bevel,
    },
    myback/.style={
        myoutline,
        fill=black!50,
    },
    myfront/.style={
        myoutline,
        fill=black!20,
        fill opacity=0.75,
    },
    myball/.style={
        ball color=white,
    },
]

\def\boxWidth{0.9}
\def\boxHeight{0.4}
\def\boxDepth{0.4}
\def\boxLeftX{0.16}
\def\boxLeftY{0.08}
\def\boxRightX{0.2}
\def\boxRightY{0.1}
\def\ballRadius{0.22}

\newcommand{\drawBoxBack}{
    %% Back
    \draw[myback] (0,0)
    -- ++(0,\boxHeight)
    -- ++(\boxDepth,\boxDepth)
    -- ++(\boxWidth,0)
    -- ++(0,-\boxHeight)
    -- ++(-\boxDepth,-\boxDepth)
    -- cycle;
    \draw[myoutline] (0,0)
    -- ++(\boxDepth,\boxDepth)
    -- ++(0,\boxHeight) ++(0,-\boxHeight)
    -- ++(\boxWidth,0);

    %% Left thing
    \draw[myfront,fill opacity=1] (0,\boxHeight)
    -- ++(\boxDepth,\boxDepth)
    -- ++(-\boxLeftX,\boxLeftY)
    -- ++(-\boxDepth,-\boxDepth)
    -- cycle;
}

\newcommand{\drawBoxFront}[1]{% {<label>}
    %% Front
    \draw[myfront] (0,0)
    -- ++(\boxWidth,0)
    -- ++(\boxDepth,\boxDepth)
    -- ++(0,\boxHeight)
    -- ++(-\boxDepth,-\boxDepth)
    -- ++(-\boxWidth,0)
    -- cycle;
    \draw[myoutline] (\boxWidth,0)
    -- ++(0,\boxHeight);

    %% Right thing
    \draw[myfront] (\boxWidth,\boxHeight)
    -- ++(\boxDepth,\boxDepth)
    -- ++(\boxRightX,\boxRightY)
    -- ++(-\boxDepth,-\boxDepth)
    -- cycle;

    %% Label
    \node (box-label) at (\boxWidth/2,\boxHeight/2) {\vphantom{12345ABCDE}#1};
}

\newcommand{\drawBall}[3][]{% [<style>]{<pos>}{<label>}
    \draw[myball,#1] (#2) circle [radius=\ballRadius] node {#3};}
\newcommand{\drawBallOne}[2][]{% [<style>]{<label>}
    \drawBall[#1]{ 0.52, 0.55 }{#2}}
\newcommand{\drawBallTwo}[2][]{% [<style>]{<label>}
    \drawBall[#1]{ 0.88, 0.51 }{#2}}
\newcommand{\drawBallThree}[2][]{% [<style>]{<label>}
    \drawBall[#1]{ 0.31, 0.33 }{#2}}
\newcommand{\drawBallFour}[2][]{% [<style>]{<label>}
    \drawBall[#1]{ 0.71, 0.3 }{#2}}
\newcommand{\drawBallCenter}[2][]{% [<style>]{<label>}
    \drawBall[#1]{ 0.62, 0.38 }{#2}}
\newcommand{\drawBallLowCenter}[2][]{% [<style>]{<label>}
    \drawBall[#1]{ 0.54, 0.31 }{#2}}

\newcommand{\drawBox}[3][]{% [<style>]{<box-label>}{<body>}
    \begin{tikzpicture}[myboxstyle,scale=1.5,transform shape,#1]
        \drawBoxBack
        #3
        \drawBoxFront{#2}
    \end{tikzpicture}%
}

\newcounter{balls}
\newcommand{\newBall}{\stepcounter{balls}\theballs}
\newcommand{\resetBalls}{\setcounter{balls}{0}}
\renewcommand{\theballs}{\arabic{balls}} % Use this to redefine counter format

\newcounter{boxes}
\newcommand{\newBox}{\stepcounter{boxes}\theboxes}
\newcommand{\resetBoxes}{\setcounter{boxes}{0}}
\renewcommand{\theboxes}{\Alph{boxes}} % Use this to redefine counter format

\newcommand{\drawBoxWithoutBalls}[1][]{% [<style>]
    \drawBox[#1]{\newBox}{}}
\newcommand{\drawBoxWithOneBall}[1][]{% [<style>]
    \drawBox[#1]{\newBox}{
        \drawBallCenter{\newBall}
    }}
\newcommand{\drawBoxWithTwoBalls}[1][]{% [<style>]
    \drawBox[#1]{\newBox}{
        % \drawBallOne{\newBall}
        % \drawBallFour{\newBall}
        \drawBallThree{\newBall}
        \drawBallTwo{\newBall}
    }}
\newcommand{\drawBoxWithThreeBalls}[1][]{% [<style>]
    \drawBox[#1]{\newBox}{
        \drawBallOne{\newBall}
        \drawBallTwo{\newBall}
        % \drawBallThree{\newBall}
        \drawBallLowCenter{\newBall}
    }}
\newcommand{\drawBoxWithFourBalls}[1][]{% [<style>]
    \drawBox[#1]{\newBox}{
        \drawBallOne{\newBall}
        \drawBallTwo{\newBall}
        \drawBallThree{\newBall}
        \drawBallFour{\newBall}
    }}


    \item One of the classical combinatorial problems is counting the number of arrangements of $n$ balls into $k$ boxes.
    There are at least \href{https://en.wikipedia.org/wiki/Twelvefold_way}{12 variations} of this problem: four cases (a--d) with three different constraints (1--3).
    For each problem (case+constraint), derive the corresponding generic formula.
    Additionally, pick several (representative) values for~$n$ and~$k$ and use your derived formulae to find the numbers of arrangements.
    Visualize several possible arrangements for the chosen~$n$ and~$k$.

    \vspace{2pt}
    \textit{\uline{Cases with arrangement examples}}:

    \begin{enumerate}[label=\alph*., itemsep=4pt]
        %% U -> L
        \item $\textbf{U} \to \textbf{L}$: Balls are \textbf{U}nlabeled, Boxes are \textbf{L}abeled.

        \resetBalls
        \resetBoxes
        \renewcommand{\theballs}{}
        \renewcommand{\theboxes}{\Alph{boxes}}
        \drawBoxWithThreeBalls
        \drawBoxWithoutBalls
        \drawBoxWithFourBalls
        \drawBoxWithTwoBalls
        \drawBoxWithOneBall
        % ($n = \arabic{balls}$, $k = \arabic{boxes}$)

        %% L -> U
        \item $\textbf{L} \to \textbf{U}$: Balls are \textbf{L}abeled, Boxes are \textbf{U}nlabeled.

        \resetBoxes
        \resetBalls
        \renewcommand{\theballs}{\arabic{balls}}
        \renewcommand{\theboxes}{}
        \drawBoxWithOneBall
        \drawBoxWithTwoBalls
        \drawBoxWithoutBalls
        \drawBoxWithFourBalls
        \drawBoxWithThreeBalls
        % ($n = \arabic{balls}$, $k = \arabic{boxes}$)

        %% L -> L
        \item $\textbf{L} \to \textbf{L}$: Balls are \textbf{L}abeled, Boxes are \textbf{L}abeled.

        \resetBoxes
        \resetBalls
        \renewcommand{\theballs}{\arabic{balls}}
        \renewcommand{\theboxes}{\Alph{boxes}}
        \drawBoxWithTwoBalls
        \drawBoxWithOneBall
        \drawBoxWithThreeBalls
        \drawBoxWithoutBalls
        \drawBoxWithFourBalls
        % ($n = \arabic{balls}$, $k = \arabic{boxes}$)

        %% U -> U
        \item $\textbf{U} \to \textbf{U}$: Balls are \textbf{U}nlabeled, Boxes are \textbf{U}nlabeled.

        \resetBoxes
        \resetBalls
        \renewcommand{\theballs}{}
        \renewcommand{\theboxes}{}
        \drawBoxWithFourBalls
        \drawBoxWithThreeBalls
        \drawBoxWithTwoBalls
        \drawBoxWithOneBall
        \drawBoxWithoutBalls
        % ($n = \arabic{balls}$, $k = \arabic{boxes}$)
    \end{enumerate}

    \vspace{2pt}
    \textit{\uline{Constraints}}:

    \begin{enumerate}[label=\arabic*., noitemsep]
        \item $\leq 1$ ball per box \--- \emph{injective} mapping.
        \item $\geq 1$ ball per box \--- \emph{surjective} mapping.
        \item Arbitrary number of balls per box.
    \end{enumerate}

    % \vspace{2pt}
    \textit{\uline{Notes}}:

    \begin{itemize}[label=$\ast$, noitemsep]
        \item \textbf{U}nlabeled means \enquote{indistinguishable}, and \textbf{L}abeled means \enquote{distinguishable}.

        \item \href{https://en.wikipedia.org/wiki/Stirling_numbers_of_the_second_kind}{Stirling number of the second kind} $s^{II}_k(n)$ \--- number of ways to partition a set of~$n$ elements into~$k$ non-empty subsets.
        Use $s^{II}_k(n)$ directly without expanding the closed formula.

        \item \href{https://en.wikipedia.org/wiki/Partition_(number_theory)#Restricted_part_size_or_number_of_parts}{Partition function} $p_k(n)$ \--- number of ways to partition the integer $n$ into~$k$ positive parts, \ie $n = a_1 + \dotsb + a_k$, where $a_1 \geq \dotsb \geq a_k \geq 1$.
        Use $p_k(n)$ directly.
        % Use $p_k(n)$ directly, since the closed-form expression is unknown.
    \end{itemize}


    \item Proof the Generalized Pascal's Formula (for $n \geq 1$ and $k_1,\dotsc,k_r \geq 0$ with $k_1 + \dotsb + k_r = n$):
    \[
        \binom{n}{k_1,\dotsc,k_r} = \sum_{i=1}^{r} \binom{n-1}{k_1,\dotsc,k_i-1,\dotsc,k_r}
    \]

    Count the number of permutations of a multiset $\Sigma^{*} = \Set{2 \cdot \triangle, 3 \cdot \square, 1 \cdot \Cat}$ using GPF.


\tikzstyle{myrnastyle}=[
    scale=0.8, transform shape,
    dot/.style={
        draw,
        fill=black,
        shape=circle,
        minimum size=4pt,
        inner sep=0pt,
        outer sep=0pt,
    },
    matching/.style={
        draw,
        % dashed,
        red,
        ultra thick,
    },
    basepair/.style={
        draw,
        dashed,
        lightgray,
        thick,
    },
]

\newcommand{\drawRNA}{%
    \def\myRadNode{1.6}
    \def\myRadLabel{2.0}

    % Outside edges
    % \draw (75:\myRadNode) arc (75:-255:\myRadNode);

    % Nodes
    \foreach [count=\i, evaluate=\x as \a using {105-\i*30}] \x in {A,U,C,G,U,A,A,U,C,G,C,G}
    {
        \node[dot] (v\i) at (\a:\myRadNode) {};
        \node at (\a:\myRadLabel) {\x};
        % \node at (\a:1.3) {\i}; % DEBUG index
    }
}


    \item A \emph{non-crossing perfect matching}\footnote{Credits to \href{https://rosalind.info/about}{Rosalind} for this task.} in a graph is a set of pairwise disjoint edges that cover all vertices and do not intersect with each other.
    For example, consider a graph on~$2n$ vertices numbered from~1~to~$2n$ and arranged in a circle.
    Additionally, assume that edges are straight lines.
    In~this~case, edges $\Set{i,j}$ and $\Set{a,b}$ intersect whenever $i < a < j < b$.

    \begin{subtasks}
        \item Count the number of all possible non-crossing perfect matchings in a complete graph~$K_{2n}$.

        \item Consider a graph on vertices labeled with letters from $\Set{\texttt{A}, \texttt{C}, \texttt{G}, \texttt{U}}$.
        Each pair of vertices labeled with \texttt{A} and \texttt{U} is connected with a \emph{basepair edge}.
        Similarly, \texttt{C}--\texttt{G} pairs are also connected.

        The picture below illustrates some of possible non-crossing perfect matchings in the graph with 12 vertices \texttt{AUCGUAAUCGCG} arranged in a circle.
        Basepair edges are drawn dashed gray, matching is red.

        % Matching 1
        \begin{tikzpicture}[myrnastyle]
            \drawRNA
            \draw[matching] (v1) -- (v2);
            \draw[basepair] (v1) -- (v5);
            \draw[basepair] (v1) -- (v8);
            \draw[basepair] (v2) -- (v6);
            \draw[basepair] (v2) -- (v7);
            \draw[matching] (v3) -- (v4);
            \draw[basepair] (v3) -- (v10);
            \draw[basepair] (v3) -- (v12);
            \draw[basepair] (v4) -- (v9);
            \draw[basepair] (v4) -- (v11);
            \draw[matching] (v5) -- (v6);
            \draw[basepair] (v5) -- (v7);
            \draw[basepair] (v6) -- (v8);
            \draw[matching] (v7) -- (v8);
            \draw[basepair] (v9) -- (v10);
            \draw[matching] (v9) -- (v12);
            \draw[matching] (v10)-- (v11);
            \draw[basepair] (v11)-- (v12);
        \end{tikzpicture}%
        \hfill%
        % Matching 2
        \begin{tikzpicture}[myrnastyle]
            \drawRNA
            \draw[basepair] (v1) -- (v2);
            \draw[basepair] (v1) -- (v5);
            \draw[matching] (v1) -- (v8);
            \draw[basepair] (v2) -- (v6);
            \draw[matching] (v2) -- (v7);
            \draw[matching] (v3) -- (v4);
            \draw[basepair] (v3) -- (v10);
            \draw[basepair] (v3) -- (v12);
            \draw[basepair] (v4) -- (v9);
            \draw[basepair] (v4) -- (v11);
            \draw[matching] (v5) -- (v6);
            \draw[basepair] (v5) -- (v7);
            \draw[basepair] (v6) -- (v8);
            \draw[basepair] (v7) -- (v8);
            \draw[basepair] (v9) -- (v10);
            \draw[matching] (v9) -- (v12);
            \draw[matching] (v10)-- (v11);
            \draw[basepair] (v11)-- (v12);
        \end{tikzpicture}%
        \hfill%
        % Matching 3
        \begin{tikzpicture}[myrnastyle]
            \drawRNA
            \draw[matching] (v1) -- (v2);
            \draw[basepair] (v1) -- (v5);
            \draw[basepair] (v1) -- (v8);
            \draw[basepair] (v2) -- (v6);
            \draw[basepair] (v2) -- (v7);
            \draw[basepair] (v3) -- (v4);
            \draw[basepair] (v3) -- (v10);
            \draw[matching] (v3) -- (v12);
            \draw[basepair] (v4) -- (v9);
            \draw[matching] (v4) -- (v11);
            \draw[matching] (v5) -- (v6);
            \draw[basepair] (v5) -- (v7);
            \draw[basepair] (v6) -- (v8);
            \draw[matching] (v7) -- (v8);
            \draw[matching] (v9) -- (v10);
            \draw[basepair] (v9) -- (v12);
            \draw[basepair] (v10)-- (v11);
            \draw[basepair] (v11)-- (v12);
        \end{tikzpicture}%
        \hfill%
        % Matching 4
        \begin{tikzpicture}[myrnastyle]
            \drawRNA
            \draw[basepair] (v1) -- (v2);
            \draw[basepair] (v1) -- (v5);
            \draw[matching] (v1) -- (v8);
            \draw[basepair] (v2) -- (v6);
            \draw[matching] (v2) -- (v7);
            \draw[matching] (v3) -- (v4);
            \draw[basepair] (v3) -- (v10);
            \draw[basepair] (v3) -- (v12);
            \draw[basepair] (v4) -- (v9);
            \draw[basepair] (v4) -- (v11);
            \draw[matching] (v5) -- (v6);
            \draw[basepair] (v5) -- (v7);
            \draw[basepair] (v6) -- (v8);
            \draw[basepair] (v7) -- (v8);
            \draw[matching] (v9) -- (v10);
            \draw[basepair] (v9) -- (v12);
            \draw[basepair] (v10)-- (v11);
            \draw[matching] (v11)-- (v12);
        \end{tikzpicture}

        \def\myRNA{CGUAAUUACGGCAUUAGCAU}
        Count the number of all possible non-crossing perfect matchings in the graph on \stringlength{\myRNA} vertices labeled with \texttt{\myRNA}.
    \end{subtasks}

    % \item \ldots
\end{tasks}

\end{document}
